\section{Overview of the source code}

Here we will discuss general aspects of the source code, i.e. the files contained in the directory
\C{sources}.

\FORM\ is written in ANSI C. The code is split up in header files \C{*.h} and source files
\C{*.c}. Files usually don't come in pairs of a header file with the declarations and a source file
with the definitions, but instead most declarations are collected in a few headers. The declaration
of function headers is done in \C{declare.h} for example. The most prominent exceptions are
\C{parallel.h} and \C{minos.h}.

Each file usually contains many hundred lines of code. To make the files more accessible, the code
is structure by so--called folds. If you use the editor STedi, the code will be visualized
correctly. If you use a vi--compatible editor, it is advisable to activate folds and set the
foldmarkers to \C{set foldmarker=\#[,\#]}

% Folds in Emacs anybody??

\subsection{The header files}

% INDENTATION HACK to be improved!
$\quad\;\:$\begin{tabular}{p{0.2\textwidth}p{0.65\textwidth}}
\C{declare.h} & Contains the declaration of all publicly relevant functions and also some macros. \\
\C{form3.h} & Global settings and macro definitions like word sizes. It includes several different system
header files depending on the computer's architecture.\\
\C{fsizes.h} & Defines macros that determine the size and lauyout of \FORM's internal data like the
size of the work bugger etc. \\
\end{tabular}

\begin{tabular}{p{0.2\textwidth}p{0.65\textwidth}}
\C{ftypes.h} & Contains preprocessor definitions of codes used in the internal representation of
parsed input and expressions. \\
\C{fwin.h} & Special settings for the windows OS. \\
\C{inivar.h} & Contains the initialization of various global variables. \\
\end{tabular}

\begin{tabular}{p{0.2\textwidth}p{0.65\textwidth}}
\C{minos.h} &  Header to the minos.c source file. \\
\C{parallel.h} & Header to the parallel.c source file. \\
\C{portsignals.h} & Preprocessor definition of the OS signals \FORM\ can deal with. \\
\end{tabular}

\begin{tabular}{p{0.2\textwidth}p{0.65\textwidth}}
\C{structs.h} & Contains the global structs that contain almost all of \FORM's internal data. \\
\C{unix.h} & Special definitions for Unix--like OS. \\
\C{variable.h} & Some convinience preprocessor definitions to ease the access to global variables. \\
\end{tabular}

\subsection{The source files}

% INDENTATION HACK to be improved!
$\quad\;\:$\begin{tabular}{p{0.2\textwidth}p{0.65\textwidth}}
\C{argument.c} & TODO \\
\C{bugtool.c} & \ldots \\
\C{checkpoint.c} &  \\
\end{tabular}

\begin{tabular}{p{0.2\textwidth}p{0.65\textwidth}}
\C{comexpr.c} &  \\
\C{compcomm.c} &  \\
\C{compiler.c} &  \\
\end{tabular}

\begin{tabular}{p{0.2\textwidth}p{0.65\textwidth}}
\C{compress.c} &  \\
\C{comtool.c} &  \\
\C{dollar.c} &  \\
\end{tabular}

\begin{tabular}{p{0.2\textwidth}p{0.65\textwidth}}
\C{execute.c} &  \\
\C{extcmd.c} &  \\
\C{factor.c} &  \\
\end{tabular}

\begin{tabular}{p{0.2\textwidth}p{0.65\textwidth}}
\C{findpat.c} &  \\
\C{function.c} &  \\
\C{if.c} &  \\
\end{tabular}

\begin{tabular}{p{0.2\textwidth}p{0.65\textwidth}}
\C{index.c} &  \\
\C{lus.c} &  \\
\C{message.c} &  \\
\end{tabular}

\begin{tabular}{p{0.2\textwidth}p{0.65\textwidth}}
\C{minos.c} &  \\
\C{module.c} &  \\
\C{mpi2.c} &  \\
\end{tabular}

\begin{tabular}{p{0.2\textwidth}p{0.65\textwidth}}
\C{mpi.c} &  \\
\C{names.c} &  \\
\C{normal.c} &  \\
\end{tabular}

\begin{tabular}{p{0.2\textwidth}p{0.65\textwidth}}
\C{opera.c} &  \\
\C{optim.c} &  \\
\C{parallel.c} &  \\
\end{tabular}

\begin{tabular}{p{0.2\textwidth}p{0.65\textwidth}}
\C{pattern.c} &  \\
\C{poly.c} &  \\
\C{polynito.c} &  \\
\end{tabular}

\begin{tabular}{p{0.2\textwidth}p{0.65\textwidth}}
\C{pre.c} &  \\
\C{proces.c} &  \\
\C{ratio.c} &  \\
\end{tabular}

\begin{tabular}{p{0.2\textwidth}p{0.65\textwidth}}
\C{reken.c} &  \\
\C{reshuf.c} &  \\
\C{sch.c} &  \\
\end{tabular}

\begin{tabular}{p{0.2\textwidth}p{0.65\textwidth}}
\C{setfile.c} &  \\
\C{smart.c} &  \\
\C{sort.c} &  \\
\end{tabular}

\begin{tabular}{p{0.2\textwidth}p{0.65\textwidth}}
\C{startup.c} &  \\
\C{store.c} &  \\
\C{symmetr.c} &  \\
\end{tabular}

\begin{tabular}{p{0.2\textwidth}p{0.65\textwidth}}
\C{tables.c} &  \\
\C{threads.c} &  \\
\C{token.c} &  \\
\end{tabular}

\begin{tabular}{p{0.2\textwidth}p{0.65\textwidth}}
\C{tools.c} &  \\
\C{unixfile.c} &  \\
\C{wildcard.c} &
\end{tabular}

% mention groups of files relevant to certain tasks? E.g. all files concerned with preprocessor
% actions, all files concerned with sorting, etc.

\subsection{The global structs}

\FORM\ keeps its data in several global structs. These structs are defined in \C{structs.h}. The
various global variables contained in these structs are grouped according to their role in the
program. There exist the structs \C{P\_const}, \C{M\_const}, \ldots. The global variable \C{A}
defined in \C{inivar.h} is a struct containing several of these global structs. 

% A with and without threads, AB
% contents of ?\_const structs
% access macros to these structs
% initialization in startup.c
% data type preprocessor definition, e.g. WORD, to be found in form3.h and unix.h or fwin.h.
% ALLGLOBALS ALLPRIVATES
% structs like FILE, etc. also in structs.h

\subsection{Configuration and special files}

% config.h
% tform, parform, windows special files
