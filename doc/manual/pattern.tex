\chapter{Pattern matching}
\label{pattern}

Substitutions\index{substitutions}\index{pattern matching} are made in FORM 
by specifying a generic object that should be replaced by an expression. 
This generic object is called a pattern\index{pattern}. Patterns that the 
user may already be familiar with are the regular expressions in many 
UNIX\index{UNIX} based systems or just a statement like \verb:ls *.frm: to 
list only files of which the name ends in \verb:.frm:. In this case the 
\verb:*: is called a wildcard\index{wildcard} that can take any string 
value. In symbolic manipulation there will be wildcards also, but their 
nature will be different. They are also indicated in a different way.

In FORM wildcard variables are indicated by attaching a 
question\index{question mark} mark (?) to the name of a variable. The type 
of the variable indicates what type of object we are looking for. Assume 
the following id\index{id} statements:
\begin{verbatim}
    Functions f,g;
    Symbol x;

    id f(g?,x) = g(x,x);
\end{verbatim}
In this statement g will match any function and hence all occurrences 
of f, in which the first argument is a function and the second argument is 
the symbol x, will match. In the right hand side the function g will be 
substituted by whatever identity g had to assume in the left hand side to 
make the match. Hence \verb:f(f,x): will be replaced by \verb:f(x,x):.

In general function wildcards\index{wildcard!function} can only match 
functions. Even though tensors are special functions, regular function 
wildcards cannot match tensors, and tensor wildcards cannot match 
functions. However commuting\index{commuting} function wildcards can match 
noncommuting\index{noncommuting} functions {\sl et vise versa}.

Index\index{wildcard!index} wildcards can only match indices. The 
dimension of the indices is not relevant. Hence:
\begin{verbatim}
    id f(mu?,mu?) = 4;
\end{verbatim}
would match both \verb:f(ka,ka): and \verb:f(2,2):. We will see later 
how to be more selective about such matches.

When the same wildcard occurs more than once in a pattern, it should be 
matched by the same object in all its occurrences. Hence the above pattern 
would not match \verb:f(mu,nu):.

There is one complication concerning the above rule of index wildcards only 
matching indices. FORM writes contractions with vectors in a special 
shorthand notation called Schoonschip\index{Schoonschip} notation. Hence 
\verb:f(mu)*p(mu): becomes \verb:f(p):. This means that the substitution
\begin{verbatim}
    id f(mu?)*g(nu?) = fg(mu,nu);
\end{verbatim}
should also replace the term \verb:f(p)*g(q): by \verb:fg(p,q):. In this 
case it looks like the wildcard indices matched the vectors. This is 
however not the case, because if we take the previous pattern (with the 
\verb:f(mu?,mu?):), it is not going to match the term \verb:f(p,p):, 
because this term should be read as something of the type
\verb:f(mu,nu)*p(mu)*p(nu):
and that term does not fit the pattern \verb:f(mu?,mu?):.

Vector\index{wildcard!vector} wildcards can match vectors, but they can 
also match vector-like expressions in function arguments. A vector-like 
expression is an expression in which all terms contain one single vector 
without indices, possibly multiplied by other objects like coefficients, 
functions or symbols. Hence
\begin{verbatim}
    id f(p?) = p.p;
\end{verbatim}
would match \verb:f(q):, \verb:f(2*q-r): and \verb:f(a*q+f(x)*r):, if p, q 
and r are vectors, and a and x are symbols, and f is a function. It would 
not match \verb:f(x): and neither would it match \verb:f(q*r):, nor 
\verb:f(a*q+x):.

Wildcard\index{wildcard!symbol} symbols are the most flexible objects. They 
can match symbols, numbers and expressions that do not contain loose 
indices or vectors without indices. These last objects are called 
scalar\index{scalar objects} objects. Hence wildcard symbols can match all 
scalar objects. In
\begin{verbatim}
    id x^n? = x^(n+1)/(n+1);
\end{verbatim}
the wildcard symbol n would normally match a numerical integer power. In
\begin{verbatim}
    id f(x?) = x^2;
\end{verbatim}
there would be a match with \verb:f(y):, with \verb:f(1+2*y): and with 
\verb:f(p.p):, but there would not be a match with \verb:f(p): if p is a 
vector.

There is one extra type of wildcards. This type is rather special. It 
refers to groups of function
arguments\index{wildcard!argument field}\index{argument field wildcard}. 
The number of arguments is not specified. These variables are indicated by 
a question mark followed by a name (just the opposite of the other wildcard 
variables), and in the right hand side they are also written with the 
leading question mark:
\begin{verbatim}
    id f(?name) = g(1,?name);
\end{verbatim}
In this statement\index{?name} all occurrences of f with any number of 
arguments (including no arguments) will match. Hence \verb:f(mu,nu): will 
be replaced by \verb:g(1,mu,nu):. In the case that f is a regular function 
and g is a tensor, it is conceivable that the arguments in \verb:?name: 
will not fit inside a tensor. For instance \verb:f(x):, with x a symbol, 
would match and FORM would try to put the symbol inside the tensor g. This 
would result in a runtime error. In general FORM will only accept arguments 
that are indices or single vectors for a substitution into a tensor. The 
object \verb:?name: is called an {\bf argument field wildcard}.

One should realize that the use of multiple argument field wildcards can 
make the pattern matching slow.
\begin{verbatim}
    id f(?a,p1?,?b,p2?,?c,p3?,?d)*g(?e,p3?,?f,p1?,?g,p2?,?h) = ....
\end{verbatim}
may involve considerable numbers of tries, especially when there are many 
occurrences of f and g in a term. One should be very careful with this.
 
A complication is the pattern matching in functions with symmetry 
properties. In principle FORM has to try all possible permutations before 
it can conclude that a match does not occur. This can become rather time 
consuming when many wildcards are involved. FORM has a number of tricks 
built in, in an attempt to speed this up, but it is clear that for many 
cases these tricks are not enough. This type of pattern matching is one of 
the weakest aspects of `artificial intelligence' in general. It is hoped 
that in future versions it can be improved. For the moment the practical 
consequence is that argument field wildcards cannot be used in symmetric 
and antisymmetric functions. If one needs to make a generic replacement in 
a symmetric function one cannot use
\begin{verbatim}
    CFunction f(symmetric),g(symmetric);
    id  f(?a) = ....;
\end{verbatim}
but one could try something like
\begin{verbatim}
    CFunction f(symmetric),ff,g(symmetric);
    id  f(x1?,...,x5?) = ff(x1,...,x5);
    id  ff(?a) = ...;
    id  ff(?a) = f(?a);
\end{verbatim}
if f has for instance 5 arguments. If different numbers of arguments are 
involved, one may need more than one statement here or a statement with the 
replace\_\index{replace\_} function:
\begin{verbatim}
    Multiply replace_(f,ff);
\end{verbatim}
It just shows that one should at times be a bit careful with overuse of 
(anti)symmetric functions. Cyclic functions do not have this restriction.

When there are various possibilities for a match, FORM will just take the 
first one it encounters. Because it is not fixed how FORM searches for 
matches (in future versions the order of trying may be changed without 
notice) one should try to avoid ambiguities\index{ambiguity} as in
\begin{verbatim}
    id f(?a,?b) = g(?a)*h(?b);
\end{verbatim}
Of course the current search method is fully consistent (and starts with 
all arguments in \verb:?a: and none in \verb:?b: etc, but a future pattern 
matcher may do it in a different order.

When two argument field wildcards in the left hand side have the same name, 
a match will only occur, when they match the same objects. Hence
\begin{verbatim}
    id f(?a,?a) = g(?a);
\end{verbatim}
will match \verb:f(a,b,a,b): or just \verb:f: (in which case \verb:?a: will 
have zero arguments), but it will not match \verb:f(b,b,b):.

Sometimes it is useful when a search can be restricted to a limited set of 
objects. For this FORM knows the concept of sets\index{set}. If the name of 
a set is attached after the question mark, this is an indication for FORM 
to look only for matches in which the wildcard becomes one of the members 
of the set:
\begin{verbatim}
    Symbols a,a1,a2,a3,b,c;
    Set aa:a1,a2,a3;

    id f(a?aa) = ...
\end{verbatim}
would match \verb:f(a1): but not \verb:f(b):. Sets can also be defined 
dynamically\index{set!dynamical} by enclosing the elements between curly 
brackets\index{bracket!curly} as in:
\begin{verbatim}
    Symbols a,a1,a2,a3,b,c;

    id f(a?{a1,a2,a3}) = ...
\end{verbatim}
Sets\index{Set of symbols} of symbols can contain (small integer) numbers 
as well. Similarly sets\index{set of indices} of indices can contain fixed 
indices (positive numbers less than the value of fixindex\index{fixindex} 
(see the chapter on the setup \ref{setup}). This means that some sets can 
be ambiguous\index{set!ambiguous} in their nature.

Sometimes sets\index{sets!array} can be used as some type of 
array\index{array}. In the case of
\begin{verbatim}
    Symbols a,a1,a2,a3,b,c,n;
    Set aa:a1,a2,a3;

    id f(a?aa[n]) = ...
\end{verbatim}
not only does `a' have to be an element of the set aa, but if it is an 
element of that set, n will become the number of the element that has been 
matched. Hence for \verb:f(a2): the wildcard a would become \verb:a2: and 
the wildcard n would become 2. These objects can be used in the 
righthandside. One can also use sets in the righthandside with an index 
like the n of the previous example:
\begin{verbatim}
    Symbols a,a1,a2,a3,b1,b2,b3,c,n;
    Functions f,g1,g2,g3;
    Set aa:a1,a2,a3;
    Set bb:b1,b2,b3;
    Set gg:g1,g2,g3;

    id f(a?aa[n]) = gg[n](bb[n]);
\end{verbatim}
which would replace \verb:f(a2): by \verb:g2(b2):. One cannot do 
arithmetic\index{arithmetic} with these number of the array element. 
Constructions like \verb:bb[n+1]: are not allowed.

There is one more mechanism by which the array nature of sets can be used. 
In the statement (declarations as before)
\begin{verbatim}
    id f(a?aa?bb) = a*f(a);
\end{verbatim}
a will have to be an element of the set aa, but after the matching it takes 
the identity of the corresponding\index{set!corresponding element} element of 
the set bb. Hence \verb:f(a2): becomes after this statement 
\verb:b2*f(b2):.

Wildcards can also give their value directly to 
\$-variables\index{wildcard!\$-variable}\index{\$-variable} (see the chapter of the \$-variables 
\ref{dollars}). If a \$-variable is attached to a wildcard (if there is a 
set restriction, it should be after the set) the \$-variable will obtain 
the same contents as the wildcard, provided a match occurs. If there is 
more than one match, the last match will be in the \$-variable.
\begin{verbatim}
    id f(a?$w) = f(a);
\end{verbatim}
will put the match of a in \verb:$w:. Hence in the case of \verb:f(a2): the 
\$-variable will have the value \verb:a2:. In the case of 
\verb:f(a2)*f(a3): the eventual value of \verb:$w: depends on the order in 
which FORM does the matching. This is not specified and it would not be 
good strategy to make programs that will depend on it. A future pattern 
matcher might do it differently! But one could do things like
\begin{verbatim}
    while ( match(f(a?$w)) );
        id f($w) = ....
        id g($w) = ....
    endwhile;
\end{verbatim}
just to make sure with which match one is working.

