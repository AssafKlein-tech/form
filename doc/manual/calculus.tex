


The routines inside FORM that deal with the coefficients are all written in 
C (and hence not in assembler language!). This enhances the portability 
greatly. It are these routines that determine the word size in FORM. The 
requirement that the multiplication of two words must be a rather natural 
operation makes that on a 32-bits architecture the word size becomes 16 
bits and on a 64-bits architecture it becomes 32 bits. In some 32-bits 
processors one could use a 32 bits multiplication and recover the full 64 
bits result by looking at two registers. Similarly divisions can be done 
that way. But this requires assembly language programming, because in C the 
only way one can do this is by first casting one of the numbers to (long 
long int) and then the compiler usually creates several multiplications all 
except one being superfluous. The fact that the use of low level GMP 
routines can give a slightly faster code is entirely due to the fact that 
indeed they work with these longer words and use assembly level routines.

Originally the low level calculus routines (addition, multiplication, 
division and the calculation of GCD's) were fully optimized for relatively 
short numbers. The idea being that for most calculations that is what 
occurs most of the time. Over the years computers became bigger and people 
were taking expansions further and further and hence the speed of these 
routines became a noticeable factor. This was mostly the GCD routine. In 
the past the GCD algorithm had been studied and compares had been made 
between the Euclidean and the binary algorithms. The performance of the 
binary algorithm depends rather crucially on how one can shift through an 
array of integers and in the C language this isn't very efficient. Hence 
this algorithm was abandoned in the late 80's. Of course, due to the fact 
that it uses only shifts and subtractions asymptotically it is faster than 
the Euclidean algorithm, but in the past the region in which it was more 
efficient wasn't reached.

When the calculation of GCD's became a real problem a new algorithm for 
longer numbers was invented which turned out to be a little bit like an 
improved version of the Lehmer-Euclid algorithm. This made the behaviour 
for big integers much better. At yet a later stage the GMP library was 
introduced and applied for numbers that are longer than just a few words. 
Here we need some conversion from FORM words to the words that the GMP 
routines need. For a GCD calculation this is however a negligible factor. 
The improvement in speed was far less dramatic than hoped for, even though 
GMP works with longer words and has its central code in assembler language. 
But faster is faster and hence FORM can use now three of the low level GMP 
routines (GCD, multiplication and division) for its big numbers. The 
improvement coming from the multiplication and the division routines has 
thus far been only a few percent. This was for calculations with numbers 
that occupy tenths of words. If FORM could use the double words from the C 
code, probably the conversion to the GMP notation would more than offset 
the benefit of the use of low level assembly routines.

When very long numbers will be used with thousands of words the situation 
is different. In that case GMP has special algorithms that were never built 
for FORM. But for the moment most FORM programs have not reached such cases 
yet. Whenever this is the case for a program it is best to run this program 
on a computer that provides the GMP library with its operating system. When 
the GMP library is not available on a system, FORM will use its own 
routines, which, as mentioned, are not bad, but were not meant for such 
extreme cases.

The higher level routines for calculus of rational numbers can be done in 
any language. The algorithms are standard and can be found in any decent 
text book. It are the low level routines that determine the eventual 
speed. For these objects the GMP library would probably slow FORM down to a 
significant degree as on the object level they are much messed up with 
memory allocation problems, and at the memory management level FORM doesn't 
have any of these problems.
