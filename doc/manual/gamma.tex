
\chapter{Dirac algebra}
\label{gammaalgebra}

For its use in high\index{high energy physics} energy physics \FORM\ is 
equipped with a built-in class of functions. These are the 
gamma\index{gamma matrices} matrices of the Dirac\index{Dirac algebra} 
algebra which are generically denoted by g\_\index{g\_}. The gamma matrices 
fulfill the relations:
\begin{verbatim}
     {g_(j1,mu),g_(j1,nu)} = 2 * d_(mu,nu)
     [g_(j1,mu),g_(j2,nu)] = 0    j1 not equal to j2.
\end{verbatim}
The first argument is a so-called spin\index{spin line} line index. When 
gamma matrices have the same spin line, they belong to the same Dirac 
algebra and commute with the matrices of other Dirac algebra's. The indices 
mu and nu are over space-time and are therefore usually running from 1 to 4 
(or from 0 to 3 in Bjorken \& Drell metric\index{Bjorken \& Drell metric}). 
The totally antisymmetric product e\_(m1,m2,...,mn)\*g\_(j,m1)\*...\*g\_(j,
mn)/n! is defined to be gamma5 or g5\_(j). The notation 5\index{g5\_} finds 
its roots in 4 dimensional space-time. The unit matrix is denoted by 
gi\_(j). In four dimensions a basis of the Dirac algebra can be given by: 
\begin{verbatim}
     gi_(j)
     g_(j,mu)
     [g_(j,mu),g_(j,nu)]/2
     g5_(j)*g_(j,mu)
     g5_(j)
\end{verbatim}
In a different number of dimensions this basis is correspondingly 
different. We introduce the following notation for convenience: 
\begin{verbatim}
     g6_(j) = gi(j) + g5_(j)     (from Schoonschip)
     g7_(j) = gi(j) - g5_(j)
     g_(j,mu,nu) = g_(j,mu)*g_(j,nu)  (from Reduce)
     g_(j,mu,nu,.....,ro,si) =
                      g_(j,mu,nu,.....,ro)*g_(j,si)
     g_(j,5_) = g5_(j)
     g_(j,6_) = g6_(j)
     g_(j,7_) = g7_(j)
\end{verbatim}
The common operation on gamma matrices is to obtain the trace\index{trace} 
of a string of gamma matrices. This is done with the statement:

\leftvitem{4cm}{trace4\index{trace4}, j}
\rightvitem{12cm}{Take the trace in 4 dimensions of the combination of all 
gamma matrices with spin line j in the current term. Any non-commuting 
objects that may be between some of these matrices are ignored. It is the 
users responsibility to issue this statement only after all functions of 
the relevant matrices are resolved. The four refers to special 
tricks\index{tricks} that 
can be applied in four dimensions. This allows for relatively compact 
expressions. For the complete syntax, consult \ref{substatrace}}

\leftvitem{4cm}{tracen\index{tracen}, j}
\rightvitem{12cm}{Take the trace in an unspecified number of dimensions. 
This number of dimensions is considered to be even. The traces are 
evaluated by only using the anticommutation properties of the matrices. As 
the number of dimensions is not specified the occurrence of a g5\_(j) is a 
fatal error. In general the expressions that are generated this way are 
longer than the four dimensional expressions. For the complete syntax, 
consult \ref{substatracen}}

It is possible to alter the value of the trace of the
unit\index{unit matrix} matrix gi\_(j).\index{gi\_} Its
default value is 4, but by using the 
statement (see \ref{substaunittrace})
\begin{verbatim}
    unittrace value;
\end{verbatim}
it can be altered. Value may be any positive short number ($< 2^{15}$ on 
32\index{32 bits} bit machines and $< 2^{31}$ on 64\index{64 bits} bit 
machines) or a single symbol with the exception of the symbol 
i\_.\index{i\_}

There are several options for the 4-dimensional traces. These options find 
their origin in the Chisholm\index{Chisholm} relation that is valid in 4 
dimensions but not in a general number of dimensions. This relation can be 
found in the literature. It is given by:
\begin{equation}
    \gamma_\mu Tr[\gamma_\mu S] = 2(S + S^R)
\end{equation}
\noindent in which S is a string of gamma matrices with an odd number of 
matrices ($\gamma_5$ counts for an even number of matrices). $S^R$ is the 
reversed string. This relation can be used to combine traces with common 
indices. The use of this relation is the default for trace4\index{trace4}. 
If it needs to be switched off, one should add the extra option
`nocontract':
\begin{verbatim}
    trace4,nocontract,j;
\end{verbatim}
The option `contract'\index{contract} is the default but it can be used to 
enhance the readability of the program. The second option that refers to 
this relation is the option `symmetrize'\index{symmetrize}. Often it 
happens that there are two or more common indices in two spin lines. 
Without the symmetrize option (or with the 
`nosymmetrize'\index{nosymmetrize} option) the first of these indices is 
taken and the relation is applied to it. With the `symmetrize' option 
the average over all possibilities is taken. This means of course that if 
there are two common indices the amount of work is doubled. There is 
however a potentially large advantage. In some traces that involve the use of 
$\gamma_5$ the use of automatic algorithms results often in an avalanche of 
terms with a single Levi-Civita tensor, while symmetry arguments can show 
that these terms should add up to zero. By working out the traces in a more 
symmetric fashion \FORM\ is often capable of eliminating all or nearly all of 
these Levi-Civita tensors. Normally such an elimination is rather 
complicated. It involves relations that have so far defied proper 
implementation, even though people have been looking for such algorithms 
already for a long time. Hence the use of the symmetry from the beginning 
seems at the moment the best bet.

It is possible to only apply the Chisholm\index{Chisholm} identity without 
taking the trace. This is done with the chisholm statement (see 
\ref{substachisholm}). 

The n dimensional traces can use a special feature, when the declaration 
of the indices involved will allow it. When an index has been declared 
as n-dimensional and the dimension is followed by a second symbol as in
\begin{verbatim}
    symbols n,nn;
    index mu=n:nn;
\end{verbatim}
and if the index \verb:mu: is a contracted index in a single 
n-dimensional trace, then the formula for this trace can be shortened by 
using \verb:nn: (one term) instead of the quantity $(n-4)$ (two terms). 
This can make the taking of the n-dimensional traces significantly 
faster.

\vspace{3mm}

\noindent Algorithms\index{algorithms}:

\FORM\ has been equipped with several built in rules to keep the 
number of generated terms to a minimum during the evaluation of a 
trace. These rules are:

\begin{description}
\item [rule 0]
    Strings with an odd number of matrices (gamma5 counts for an even number 
    of matrices) have a trace that is zero, when using trace4 or tracen.
\item [rule 1]
    A string of gamma matrices is first scanned for adjacent 
matrices that have the same contractable index, or that are contracted with 
the same vector. If such a pair is found, the relations
%\begin{eqnarray}
%   \gamma^\mu\gamma^\nu & = & 1\times \delta^{\mu\nu} \nonumber \\
%   \gamma^p\gamma^p & = & 1\times p\mydot p \nonumber
%\end{eqnarray}
\begin{verbatim}
      g_(1,mu,mu) = gi_(1)*d_(mu,mu)
      g_(1,p1,p1) = gi_(1)*p1.p1
\end{verbatim}
\noindent are applied.
\item [rule 2]
   Next there is a scan for a pair of the same contractable 
indices that has an odd number of other matrices in between. This is done 
only for 4 dimensions (trace4) and the dimension of the indices must be 4. 
If found, the Chisholm\index{Chisholm} identity is applied:
%\begin{eqnarray}
%   \gamma^\mu\gamma^{m_1}\gamma^{m_2}\cdots\gamma^{m_n}\gamma^\mu & = &
%        -2\gamma^{m_n}\cdots\gamma^{m_2}\gamma^{m_1}  \nonumber \\
%\end{eqnarray}
\begin{verbatim}
      g_(1,mu,m1,m2,...mn,mu) = -2*g_(1,mn,...,m2,m1)
\end{verbatim}
\item [rule 3]
  Then (again only for trace4) there is a search for a pair 
of matrices with the same 4 dimensional index and an even number of 
matrices in between. If found, one of the following variations of the 
Chisholm\index{Chisholm} identity is applied:
\begin{verbatim}
      g_(1,mu,m1,m2,mu) = 4*gi_(1)*d_(m1,m2)
      g_(1,mu,m1,m2,...,mj,mn,mu) =
                      2*g_(1,mn,m1,m2,...,mj)
                     +2*g_(1,mj,...,m2,m1,mn)
\end{verbatim}
\item [rule 4]
  Then there is a scan for pairs of matrices that have the 
same index or that are contracted with the same vector. If found, the 
identity:
\begin{verbatim}
      g_(1,mu,m1,m2,...,mj,mn,mu) =
                2*d_(mu,mn)*g_(1,mu,m1,m2,...,mj)
               -2*d_(mu,mj)*g_(1,mu,m1,m2,...,mn)
                ....
             -/+2*d_(mu,m2)*g_(1,mu,m1,...,mj,mn)
             +/-2*d_(mu,m1)*g_(1,mu,m2,...,mj,mn)
             -/+2*d_(mu,mu)*g_(1,m1,m2,...,mj,mn)
\end{verbatim}
\noindent is used to 'anticommute'\index{anticommute} these identical 
objects till they become adjacent and can be eliminated with the 
application of rule 1. In the case of an n-dimensional trace and when 
\verb:mu: is an index (it might also be a vector in the above formula) for 
which the definition of the dimension involved two symbols, there is a 
shorter formula. In that case the last three terms can be combined into two 
terms:
\begin{verbatim}
     -/+(n-4)*g_(1,m1,m2,...,mj,mn)
     -/+4*d_(m1,m2)*g_(1,m3,m4,...,mj,mn)
\end{verbatim}
\noindent It should be clear now that this formula is only superior, when 
there is a single symbol to represent $(n-4)$. After this all gamma 
matrices that are left have a different index or are contracted with 
different vectors. These are treated using:
\item [rule5]
  Traces in 4 dimensions for which all gamma matrices have 
a different index, or are contracted with a different four-vector are 
evaluated using the reduction formula
\begin{verbatim}
      g_(1,mu,nu,ro) =
                 g_(1,5_,si)*e_(mu,nu,ro,si)
                +d_(mu,nu)*g_(1,ro)
                -d_(mu,ro)*g_(1,nu)
                +d_(nu,ro)*g_(1,mu)
\end{verbatim}
For tracen the generating algorithm is based on the generation of all 
possible pairs of indices/vectors that occur in the gamma matrices in 
combination with their proper sign. When the dimension is not specified, 
there is no shorter expression.
\end{description}

\noindent Remarks:

When an index is declared to have dimension n and the command trace4 is 
used, the special 4 dimensional rules 2 and 3 are not applied to this 
index. The application of rule 1 or 4 will then give the correct 
results. The result will nevertheless be wrong due to rule 5, when there 
are at least 10 gamma matrices left after the application of the first 4 
rules, as the two algorithms in rule 5 give a difference only, when 
there are at least 10 gamma matrices. For counting gamma matrices the 
$\gamma_5$ counts for 4 matrices with respect to this rule. The result 
is unpredictable, when both indices in four dimensions and indices in n 
dimensions occur in the same string of gamma matrices. Therefore one 
should be very careful, when using the four dimensional trace under the 
condition that the results need to be correct in n dimensions. This is 
sometimes needed, when a $\gamma_5$ is involved. The tracen-statement 
will not allow the presence of a $\gamma_5$. In general it is best to 
emulate n-dimensional traces with a $\gamma_5$ separately. The eventual 
trace, with all matrices with a different index, can be generated with 
the use of the 'distrib\_' function:
% THIS EXAMPLE IS PART OF THE TESTSUITE. CHANGES HERE SHOULD BE APPLIED THERE AS
% WELL!
\begin{verbatim}
    *
    *   Symmetric trace of a gamma5 and 12 regular matrices
    *
    I   m1,...,m12;
    F   G5,g1,g2;
    L   F = G5(m1,...,m12);
    id  G5(?a) = distrib_(-1,4,g1,g2,?a);
    id  g1(?a) = e_(?a);
    id  g2(?a) = g_(1,?a);
    tracen,1;
    .end

Time =       1.07 sec    Generated terms =      51975
                F        Terms in output =      51975
                         Bytes used      =     919164
\end{verbatim}
This rather symmetric result is in contrast to the 4-dimensional result 
which is much shorter, but it is very unsymmetric:
% THIS EXAMPLE IS PART OF THE TESTSUITE. CHANGES HERE SHOULD BE APPLIED THERE AS
% WELL!
\begin{verbatim}
    *
    *   Regular trace of a gamma5 and 12 regular matrices
    *
    I   m1,...,m12;
    L   F = g_(1,5_,m1,...,m12);
    trace4,1;
    .end

Time =       0.02 sec    Generated terms =       1053
                F        Terms in output =       1029
                         Bytes used      =      20284
\end{verbatim}
The precise workings of the distrib\_\index{distrib\_} function is given in 
\ref{fundistrib}.


