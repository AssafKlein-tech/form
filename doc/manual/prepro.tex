
\chapter{The preprocessor}
\label{preprocessor}

%--#[ General :

The preprocessor\index{preprocessor} is a program segment that reads and 
edits\index{edit} the input, after which the processed input is offered to 
the compiler\index{compiler} part of \FORM. When a module\index{module} 
instruction is encountered by the preprocessor, the compilation is halted 
and the module is executed. The compiler buffers are cleared and \FORM\  
will continue with the next module. The preprocessor acts almost purely on 
character strings. As such it does not know about the algebraic properties 
of the objects it processes. Additionally the preprocessor also filters out 
the commentary\index{commentary}.

The commands for the preprocessor are called instructions. Preprocessor 
instructions start with the character \# as the first non-blank character 
in a line. After this there are several possibilities.
\begin{description}
\item[\#:]\index{\#:} Special syntax for setup parameters at the beginning 
of the program. See the chapter on the setup parameters.
\item[\#$-$, \#$+$]\index{\#$-$}\index{\#$+$} Turns the listing of the input off or 
on.
\item[\#name]\index{\#name} Preprocessor command. The syntax of the various 
commands will be discussed below.
\item[\#\$name]\index{\#\$name} Giving a value to a dollar variable in the 
preprocessor. See chapter \ref{dollars} on dollar variables.
\end{description}

%--#] General : 
%--#[ The preprocessor variables :

\section{The preprocessor variables}
\label{preprovariables}

In order to help in the edit\index{edit} function the preprocessor is 
equipped with variables\index{preprocessor variables} that can be defined 
or redefined by the user or by other preprocessor actions. Preprocessor 
variables have regular names that are composed of strings of alphanumeric 
characters of which the first one must be alphabetic. When they are defined 
one just uses this name. When they are used the name should be enclosed 
between a backquote\index{backquote} and a quote\index{quote} as if these 
were some type of brackets. Hence `a2r' is the reference to a regular 
preprocessor variable. Preprocessor variables contain strings of 
characters. No interpretation is given to these strings. The 
backquote/quote pairs can be nested. Hence `a`i'r' will result in the 
preprocessor variable `i' to be substituted first. If this happens to be 
the string "2", the result after the first substitution would be `a2r' and 
then \FORM\ would look for its string value.

The use of the backquotes is different from the earlier versions of \FORM. 
There the preprocessor variables would be enclosed in a pair of quotes and 
no nesting\index{nesting} was possible. \FORM\ still understands this old 
notation because it does not lead to ambiguities. The user is however 
strongly advised to use the new notation with the backquotes, because in 
future versions the old\index{old notation} notation may not be recognized 
any longer.

\noindent \FORM\ has a number of built in preprocessor variables. They are:

\begin{description}
\item[VERSION\_]  The current version\index{VERSION\_} as the 4 in 4.1.
\item[SUBVERSION\_]  The sub-version\index{SUBVERSION\_} as the 1 in 4.1.
\item[NAME\_]        The name\index{NAME\_} of the program file.
\item[DATE\_]        The date\index{DATE\_} of the current run.
\item[CMODULE\_]     The number\index{CMODULE\_} of the current module.
\item[SHOWINPUT\_]   If input listing\index{SHOWINPUT\_} is on: 1, if off: 0.
\item[EXTRASYMBOLS\_]  The current number of extra symbols\index{EXTRASYMBOLS\_}
       (see \ref{substaextrasymbols}).
\item[OLDNUMEXTRASYMBOLS\_]  The number of extra symbols\index{OLDNUMEXTRASYMBOLS\_}
            before the current optimization started (see chapter \ref{optimization}).
\item[OPTIMMINVAR\_]  The number of the first extra symbol\index{OPTIMMINVAR\_} needed
               for the current optimization (see chapter \ref{optimization}).
\item[OPTIMMAXVAR\_]  The number of the last extra symbol\index{OPTIMMAXVAR\_} needed
               for the current optimization (see chapter \ref{optimization}).
\end{description}

\noindent If \FORM\ cannot find a preprocessor variable, because it has 
neither been defined by the user, nor is it one of the built in variables, 
it will look in the systems environment\index{environment} to see whether 
there is an environment variable by that name. If this is the case its 
string value will be substituted.

%--#] The preprocessor variables : 
%--#[ Calculator :
\section{The preprocessor calculator}
\label{calculator}

Sometimes a preprocessor\index{preprocessor variable!numeric} variable 
should be interpreted as a number and some arithmetic\index{arithmetic} 
should be done with it. For this \FORM\ is equipped with what is called the 
preprocessor calculator\index{calculator}. When the input reading device 
encounters a left curly\index{curly bracket} bracket\index{bracket!curly} 
\verb:{:, it will read till the matching right curly bracket \verb:}: and 
then test whether the characters (after substitution of preprocessor 
variables) can be interpreted as a numerical expression. If it is not a 
valid numerical expression the whole string, including the curly brackets, 
will be passed on to the later stages of the program. If it is a numerical 
expression, it will be evaluated, and the whole string, including the curly 
brackets, will be replaced by a textual representation of the result. 
Example:
\begin{verbatim}
    Local F`i' = F{`i'-1}+F{`i'-2};
\end{verbatim}
If the preprocessor variable i has the value 11, the calculator makes this 
into
\begin{verbatim}
    Local F11 = F10+F9;
\end{verbatim}
Valid numerical expressions can contain the characters
\begin{verbatim}
   0 1 2 3 4 5 6 7 8 9 + - * / % ( ) { } & | ^ !
\end{verbatim}
The use of parentheses is as in regular arithmetic. The curly 
brackets fulfil the same role, as one can nest these brackets of course. 
Operators are:
\begin{description}
\item[$+$] Regular addition\index{addition}.
\item[$-$] Regular subtraction\index{subtraction}.
\item[$\ast$] Regular multiplication\index{multiplication}.
\item[$/$] Regular (integer) division\index{division}.
\item[$\%$] The remainder\index{remainder} after (integer) division as in 
the language C\index{C}.
\item[$\&$] And\index{and} operator. This is a bitwise operator.
\item[$|$] Or\index{or} operator. This is a bitwise or.
\item[$\wedge$] Exponent\index{exponent} operator.
\item[$!$] Factorial\index{factorial}. This is a postfix operator.
\item[$\wedge\%$] A postfix ${}^2\!\log$. This means that it 
takes\index{twolog} the ${}^2\!\log$ of the object to the left of it.
\item[$\wedge/$] A postfix square\index{square root} root. This means that 
it takes the square root of the object to the left of it.
\end{description}
Note that all arithmetic\index{arithmetic} is done over the integers and 
that there is a finite range. On 32\index{32 bits} bit systems this range 
will be $2^{31}-1$ to $-2^{31}$, while on 64\index{64 bits} bit systems 
this will be  $2^{63}-1$ to $-2^{63}$. In particular this means that 
\verb:{13^/}: becomes \verb:3:. The preprocessor calculator is only meant 
for some simple counting and organization of the program flow. Hence there 
is no large degree of sophistication. Very important is that the 
comma\index{comma} character is not a legal character for the preprocessor 
calculator. This can be used to avoid some problems. Suppose one needs to 
make a substitution of the type:
\begin{verbatim}
    id f(x?!{0}) = 1/x;
\end{verbatim}
in which the value zero should be excluded from the pattern matching (see 
dynamical\index{set!dynamical} sets in chapter \ref{pattern} on pattern 
matching). This would not work, because the preprocessor would make this 
into
\begin{verbatim}
    id f(x?!0) = 1/x;
\end{verbatim}
which is illegal syntax. Hence the proper trick is to write
\begin{verbatim}
    id f(x?!{,0}) = 1/x;
\end{verbatim}
With the comma the preprocessor will leave this untouched, and hence now 
the set is passed properly.

Good use of the preprocessor calculator can make life much easier for 
\FORM. For example the following statements
\begin{verbatim}
    id  f(`i') = 1/(`i'+1);
    id  f(`i') = 1/{`i'+1};
\end{verbatim}
are quite different in nature. In the first statement the compiler gets an 
expression with a composite denominator. The compiler never tries to 
simplify expressions by doing algebra on them. Sometimes this may not be 
optimal, but there are cases in which it would cause wrong results (in 
particular when noncommuting and commuting functions are mixed and 
wildcards are used). Hence the composite denominator has to be worked out 
during run time for each term separately. The second statement has the 
preprocessor work out the sum and hence the compiler gets a simple fraction 
and less time will be needed during running. Note that
\begin{verbatim}
    id  f(`i') = {1/(`i'+1)};
\end{verbatim}
would most likely not produce the desired result, because the preprocessor 
calculator works only over the integers. Hence, unless i is equal to zero 
or -2, the result would be zero (excluding of course the fatal error when i 
is equal to -1).

%--#] Calculator : 
%--#[ ... :
\section{The triple dot operator}
\label{tripledot}

The last\index{...} stage of the actions of the preprocessor involves the 
triple dot operator. It indicates a repeated pattern as in \verb:a1+...+a4: 
which would expand into \verb:a1+a2+a3+a4:. This operator is used in two 
different ways. First the most general way:
\begin{verbatim}
    <pattern1>operator1...operator2<pattern2>
\end{verbatim}
in which the less\index{less than} than and greater\index{greater than} 
than signs serve as boundaries for the patterns. The operators can be any 
pair of the following:
\begin{description}
\item[+\ +]\index{+...+} Repetitions will be separated by plus signs.
\item[--\ --]\index{-...-} Repetitions will be separated by minus signs.
\item[+\ --]\index{+...-} Repetitions will be separated by alternating signs. 
First will be plus.
\item[--\ +]\index{-...+} Repetitions will be separated by alternating signs. 
First will be minus.
\item[$\ast\ \ast$]\index{*...*} Repetitions will be separated by $\ast$.
\item[/\ /]\index{/.../} Repetitions will be separated by /.
\item[,\ ,]\index{,...,} Repetitions will be separated by comma's.
\item[:\ :]\index{:...:} Repetitions will be separated by {\it single} dots.
%\item[+\ +]\index{.@$+\cdots+$} Repetitions will be separated by plus signs.
%\item[--\ --]\index{.@$-\cdots-$} Repetitions will be separated by minus signs.
%\item[+\ --]\index{.@$+\cdots-$} Repetitions will be separated by alternating signs. 
%First will be plus.
%\item[--\ +]\index{.@$-\cdots+$} Repetitions will be separated by alternating signs. 
%First will be minus.
%\item[$\ast\ \ast$]\index{.@$\ast\cdots\ast$} Repetitions will be separated by $\ast$.
%\item[/\ /]\index{.@$/\cdots/$} Repetitions will be separated by /.
%\item[,\ ,]\index{.@$,\cdots,$} Repetitions will be separated by comma's.
%\item[:\ :]\index{.@$:\cdots:$} Repetitions will be separated by {\it single} dots.
\end{description}
For such a pair of operators \FORM\ will inspect the patterns\index{pattern} 
and see whether the differences between the two patterns are just numbers. 
If the differences are numbers and the absolute value of the difference of 
each matching pair is always the same (a difference of zero is allowed too; 
it leads to no action for the pair), then \FORM\ will expand the pattern, 
running from the first to the last in increments of one. For each pair the 
counter can either run up or run down, depending on whether the number in 
the first pattern is greater or less than the number in the second pattern. 
Example:
\begin{verbatim}
    Local F = <a1b6(c3)>-...+<a4b3(c6)>;
\end{verbatim}
leads to
\begin{verbatim}
    Local F = a1b6(c3)-a2b5(c4)+a3b4(c5)-a4b3(c6);
\end{verbatim}
The second form is a bit simpler. It recognizes that there are special 
cases that can be written in a more intuitive way. If there is only a 
single number to be varied, and it is the end of the pattern, and the rest 
of the patterns consists only of alphanumeric characters of which the first 
is an alphabetic character, we do not need the less than/greater than 
combination. This is shown in
\begin{verbatim}
    Symbol a1,...,a12;
\end{verbatim}
There is one extra exception. The variables used this way may have a 
question mark after them to indicate that they are wildcards:
\begin{verbatim}
    id  f(a1?,...,a4?) = g(a1,...,a4,a1+...+a4);
\end{verbatim}
This construction did not exist in earlier versions of \FORM\ (version 1 and 
version 2). There one needed the \#do\index{\#do} instruction for many of 
the above constructions, creating code that was very hard to read. The 
\verb:...: operator should improve the readability of the programs very 
much.

%--#] ... : 
%--#[ addseparator :
 
\section{\#addseparator}
\label{preaddseparator}

\noindent Syntax:

\#addseparator character
 
\noindent See also \#rmseparator (\ref{prermseparator}),
            \#call (\ref{precall}), \#do (\ref{predo})

\noindent Adds a character\index{\#addseparator} to the list of permissible 
separator characters for arguments of \#call or \#do instructions. By 
default the two characters that are permitted are the comma and the 
character \verb:|:. Blanks, tabs and double quotes are ignored. Note that 
the comma must be specified between double quotes as in
\begin{verbatim}
  #addseparator ","
\end{verbatim}


%--#] addseparator : 
%--#[ append :
 
\section{\#append}
\label{preappend}

\noindent Syntax:

\#append $<$filename$>$
 
\noindent See also write (\ref{prewrite}),
            close (\ref{preclose}), create (\ref{precreate}),
            remove (\ref{preremove})

\noindent Opens\index{\#append} the named file for writing. The file will 
be positioned at the end. The next \#write\index{\#write} instruction will 
add to it.

%--#] append : 
%--#[ break :

\section{\#break}
\label{prebreak}

\noindent Syntax:

\#break
 
\noindent See also switch (\ref{preswitch}),
        endswitch (\ref{preendswitch}),
        case (\ref{precase}),
        default (\ref{predefault})

\noindent If the\index{\#break} lines before were not part of the control 
flow ({\it i.e.} these lines are used for the later stages of the program), 
this instruction is ignored. If they are part of the control flow, the flow 
will continue after the matching \#endswitch\index{\#endswitch} 
instruction. The \#break instruction must of course be inside the range of 
a \#switch\index{\#switch}/\#endswitch construction.

%--#] break : 
%--#[ breakdo :
 
\section{\#breakdo}
\label{prebreakdo}

\noindent Syntax:

\#breakdo [{\tt<}number{\tt>}]

\noindent See also \#do (\ref{predo}) and \#enddo (\ref{preenddo})

\noindent The \#breakdo\index{\#breakdo} instruction allows one to jump out 
of a \#do loop. If a (nonzero integer) number is specified it indicates the 
number of loops the program should terminate. Control will continue after 
the \#enddo instruction of the number of loops indicated by `number'. 
The default value is one. If the value is zero the statement has no effect.

%--#] breakdo : 
%--#[ call :

\section{\#call}
\label{precall}

\noindent Syntax:

\#call procname(var1,...,varn)
 
\noindent See also procedure (\ref{preprocedure}), endprocedure
(\ref{preendprocedure})

\noindent This instruction\index{\#call} calls the 
procedure\index{procedure} with the name procname. The result is that \FORM\ 
looks for this procedure, first in its procedure 
buffers\index{buffer!procedure} (for procedures that were defined in the 
regular text stream as explained under the \#procedure\index{\#procedure} 
instruction), then it looks for a file by the name procname.prc in the 
current directory, and if it still has not found the procedure, it looks in 
the directories indicated by the path\index{path} variable in either the setup 
file or at the start of the program (see chapter \ref{setup} on the setup 
file). Next it looks for the -p option in the command that started \FORM\ 
(see the chapter on running \FORM). If this -p option has not been used \FORM\ 
will see whether there is an environment variable by the name 
FORMPATH\index{FORMPATH}. The directories indicated there will be searched 
for the file procname.prc. If \FORM\ cannot find the file, there will be an 
error message and execution will be stopped immediately.

Once the procedure has been located, \FORM\ reads the whole file and then 
determines whether the number of parameters is identical in the 
\#call\index{\#call} instruction and the \#procedure\index{\#procedure} 
instruction. A difference is a fatal error.

The parameter field consists of strings, separated by commas. If a string 
contains a comma, this comma should be preceded by a 
backslash\index{backslash} character (\verb:\:). If a string should contain 
a linefeed\index{linefeed}, one should `escape' this linefeed by putting a 
backslash and continue on the next line.

Before version 3 of \FORM\ the syntax was different. The parentheses 
were curly brackets and the separators the symbol \verb:|:. This was made 
to facilitate the use of strings that might contain commas. In practise 
however, this turned out to be far from handy. In addition the new 
preprocessor calculator is a bit more active and hence an instruction of 
the type
\begin{verbatim}
    #call test{1}
\end{verbatim}
will now be intercepted by the preprocessor calculator\index{calculator} 
and changed into
\begin{verbatim}
    #call test1
\end{verbatim}
Because there are many advantages to the preprocessor calculator treating 
the parameters of the procedures before they are called (in the older 
versions it did not do this), the notation has been changed. \FORM\ still 
understands the old notation, provided that there is no conflict with the 
preprocessor calculator. Hence
\begin{verbatim}
    #call test{1|a}
    #call test{1,a}
    #call test(1|a)
    #call test(1,a)
\end{verbatim}
are all legal and give the same result, but only the last notation will 
work in future versions of \FORM.

Nowadays also the use of the argument field wildcard (see chapter 
\ref{pattern} on pattern matching) is allowed as in the 
regular functions:
% THIS EXAMPLE IS PART OF THE TESTSUITE. CHANGES HERE SHOULD BE APPLIED THERE AS
% WELL!
\begin{verbatim}
    #define a "1"
    #define bc2 "x"
    #define bc3 "y"
    #define b "c`~a'"
    #procedure hop(c,?d);
    #redefine a "3"
    #message This is the call: `c',`?d'
    #endprocedure
    
    #redefine a "2"
    #message This is b: `b'
~~~This is b: c2
    
    #call hop(`b`!b''`!b'`b'`!b'`b',`~a',`b',`a')
~~~This is the call: xc2c3c2c3,3,c3,2
    
    .end
\end{verbatim}
We also see here that the rules about delayed substitution (see also the 
\#define\index{\#define} instruction in section \ref{predefine}) apply. The 
use of `!b' cancels the delayed substitution that is asked for in the 
definition of b.

The default extension for procedure files is .prc\index{.prc}, but it is 
possible to change this. There are two different ways: One is with the 
\#procedureExtension\index{\#procedureExtension} instruction in section 
\ref{preprocedureextension}. The other is via the setup (see the chapter on 
the setup file, chapter \ref{setup}).

%--#] call : 
%--#[ case :

\section{\#case}
\label{precase}

\noindent Syntax:

\#case string
 
\noindent See also switch (\ref{preswitch}),
        endswitch (\ref{preendswitch}),
        break (\ref{prebreak}),
        default (\ref{predefault})

\noindent The lines after the \#case\index{\#case} instruction will be used 
if either this is the first \#case\index{\#case} instruction of which the 
string matches the string in the \#switch\index{\#switch} instruction, or 
the control flow was already using the lines before this \#case instruction 
and there was no \#break\index{\#break} instruction (this is called 
fall-through). The control flow will include lines either until the next 
matching \#break instruction, or until the matching 
\#endswitch\index{\#endswitch} instruction.

%--#] case : 
%--#[ clearoptimize :

\section{\#clearoptimize}
\label{preclearoptimize}

\noindent Syntax:

\#clearoptimize

See the chapter about optimization \ref{optimization}
 
%--#] clearoptimize : 
%--#[ close :

\section{\#close}
\label{preclose}

\noindent Syntax:

\#close $<$filename$>$
 
\noindent See also write (\ref{prewrite}), append (\ref{preappend}),
            create (\ref{precreate}), remove (\ref{preremove})

\noindent This instruction closes\index{\#close} the file\index{file!close} 
by the given name, if such a file had been opened by the previous 
\#write\index{\#write} instruction. Normally \FORM\ closes all such files at 
the end of execution. Hence the user would not have to worry about this. 
The use of a subsequent \#write instruction with the same file name will 
remove the old contents and hence start basically a new file. There are 
times that this is useful.

%--#] close : 
%--#[ commentchar :

\section{\#commentchar}
\label{precommentchar}

\noindent Syntax:

\#commentchar character

\noindent The specified\index{\#commentchar} character should be a single 
non-whitespace character. There may be white space (blanks and/or tabs) 
before or after it. The character will take over the role of the comment 
character. {\it i.e.} any line that starts with this character in column 1 
will be considered commentary\index{commentary}. This feature was provided 
because output of some other algebra programs could put the multiplication 
sign in column 1 in longer expressions.

The default commentary character is $\ast$.

%--#] commentchar : 
%--#[ create :
 
\section{\#create}
\label{precreate}

\noindent Syntax:

\#append $<$filename$>$
 
\noindent See also write (\ref{prewrite}),
            close (\ref{preclose}), append (\ref{preappend}),
            remove (\ref{preremove})

\noindent Opens the named\index{\#create} file for writing. If the file 
existed already, its previous contents will be lost. The next 
\#write\index{\#write} instruction will add to it. In principle this 
instruction is not needed, because the \#write instruction would create the 
file if it had not been opened yet at the moment of writing.

%--#] create : 
%--#[ default :

\section{\#default}
\label{predefault}

\noindent Syntax:

\#default
 
\noindent See also switch (\ref{preswitch}),
        endswitch (\ref{preendswitch}),
        case (\ref{precase}),
        break (\ref{prebreak})

\noindent Control\index{\#default} flow continues after this instruction if 
there is no \#case\index{\#case} instruction of which the string matches 
the string in the \#switch\index{\#switch} instruction. Control flow also 
continues after this instruction, if the lines before were included and 
there was no \#break\index{\#break} instruction to stop the control flow 
(fall-through). Control flow will stop either when a matching \#break 
instruction is reached, or when a matching \#endswitch\index{\#endswitch} 
is encountered. In the last case of course control flow will continue after 
the \#endswitch instruction.

%--#] default : 
%--#[ define :

\section{\#define}
\label{predefine}

\noindent Syntax:

\#define name "string"
 
\noindent See also redefine (\ref{preredefine}), undefine 
(\ref{preundefine})

\noindent in which name\index{\#define} refers to the name of the 
preprocessor\index{preprocessor variable} 
variable\index{variable!preprocessor} to be defined and the contents of the 
string will form the value of the variable. The double quotes are mandatory 
delimiters of the string.

The use of the \#define\index{\#define} instruction creates a new instance 
of the preprocessor variable with the given name. This means that the old 
instance\index{instance} remains. If for some reason the later instance 
becomes undefined (see for instance \#undefine), the older instance will be 
the one that is active. If the old definition is to be overwritten, one 
should use the \#redefine\index{\#redefine} instruction.

As of version 3.2 preprocessor variables can also have arguments as in the 
C\index{C} language. Hence

\#define var(a,b) "(`\verb:~:a'+`\verb:~:b'+`c')"

is allowed. The parameters should be referred to inside a pair of `' as 
with all preprocessor variables. A special feature is the socalled 
delayed\index{delayed substitution} 
substitution\index{substitution!delayed}. With macro's like the above the 
question is always {\sl when} a preprocessor variable will be substituted. 
Take for instance
% THIS EXAMPLE IS PART OF THE TESTSUITE. CHANGES HERE SHOULD BE APPLIED THERE AS
% WELL!
\begin{verbatim}
    #define c "3"
    #define var1(a,b) "(`~a'+`~b'+`c')"
    #define var2(a,b) "(`~a'+`~b'+`~c')"
    #redefine c "4"
    Local F1 = `var1(1,2)';
    Local F2 = `var2(1,2)';
    Print;
    .end

   F1 =
      6;

   F2 =
      7;
\end{verbatim}
The parameter c will be substituted immediately when var1 is defined. In 
var2 it will be only substituted when var2 is used. It should be clear that 
a and b should also be used in the delayed fashion because they do not 
exist yet at the moment of the definition of var1 and var2. Notice also 
that the whole macro\index{macro}, with its arguments should be placed 
between the backquote and the quote. Another example can be found with the 
\#call\index{\#call} instruction. See section \ref{precall}

%--#] define : 
%--#[ do :
 
\section{\#do}
\label{predo}

\noindent Syntax:

\#do lvar = i1,i2

\#do lvar = i1,i2,i3

\#do lvar = $\{$string1$|$...$|$stringn$\}$

\#do lvar = $\{$string1,...,stringn$\}$

\#do lvar = nameofexpression
 
\noindent See also enddo (\ref{preenddo})

\noindent The \#do\index{\#do} instruction\index{do loop} needs a matching 
\#enddo\index{\#enddo} instruction. All code in-between these two 
instructions will be read as many times as indicated in the parameter field 
of the \#do instruction. The parameter lvar is a preprocessor variable of 
which the value is determined by the other parameters. Inside the loop it 
should be referred to by enclosing its name between a backquote/quote pair 
as is usual for preprocessor variables. The various possible parameter 
fields have the following meaning:
\begin{description}
\item[\#do lvar = i1,i2] The parameters i1 and i2 should be integers or 
names of dollar expressions that evaluate into integers. The 
first time in the loop lvar will get the value of i1 (as a string) and each 
next time its value will be one greater (translated into a string again). 
The last time in the loop the value of lvar will be the greatest integer 
that is less or equal to i2. If i2 is less than i1, the loop is skipped 
completely. If i2 is the name of a dollar variable, each time the control 
reaches the end of the loop the dollar variable is evaluated and the 
current value is used.
\item[\#do lvar = i1,i2,i3] The parameters i1,i2 and i3 should be integers 
or names of dollar expressions that evaluate into integers. 
The first time in the loop lvar will get the value of i1 (as a string) and 
each next time its value will be incremented by adding i3 (translated into 
a string again). If i3 is positive, the last value of lvar will be the one 
for which lvar+i3 is greater than i2. If i2 is less than i1, the loop is 
skipped completely. If i3 is negative the last value of lvar will be the 
one for which lvar+i3 is less than i2. If i3 is zero there will be an 
error. If i2 or i3 are the names of a dollar variable, each time the control 
reaches the end of the loop the dollar variable(s) is/are evaluated and the 
current value is used.
\item[\#do lvar = $\{$string1$|$...$|$stringn$\}$] The first time in the 
loop the value of lvar is the string indicated by string1, the next time 
will be string2 etc till the last time when it will be stringn. This is 
called a listed\index{listed loop} loop\index{loop!listed}. The notation 
with the $|$ is an old notation which is still accepted. The new notation 
uses a comma instead.
\item[\#do lvar = $\{$string1,...,stringn$\}$] The first time in the loop 
the value of lvar is the string indicated by string1, the next time will be 
string2 etc till the last time when it will be stringn. This is called a 
listed\index{listed loop} loop\index{loop!listed}.
\item[\#do lvar = expression] The loop variable will take one by one for 
its value all the terms of the given expression. This is protected against 
changing the expression inside the loop by making a copy of the expression 
inside the memory. Hence one should be careful with very big expressions. 
An expression that is zero gives a loop over zero terms, hence the loop is 
never executed.
\end{description}
The first two types of \#do instructions are called 
numerical\index{numerical loop} loops\index{loop!numerical}. In the 
parameters of numerical loops the preprocessor calculator\index{calculator} 
is invoked automatically. One should make sure not to use a leading $\{$ 
for the first numerical parameter in such a loop. This would be interpreted 
as belonging to a listed loop.

After a loop has been finished, the corresponding preprocessor variable 
will be undefined. This means that if there is a previous preprocessor 
variable by the same name, the value of the \#do instruction will be used 
inside the loop, and afterwards the old value will be active again.

It is allowed to overwrite the value of a preprocessor \#do instruction 
variable. This can be very useful to create the equivalent of a repeat loop 
that contains .sort instructions as in
\begin{verbatim}
    #do i = 1,1
        id,once,x = y+2;
        if ( count(x,1) > 0 ) redefine i "0";
        .sort
    #enddo
\end{verbatim}
A few remarks are necessary here. The redefine\index{redefine} statement 
(see section \ref{substaredefine}) should be before the last 
.sort\index{.sort} inside the loop, because the \#do instruction is part of 
the preprocessor. Hence the value of i is considered before the module is 
executed. This means that if the redefine would be after the .sort, two 
things would go wrong: First the loop would be terminated before the 
redefine would ever make a chance of being executed. Second the statement 
would be compiled in the expectation that there is a variable i, but then 
the loop would be terminated. Afterwards, when the statement is being 
executed it would refer to a variable that does not exist any longer. 

If one wants to make a loop over the externals of the brackets of an 
expression only, one needs to do some work. Assume we have the expression F 
and we want to loop over the brackets in x and y:
\begin{verbatim}
    L   FF = F;
    Bracket x,y;
    .sort
    CF acc,acc2;
    Skip F;
    Collect acc,acc2;
    id  acc(x?) = 1;
    id  acc2(x?)= 1;
    B   x,y;
    .sort
    Skip F;
    Collect acc;
    id  acc(x?) = 1;
    .sort
    #do i = FF
    L   G = F[`i'];
        .
        .
    #enddo
\end{verbatim}
Notice that we have to do the collect\index{collect} trick twice because 
the first time the bracket could be too long for one term. The second time 
that restriction doesn't exist because besides the x and the y there are 
only integer coefficients.

%--#] do : 
%--#[ else :

\section{\#else}
\label{preelse}

\noindent Syntax:

\#else
 
\noindent See also if (\ref{preif}),
            endif (\ref{preendif}),
            elseif (\ref{preelseif}),
            ifdef (\ref{preifdef}),
            ifndef (\ref{preifndef})

\noindent This instruction\index{\#else} is used inside a 
\#if\index{\#if}/\#endif\index{\#endif} construction. The code that follows 
it until the \#endif instruction will be read if the condition of the \#if 
instruction (and of none of the corresponding \#elseif\index{\#elseif} 
instructions) is not true. If any of these conditions is true, this code is 
skipped. The reading is stopped after the matching \#endif is encountered 
and continued after this matching \#endif instruction.

%--#] else : 
%--#[ elseif :

\section{\#elseif}
\label{preelseif}

\noindent Syntax:

\#elseif ( condition )
 
\noindent See also if (\ref{preif}),
            endif (\ref{preendif}),
            else (\ref{preelse})

\noindent The syntax\index{\#elseif} of the condition is identical to the 
syntax for the condition in the \#if\index{\#if} instruction. The \#elseif 
instruction can occur between an \#if and an \#endif\index{\#endif} 
instruction, before a possible matching \#else\index{\#else} instruction. 
The code after this condition till the next \#elseif instruction, or till a 
\#else instruction or till a \#endif instruction, whatever comes first, 
will be read if the condition in the \#elseif instruction is true and none 
of the conditions in matching previous \#if or \#elseif instructions were 
true. The reading is stopped after the matching \#elseif/\#else/\#endif is 
encountered and continued after the matching \#endif instruction.

Example
\begin{verbatim}
    #if ( `i' == 2 )
        some code
    #elseif ( `i' == 3 )
        more code
    #elseif ( `j' >= "x2y" )
        more code
    #else
        more code
    #endif
\end{verbatim}

%--#] elseif : 
%--#[ enddo :

\section{\#enddo}
\label{preenddo}

\noindent Syntax:

\#enddo
 
\noindent See also do (\ref{predo})

\noindent Used to\index{\#enddo} terminate\index{terminate} a preprocessor 
do\index{do loop} loop. See the \#do\index{\#do} instruction.

%--#] enddo : 
%--#[ endif :

\section{\#endif}
\label{preendif}

\noindent Syntax:

\#endif
 
\noindent See also if (\ref{preif}),
            else (\ref{preelse}),
            elseif (\ref{preelseif}),
            ifdef (\ref{preifdef}),
            ifndef (\ref{preifndef})

\noindent Used to terminate\index{\#endif} a \#if\index{\#if}, 
\#ifdef\index{\#ifdef} or \#ifndef\index{\#ifndef} construction. 
Reading will continue after it.

%--#] endif : 
%--#[ endinside :

\section{\#endinside}
\label{preendinside}

\noindent Syntax:

\#endinside
 
\noindent See also \#inside (\ref{preinside})

\noindent Used to\index{\#endinside} terminate a \#inside construction in 
the preprocessor. For more details, see the \#inside\index{\#inside} 
instruction.

%--#] endinside : 
%--#[ endprocedure :

\section{\#endprocedure}
\label{preendprocedure}

\noindent Syntax:

\#endprocedure
 
\noindent See also procedure (\ref{preprocedure}), call
(\ref{precall})

\noindent Each procedure\index{procedure} must be terminated by an 
\#endprocedure\index{\#endprocedure} instruction. If the procedure resides 
in its own file, the \#endprocedure will cause the closing of the file. 
Hence any text that is in the file after the \#endprocedure instruction 
will be ignored.

When control reaches the \#endprocedure instruction, all (local) 
preprocessor variables\index{variables!preprocessor} that were defined 
inside the procedure and all parameters of the call of the procedure will 
become undefined.

%--#] endprocedure : 
%--#[ endswitch :

\section{\#endswitch}
\label{preendswitch}

\noindent Syntax:

\#endswitch
 
\noindent See also switch (\ref{preswitch}),
        case (\ref{precase}),
        break (\ref{prebreak}),
        default (\ref{predefault})

\noindent This instruction marks the end\index{\#endswitch} of a 
\#switch\index{\#switch} construction. After none or one of the cases of 
the \#switch construction has been included in the control flow, reading 
will continue after the matching \#endswitch instruction. Each \#switch 
needs a \#endswitch, unless a .end instruction is encountered first.

%--#] endswitch : 
%--#[ exchange :

\section{\#exchange}
\label{preexchange}

\noindent Syntax:

\#exchange expr1,expr2

\#exchange \$var1,\$var2

\noindent Exchanges\index{\#exchange} the names of two 
expressions\index{expression}. This means that the contents of the 
expressions remain where they are. Hence the order in which the expressions 
are processed remains the same, but the name under which one has to refer 
to them has been changed.

In the variety with the dollar variables\index{\$-variable} the contents of 
the variables are exchanged. This is not much work, because dollar 
variables reside in memory and hence only two pointers to the contents have 
to be exchanged (and some extra information about the contents).

This instruction can be very useful when sorting expressions or dollar 
variables by their contents.

%--#] exchange : 
%--#[ external :

\section{\#external}
\label{preexternal}

\noindent Syntax:

\#external ["prevar"] systemcommand

\noindent Starts the command\index{\#external} in the background, 
connecting to its standard\index{standard output}\index{standard input} 
input\index{input!standard} and output\index{output!standard}. By default, 
the \#external command has no controlling terminal, the standard error stream 
is redirected to \verb|/dev/null| and the command is run in a subshell in a 
new session and in a new process group (see the preprocessor instruction 
\verb|#setexternalattr|).

The optional parameter ``prevar'' is the name of a preprocessor variable 
placed between double quotes. If it is present, the ``descriptor'' (small 
positive integer number) of the external command is stored into this 
variable and can be used for references to this external command (if there 
is more than one external command running simultaneously).

The external command that is started last becomes the ``current'' (active) 
external command.  All further instructions 
\#fromexternal\index{\#fromexternal} and \#toexternal\index{\#toexternal} 
deal with the current external command.

%--#] external : 
%--#[ factdollar :

\section{\#factdollar}
\label{prefactdollar}

\noindent Syntax:

\#factdollar \$-variable
 
\noindent See also the chapters on polynomials \ref{polynomials} and 
\$-variables \ref{dollars}

\noindent The \#factdollar\index{\#factdollar} instruction causes the 
factorization of the indicated \$-variable. After this instruction and 
until the \$-variable is redefined there will be two versions of the 
variable: one is the original unfactorized version and the other is a list 
of factors. If the name of the variable is \$a the factors can be accessed 
as $\$a[1],\cdots,\$a[n]$. The total number of factors is given by 
$\$a[0]$. These factors can also be treated as preprocessor variables by 
putting them between quotes as in `$\$a[2]$'.

%--#] factdollar : 
%--#[ fromexternal :

\section{\#fromexternal}
\label{prefromexternal}

\noindent Syntax:

\#fromexternal[$+-$] ["[\$]varname" [maxlength]]

\noindent Appends\index{\#fromexternal} the output of the current external 
command to the \FORM\ program. The semantics differ depending on the optional 
arguments. After the external command sends the prompt\index{prompt}, \FORM\ 
will continue with a next line after the line containing the \#fromexternal 
instruction. The prompt string is not appended. The optional $+$ or $-$ sign 
after the name has influence on the listing of the content. The varieties 
are:
                                                           
\#fromexternal[$+-$]

\noindent The semantics is similar to the \#include\index{\#include} 
instruction but folders are not supported. 

\#fromexternal[$+-$] "[\$]varname"

\noindent is used to read the text from the running external command into 
the preprocessor variable varname, or into the dollar variable \$varname if 
the name of the variable starts with the dollar sign ``\$''.

\#fromexternal[$+-$] "[\$]varname" maxlength

\noindent is used to read the text from the running external command into 
the preprocessor (or dollar) variable varname. Only the first maxlength 
characters are stored.

%--#] fromexternal : 
%--#[ if :

\section{\#if}
\label{preif}

\noindent Syntax:

\#if ( condition )
 
\noindent See also endif (\ref{preendif}),
            else (\ref{preelse}),
            elseif (\ref{preelseif}),
            ifdef (\ref{preifdef}),
            ifndef (\ref{preifndef})

\noindent The \#if\index{\#if} instruction should be accompanied by a 
matching \#endif\index{\#endif} instruction. In addition there can be 
between the \#if and the \#endif some \#elseif\index{\#elseif} instructions 
and/or a single \#else\index{\#else} instruction. The condition is a 
logical variable that is true if its value is not equal to zero, and false 
if its value is zero. Hence it is allowed to use
\begin{verbatim}
    #if `i'
        statements
    #endif
\end{verbatim}
provided that i has a value which can be interpreted as a number. If there 
is just a string that cannot be seen as a logical\index{logical} condition 
or a number it will be interpreted as false. The regular syntax of the 
simple condition is
\begin{verbatim}
    #if `i' == st2x
        statements
    #endif
\end{verbatim}
or
\begin{verbatim}
    #if ( `i' == st2x )
        statements
    #endif
\end{verbatim}
in which the compare is a numerical compare if both strings can be seen as 
numbers, while it will be a string compare if at least one of the two 
cannot be seen as a numerical object. One can also use more complicated 
conditions as in
\begin{verbatim}
    #if ( ( `i' > 5 ) && ( `j' > `i' ) )
\end{verbatim}
These are referred to as composite conditions. The possible operators are
\begin{description}
\item[$>$] Greater than, either in numerical or in lexicographical sense.
\item[$<$] Less than, either in numerical or in lexicographical sense.
\item[$>=$] Greater than or equal to, either in numerical or in 
lexicographical sense.
\item[$<=$] Less than or equal to, either in numerical or in 
lexicographical sense.
\item[$==$ or $=$] Equal to.
\item[$!=$] Not equal to.
\item[$\&\&$] Logical and operator to combine conditions.
\item[$||$] Logical or operator to combine conditions.
\end{description}

If the condition evaluates to true, the lines after the \#if instruction 
will be read until the first matching \#elseif instruction, or a \#else 
instruction or a \#endif instruction, whatever comes first. After such an 
instruction is encountered input reading stops and continues after the 
matching \#endif instruction.

Like with the regular if-statement (see \ref{substaif}), there are some special 
functions that allow the asking of questions about objects. These are
 
\leftvitem{3cm}{exists()}
\rightvitem{13cm}{The argument of exists\index{exists} is the name of an 
expression or a \$-variable. This function then returns one if this object 
exists, cq. has been defined. Otherwise it returns zero. }
 
\leftvitem{3cm}{isfactorized()}
\rightvitem{13cm}{The argument of isfactorized\index{isfactorized} is the 
name of an expression or a \$-variable. This function then returns one if 
the object has been factorized. Otherwise it returns zero. }

\leftvitem{3cm}{isnumerical()}
\rightvitem{13cm}{The argument of isnumerical\index{isnumerical} is the 
name of an expression or a \$-variable. This function then returns one if 
the object contains a single term that is purely numerical in nature. 
Otherwise it returns zero. }
 
\leftvitem{3cm}{maxpowerof()}
\rightvitem{13cm}{The argument of maxpowerof\index{maxpowerof} is the name 
of a symbol. This function then evaluates into the maximum power of that 
symbol as it has been declared. If no maximum power has been set in the 
declaration of the symbol, the general maximum power for symbols is 
returned (see \ref{substasymbols}).}

\leftvitem{3cm}{minpowerof()}
\rightvitem{13cm}{The argument of minpowerof\index{minpowerof} is the name 
of a symbol. This function then evaluates into the minimum power of that 
symbol as it has been declared. If no minimum power has been set in the 
declaration of the symbol, the general minimum power for symbols is 
returned (see \ref{substasymbols}).}
 
\leftvitem{3cm}{termsin()}
\rightvitem{13cm}{The argument of termsin\index{termsin} is the name of an 
expression or a \$-variable. This function then evaluates into the number 
of terms in that expression.}

%--#] if : 
%--#[ ifdef :

\section{\#ifdef}
\label{preifdef}

\noindent Syntax:

\#ifdef `prevar'
 
\noindent See also if (\ref{preif}),
            endif (\ref{preendif}),
            else (\ref{preelse}),
            ifndef (\ref{preifndef})

\noindent If the named\index{\#ifdef} preprocessor variable has been 
defined the condition is true, else it is false. For the rest the 
instruction behaves like the \#if\index{\#if} instruction.

%--#] ifdef : 
%--#[ ifndef :

\section{\#ifndef}
\label{preifndef}

\noindent Syntax:

\#ifndef `prevar'
 
\noindent See also if (\ref{preif}),
            endif (\ref{preendif}),
            else (\ref{preelse}),
            ifdef (\ref{preifdef})

\noindent If the named\index{\#ifndef} preprocessor variable has been 
defined the condition is false, else it is true. For the rest the 
instruction behaves like the \#if\index{\#if} instruction.

%--#] ifndef : 
%--#[ include :

\section{\#include}
\label{preinclude}

\noindent Syntax:

\#include[$-+$] filename

\#include[$-+$] filename \# foldname

\noindent The named\index{\#include} file is searched for and opened. 
Reading\index{reading} continues from this file until its end. Then the 
file will be closed and reading continues after the \#include instruction. 
If a foldname\index{foldname} is specified, \FORM\ will only read the 
contents of the first fold\index{fold} it encounters in the given file that 
has the specified name.

The file is searched for in the current directory, then in the path 
specified in the path\index{path} variable in the setup file or at the 
beginning of the program (see chapter \ref{setup} on the setup file). Next 
it will look in the path specified in the -p option when \FORM\ is started 
(see the chapter on running \FORM). If this option has not been used, \FORM\ 
will look for the environment variable FORMPATH\index{FORMPATH}. If this 
variable exists it will be interpreted as a path and \FORM\ will search the 
indicated directories for the given file. If none is found there will be an 
error message and execution will be halted.

The optional $+$ or $-$ sign after the name has influence on the listing of the 
contents of the file. A $-$ sign will have the effect of a \#$-$ instruction 
during the reading of the file. A plus sign will have the effect of a \#$+$ 
instruction during the reading of the file.

A fold is defined by a starting line of the format:
\begin{verbatim}
    *--#[ name :
\end{verbatim}
and a closing line of the format
\begin{verbatim}
    *--#] name :
\end{verbatim}
in which the first character is actually the current 
commentary\index{commentary} character (see the \#commentchar instruction). 
All lines between two such lines are considered to be the contents of the 
fold. If \FORM\ decides that it needs this fold, it will read these contents 
and put them in its input stream. More about folds is explained in the 
manual of the STedi editor which is also provided in the \FORM\ 
distribution.

%--#] include : 
%--#[ inside :

\section{\#inside}
\label{preinside}

\noindent Syntax:

\#inside \$var1 [more \$variables]
 
\noindent See also \#endinside (\ref{preendinside})

\noindent Used to\index{\#inside} execute a few statements on the contents 
of one or more dollar variables (see \ref{dollars}) during compilation time.
Although this is a preprocessor instruction one can use the 
triple dot operator provided one uses the generic version with the $<>$.

\noindent The statements in the scope of the \#inside / \#endinside 
construction must be regular executable statements. They may not contain 
end-of-module instructions like the .sort instruction. It is allowed to use 
dollar variables, procedures and preprocessor do loops and if's, but it is 
not allowed to nest the \#inside / \#endinside constructions.

%--#] inside : 
%--#[ message :

\section{\#message}
\label{premessage}

\noindent Syntax:

\#message themessagestring

\noindent This instruction places a message\index{\#message} in the output 
that is clearly marked as such. It is printed with an initial three 
characters in front as in
\begin{verbatim}
    Symbols a,b,c;
    #message Simple example;
~~~Simple example;
    Local F = (a+b+c)^10;
    .end

Time =       0.00 sec    Generated terms =         66
                F        Terms in output =         66
                         Bytes used      =       1138
\end{verbatim}
Note that the semicolon\index{semicolon} is not needed and if present is 
printed as well. If one needs messages without this clear marking, one 
should use the \#write\index{\#write} instruction.
 
%--#] message : 
%--#[ optimize :

\section{\#optimize}
\label{preoptimize}

\noindent Syntax:

\#optimize nameofoneexpression

See the chapter about optimization \ref{optimization}
 
%--#] optimize : 
%--#[ pipe :
 
\section{\#pipe}
\label{prepipe}

\noindent Syntax:

\#pipe systemcommand
 
\noindent See also system (\ref{presystem})

\noindent This\index{\#pipe} forces a system command to be executed by the 
operating system. The complete string (excluding initial blanks or tabs) is 
passed to the operating system. Next \FORM\ will intercept the output of 
whatever is produced and read that as input. Hence, whenever output is 
produced \FORM\ will take action, and it will wait when no output is ready. 
After the command has been finished, \FORM\ will continue with the next line. 
This instruction has only been implemented on systems that support 
pipes\index{pipe}. This is mainly UNIX\index{UNIX} and derived systems. 
Note that this instruction also introduces operating system dependent code. 
Hence it should be used with great care.

%--#] pipe : 
%--#[ preout :
 
\section{\#preout}
\label{prepreout}

\noindent Syntax:

\#preout ON

\#preout OFF

\noindent Turns\index{\#preout} listing of the output of the preprocessor 
to the compiler on or off. Example:
% THIS EXAMPLE IS PART OF THE TESTSUITE. CHANGES HERE SHOULD BE APPLIED THERE AS
% WELL!
\begin{verbatim}
    #PreOut ON
    S   a1,...,a4;
 S,a1,a2,a3,a4
    L   F = (a1+...+a4)^2;
 L,F=(a1+a2+a3+a4)^2
    id  a4 = -a1;
 id,a4=-a1
    .end

Time =       0.00 sec    Generated terms =         10
                F        Terms in output =          3
                         Bytes used      =         52
\end{verbatim}

%--#] preout : 
%--#[ printtimes :
 
\section{\#printtimes}
\label{preprinttimes}

\noindent Syntax:

\#printtimes

\noindent Prints\index{\#printtimes} the current execution time and real 
time in the same way as done at the end of the program. Helps in monitoring 
the real time passed in TFORM jobs.
Example:
\begin{verbatim}
    #Printtimes
  423.59 sec + 5815.88 sec: 6239.47 sec out of 1215.29 sec
\end{verbatim}

%--#] printtimes : 
%--#[ procedure :

\section{\#procedure}
\label{preprocedure}

\noindent Syntax:

\#procedure name(var1,...,varn)
 
\noindent See also endprocedure (\ref{preendprocedure}), call
(\ref{precall})

\noindent Name\index{\#procedure} is the name of the 
procedure\index{procedure}. It will be referred to by this name. If the 
procedure resides in a separate file the name of the file should be 
name.prc and the \#procedure instruction should form the first line of the 
file. The \# should be the first character of the file. The parameter field 
is optional. If there are no parameters, the procedure should also be 
called without parameters (see the \#call instruction). The parameters 
(here called var1 to varn) are preprocessor variables and hence they should 
be referred to between a backquote\index{backquote}/quote\index{quote} pair 
as in `var1' to `varn'. If there exist already variables with such names 
when the procedure is called, the new definition comes on top of the old 
one. Hence in the procedure (and procedures called from it, unless the same 
problems occurs there too, as would be the case with recursions) the new 
definition is used, and it is released again when control returns from the 
procedure. After that the old definition will be in effect again.

If the procedure is included in the regular input stream, \FORM\ will read 
the text of the procedure until the \#endprocedure\index{\#endprocedure} 
instruction and store it in a special buffer. When the procedure is called, 
\FORM\ will read the procedure from this buffer, rather than from a file. In 
systems where file transfer is slow (very busy server with a slow network) 
this may be faster, especially when many small procedures are called.

One way to make libraries\index{library!making a}\index{library} that 
contain many procedures and maybe more code is to put all procedures into 
one header (.h) file and include this file at the beginning of the program 
with a \#include\index{\#include} instruction. This way one has all 
procedures load and one knows for sure that it are the proper procedures as 
it guards against the inadvertently picking up of procedures from other 
directories. It also makes for fewer files and hence makes for better 
housekeeping.

%--#] procedure : 
%--#[ procedureextension :

% NEW@@@

\section{\#procedureextension}
\label{preprocedureextension}

\noindent Syntax:

\#procedureextension string
 
\noindent See also \#call (\ref{precall})

\noindent The default\index{\#procedureextension} extension of procedures 
is .prc\index{.prc} in \FORM. It is however possible that this clashes with 
the extensions used by other programs like the Grace\index{Grace} system 
(Yuasa et al, Prog. Theor. Phys. Suppl. 138(2000)18 ). In that case it is 
possible to change the extension of the procedures in the current program. 
This is either done via the setup (page \ref{setup}) or by the 
\#procedureextension instruction of the preprocessor. The new string 
replaces the string prc, used by default. For the new string the following 
restrictions hold:
\begin{enumerate}
\item The first character must be alphabetic
\item No whitespace characters (blanks and/or tabs) are allowed
\end{enumerate}
For the rest any characters can be used.

\noindent The new extension will remain valid either till the next 
\#procedureextension instruction or to the next .clear\index{.clear} 
instruction (page \ref{instrclear}), whatever comes first.

%--#] procedureextension : 
%--#[ prompt :

\section{\#prompt}
\label{preprompt}

\noindent Syntax:

\#prompt [newprompt]

\noindent Sets a new prompt\index{\#prompt} for the current external 
command (if present) and all further (newly started) external commands.

If newprompt is an empty string, the default prompt (an empty line) will be 
used.

The prompt\index{prompt} is a line consisting of a single prompt string. By 
default, this is an empty string.

%--#] prompt : 
%--#[ redefine :

\section{\#redefine}
\label{preredefine}

\noindent Syntax:

\#redefine name "string"
 
\noindent See also define (\ref{predefine}), undefine 
(\ref{preundefine})

\noindent in which\index{\#redefine} name refers to the name of the 
preprocessor\index{preprocessor variable} 
variable\index{variable!preprocessor} to be redefined. The contents of the 
string will be its new value. If no variable of the given name exists yet, 
the instruction will be equivalent to the \#define\index{\#define} 
instruction.

%--#] redefine : 
%--#[ remove :
 
\section{\#remove}
\label{preremove}

\noindent Syntax:

\#remove $<$filename$>$
 
\noindent See also write (\ref{prewrite}), append (\ref{preappend}),
            create (\ref{precreate}), close (\ref{preclose})

\noindent Deletes\index{\#remove} the named file from the system. Under 
UNIX\index{UNIX} this would be equivalent to the instruction
\begin{verbatim}
    #system rm filename
\end{verbatim}
and under MS-DOS\index{MS-DOS} oriented systems like Windows\index{Windows} 
it would be equivalent to
\begin{verbatim}
    #system del filename
\end{verbatim}
The difference with the \#system\index{\#system} instruction is that the 
\#remove\index{\#remove} instruction does not depend on the particular 
syntax of the operating system. Hence the \#remove instruction can always 
be used.

%--#] remove : 
%--#[ reverseinclude :

\section{\#reverseinclude}
\label{prereverseinclude}

\noindent Syntax:

\#reverseinclude[$-+$] filename

\#reverseinclude[$-+$] filename \# foldname

\noindent This instruction is identical to the \#include \ref{preinclude} 
instruction, with the exception that the statements and instructions in the 
file are read in reverse order. This can be useful at times when code is 
generated in a particular order in a file and one would like to 'undo' this 
code. It is somewhat related to the effects of the debugflag option 
(\ref{optimdebugflag}) in the optimization options of the format statement 
\ref{optimization}.

There are a few limitations. If, for instance, linefeeds or semicolons 
occur inside preprocessor variables, the reading routines cannot see this. 
Additionally unfinished strings (unmatched double quotes) will result in 
a fatal error. On the other hand the fold structure remains preserved.

%--#] reverseinclude : 
%--#[ rmexternal :

\section{\#rmexternal}
\label{prermexternal}

\noindent Syntax:

\#rmexternal [n]

\noindent Terminates\index{\#rmexternal} an external command. The integer 
number n must be either the descriptor of a running external command, or 0.

If n is 0, then all external programs will be terminated.

If n is not specified, the current external command will be terminated.

The action of this instruction depends on the attributes of the external 
channel (see the \#setexternalattr\index{\#setexternalattr} (section 
\ref{setexternalcommunication}) instruction). By default, the instruction 
closes the commands' IO channels, sends a KILL\index{KILL signal} signal to 
every process in its process group and waits for the external command to be 
finished.

%--#] rmexternal : 
%--#[ rmseparator :
 
\section{\#rmseparator}
\label{prermseparator}

\noindent Syntax:

\#rmseparator character
 
\noindent See also \#addseparator (\ref{preaddseparator}),
            \#call (\ref{precall}), \#do (\ref{predo})

\noindent Removes a character\index{\#rmseparator} from the list of permissible 
separator characters for arguments of \#call or \#do instructions. By 
default the two characters that are permitted are the comma and the 
character \verb:|:. Blanks, tabs and double quotes are ignored. Note that 
the comma must be specified between double quotes as in
\begin{verbatim}
  #rmseparator ","
\end{verbatim}

%--#] rmseparator : 
%--#[ setexternal :

\section{\#setexternal}
\label{presetexternal}

\noindent Syntax:

\#setexternal n

\noindent Sets\index{\#setexternal} the ``current'' external command. The 
instructions \#toexternal\index{\#toexternal} and 
\#fromexternal\index{\#fromexternal} deal with the current external 
command.  The integer number n must be the descriptor of a running external 
command.

%--#] setexternal : 
%--#[ setexternalattr :

\section{\#setexternalattr}
\label{presetexternalattr}

\noindent Syntax:

\#setexternalattr list\_of\_attributes

\noindent sets\index{\#setexternalattr} attributes for {\em newly started} 
external commands. Already running external commands are not affected. The 
list of attributes is a comma separated list of pairs attribute=value, 
e.g.:
\begin{verbatim}
   #setexternalattr shell=noshell,kill=9,killall=false
\end{verbatim}
Possible attributes are:
\begin{description}
\item[kill\index{kill}]
specifies the signal to be sent to the external command 
either before the termination of the \FORM\ program or by the preprocessor 
instruction \verb|#rmexternal|. By default this is 9 (
SIGKILL\index{SIGKILL signal}). Number 0 means that no signal will be sent.
\item[killall\index{killall}] Indicates whether the kill signal will be sent to the whole 
group or only to the initial process. Possible values are ``\verb|true|'' 
and ``\verb|false|''. By default, the kill signal will be sent to the
whole group.
\item[daemon\index{daemon}]
Indicates whether the command should be ``daemonized'', i.e. 
the initial process will be passed to the init process and will belong
to the new process group in the new session.  
Possible values are ``\verb|true|'' and ``\verb|false|''. By default, 
``\verb|true|''.
\item[shell\index{shell}]
specifies which shell\index{shell} is used to run a
command. (Starting an external command in a subshell permits to
start not only executable files but also scripts\index{script} and 
pipelined\index{pipelined job} jobs. The disadvantage is that there is no 
way to detect failure upon startup since usually the shell is started 
successfully.) By default this is ``\verb|/bin/sh -c|''.  If set 
\verb|shell=noshell|, the command will be stared by the instruction 
\#external\index{\#external} directly but not in a subshell, so the command 
should be a name of the executable file rather than a system command. The 
instruction \#external will duplicate the actions of the shell in searching 
for an executable file if the specified file name does not contain a slash 
(/) character.  The search path is the path specified in the environment by 
the PATH\index{PATH} variable.  If this variable isn't specified, the 
default path ``\verb|:/bin:/usr/bin|''
is used.
\item[stderr\index{stderr}]
specifies a file to redirect the standard\index{standard error} error 
stream to. By default it is ``\verb|/dev/null|''. If set 
\verb|stderr=terminal|, no redirection occurs.
\end{description}
Only attributes that are explicitly mentioned are changed, all others remain 
unchanged. Note, changing attributes should be done with care. For example,
\begin{verbatim}
   #setexternalattr daemon=false
\end{verbatim}
starts a command in the subshell within the current process group with
default attributes kill=9 and killall=true.
The instruction \#rmexternal\index{\#rmexternal} sends the
KILL\index{KILL signal} signal to the wholegroup, which means that also 
\FORM\ itself will be killed.

%--#] setexternalattr : 
%--#[ setrandom :

\section{\#setrandom}
\label{presetrandom}

\noindent Syntax:

\#setrandom number
 
\noindent See also random\_ (\ref{funrandom}) and ranperm\_ (\ref{funranperm})

\noindent The \#setrandom\index{\#setrandom} instruction initializes the 
random number generator 
random\_\ref{funrandom}\index{random\_}\index{function!random\_}. The 
number that is used as a seed can have the length of two words in FORM. 
This means that on a 32-bits computer it can be an (unsigned) 32-bits 
integer and on a 64-bits computer it can be an (unsigned) 64 bits integer. 
If there is no \#setrandom instruction the random number generator is 
initialized in a built in standard way. The \#setrandom instruction also 
initializes the random number generators of the workers when one uses TFORM 
or ParFORM. They are initialized with different seeds that are derived in a 
non-trivial way from the seed given by the user and the number of the 
worker.

%--#] setrandom : 
%--#[ show :

\section{\#show}
\label{preshow}

\noindent Syntax:

\#show [preprocessorvariablename[s]]

\noindent If no names\index{\#show} are present, the contents of all 
preprocessor variables\index{variable!preprocessor} will be printed to the 
regular output. If one or more preprocessor variables are specified 
(separated by comma's), only their contents will be printed. The 
preprocessor variables should be represented by their name only. No 
enclosing backquote/quote should be used, because that would force a 
substitution of the preprocessor variable before the instruction gets to 
see the name. Example:
\begin{verbatim}
    #define MAX "3"
    Symbols a1,...,a`MAX';
    L F = (a1+...+a`MAX')^2;
    #show
#The preprocessor variables:
0: VERSION_ = "3"
1: SUBVERSION_ = "2"
2: NAMEVERSION_ = ""
3: DATE_ = "Wed Feb 28 08:43:20 2007"
4: NAME_ = "testpre.frm"
5: CMODULE_ = "1"
6: MAX = "3"
    .end

Time =       0.00 sec    Generated terms =          6
                F        Terms in output =          6
                         Bytes used      =        102
\end{verbatim}
We see that the variable MAX has indeed the value 3. There are six 
additional variables which have been defined by \FORM\ itself. Hence the 
trailing underscore which cannot be used in user defined names. The current 
version of \FORM\ is shown in the variable VERSION\_\index{VERSION\_} and the 
name of the current program is given in the variable NAME\_\index{NAME\_}. 
For more about the system defined preprocessor variables see 
\ref{preprovariables}.

There is another preprocessor variable that does not show in the listings. 
Its name is SHOWINPUT\_\index{SHOWINPUT\_}. This variable has the value one 
if the listing of the input is on and the value zero if the listing of the 
input is off.

%--#] show : 
%--#[ switch :

\section{\#switch}
\label{preswitch}

\noindent Syntax:

\#switch string
 
\noindent See also endswitch (\ref{preendswitch}),
        case (\ref{precase}),
        break (\ref{prebreak}),
        default (\ref{predefault})

\noindent the\index{\#switch} string could for instance be a preprocessor 
variable as in
\begin{verbatim}
    #switch `i'
\end{verbatim}
The \#switch\index{\#switch} instruction, together with 
\#case\index{\#case}, \#break\index{\#break}, \#default\index{\#default} 
and \#endswitch\index{\#endswitch}, allows the user to conveniently make 
code for a number of cases that are distinguished by the value of a 
preprocessor variable. In the past this was only possible with the use of 
folds\index{folds} in the \#include\index{\#include} instruction and the 
corresponding include file\index{file!include} (see \ref{preinclude}). 
Because few people have an editor like STedi (see the \FORM\ distribution 
site) that can handle the folds in a proper way, it was judged that the 
more common switch mechanism might be friendlier. The proper syntax of a 
complete construction would be
\begin{verbatim}
    #switch `par'
    #case 1
       some statements
    #break
    #case ax2
       other statements
    #break
    #default
       more statements
    #break
    #endswitch
\end{verbatim}
The number of cases is not limited. The compare between the strings in the 
\#switch instruction and in the \#case instructions is as a text string. 
Hence numerical strings have no special meaning. If a \#break instruction 
is omitted, control may go into another case. This is called 
fall-through\index{fall-through}. 
This is a way in which one can have the same statements for several cases. 
The \#default instruction is not mandatory.

\FORM\ will look for the first case of which the string matches the string 
in the \#switch instruction. Input reading (control flow) starts after this 
\#case instruction, and continues till either a \#break instruction is 
encountered, or the \#endswitch is met. After that input reading continues 
after the \#endswitch instruction. If no case has a matching string, input 
reading starts after the \#default instruction. If no \#default instruction 
is found, input reading continues after the matching \#endswitch 
instruction.

\#switch constructions can be nested\index{nested}. They can be combined 
with \#if\index{\#if} constructions, \#do\index{\#do} instructions, etc. 
but they should obey normal nesting rules (as with nesting of 
brackets\index{bracket} of different types).

%--#] switch : 
%--#[ system :
 
\section{\#system}
\label{presystem}

\noindent Syntax:

\#system systemcommand
 
\noindent See also pipe (\ref{prepipe})

\noindent This forces a system\index{\#system} command to be executed by 
the operating system. The complete string (excluding initial blanks or 
tabs) is passed to the operating system. \FORM\ will then wait until control 
is returned. Note that this instruction introduces operating system 
dependent code. Hence it should be used with great care.

%--#] system : 
%--#[ terminate :
 
\section{\#terminate}
\label{preterminate}

\noindent Syntax:

\#terminate [exitcode]
 
\noindent This forces \FORM\ to terminate\index{\#terminate} execution 
immediately. If an exit code is given (an integer number), this will be the 
return value that \FORM\ gives to the shell program from which it was run. If 
no return value is specified, the value -1 will be returned.

%--#] terminate : 
%--#[ toexternal :

\section{\#toexternal}
\label{pretoexternal}

\noindent Syntax:

\#toexternal "formatstring" $<$,variables$>$

\noindent Sends\index{\#toexternal} the output to the current external 
command. The semantics of the \verb|"formatstring"| and the
\verb|[,variables]| is the same as for the \#write\index{\#write} 
instruction, except for the trailing end-of-line symbol. In contrast to the 
\#write instruction, the \#toexternal instruction does not append any new 
line symbol to the end of its output.

%--#] toexternal : 
%--#[ undefine :

\section{\#undefine}
\label{preundefine}

\noindent Syntax:

\#undefine name
 
\noindent See also define (\ref{predefine}), redefine 
(\ref{preredefine})

\noindent \index{\#undefine} Name refers to the name of the 
preprocessor variable\index{variable!preprocessor} to be undefined. This 
statement causes the given preprocessor variable to be removed from the 
stack of preprocessor variables. If an earlier instance of this variable 
existed (other variable with the same name), it will become active again. 
There are various other ways by which preprocessor variables can become 
undefined. All variables belonging to a procedure are undefined at the end 
of a procedure, and so are all other preprocessor variables that were 
defined inside this procedure. The same holds for the preprocessor variable 
that is used as a loop parameter in the \#do\index{\#do} instruction.

%--#] undefine : 
%--#[ write :

\section{\#write}
\label{prewrite}

\noindent Syntax:

\#write [$<$filename$>$] "formatstring" [,variables]
 
\noindent See also append (\ref{preappend}),
            create (\ref{precreate}), remove (\ref{preremove}),
            close (\ref{preclose})

\noindent If there\index{\#write} is no file specified, the output will be 
to the regular output\index{output channel} channel. If a file is 
specified, \FORM\ will look whether this file is open already. If it is open 
already, the specified output will be added to the file. If it is not open 
yet it will be opened. Any previous contents will be lost. This would be 
equivalent to using the \#create\index{\#create} instruction first. If 
output has to be added to an existing file, the \#append\index{\#append} 
instruction should be used first.

The format\index{format string} string is like a format string in the 
language C\index{C}. This means 
that it is placed between double quotes. It will contain text that will be 
printed, and it will contain special character sequences for special 
actions. These sequences and the corresponding actions are:
\begin{description}
\item[$\backslash$n] A newline\index{newline} character.
\item[$\backslash$t] A tab\index{tab} character.
\item[$\backslash$"] A double\index{double quote} quote character.
\item[$\backslash$b] A backslash\index{backslash} character.
\item[\%\%] The character \%\index{\%}.
\item[\%] If the last character in the string, it causes the omission of a 
linefeed\index{linefeed} at the end of the printing. Note that if this 
happens in the regular output (as opposed to a file) there may be 
interference with the listing of the input.
\item[\%\$] A dollar variable\index{\$-variable}. The variable should be 
indicated in the list of variables. Each occurrence of \%\$ will look for 
the next variable.
\item[\%e] An active expression\index{expression}. The expression should be 
indicated in the list of variables. Each occurrence of \%e will look for 
the next variable. Unlike the output caused by the print statement the 
expression will be printed without its name and there will also be no 
\verb:=: sign unless there is one in the format string of course. If the 
current output format is fortran\index{fortran} output there is an extra option. After the 
name of the expression one should put between parentheses the name to be 
used when there are too many continuation cards.
\item[\%E] Like \%e, but whereas the \%e terminates the expression with a 
;, the \%E does not give this trailing semicolon\index{semicolon}.
\item[\%s] A string\index{string}. The string should be 
given in the list of variables and be enclosed between double quotes. Each 
occurrence of \%s will look for the next variable in the list.
\item[\%f] A file\index{file}. The name of the file will be expected in the 
list of variables. The file is searched for in the current directory, then 
in path indicated by the path variable in the setup file or at the 
beginning of the file (see chapter \ref{setup} on the setup file), then in 
the path specified in the -p option when \FORM\ is started (see the chapter 
on running \FORM). If this option has not been used, \FORM\ will look for the 
environment variable FORMPATH\index{FORMPATH}. If this variable exists it 
will be interpreted as a path and \FORM\ will search the indicated 
directories for the given file. If none is found there will be an error 
message and execution will be halted.
\item[\%X] Forces the printing of the list of extra symbols 
(\ref{sect-extrasymbols}) and their definitions\index{extrasymbols}.
\item[\%O] Forces the printing of the definitions of the extra symbols in 
the buffer with the temporary variables from the previous optimization (see 
the chapter on optimizations \ref{optimization}). 
\end{description}
If no special variables are asked for (by means of \%\$, \%e, \%E or \%s) 
the list of variables will be ignored (if present). Example:
% THIS EXAMPLE IS PART OF THE TESTSUITE. CHANGES HERE SHOULD BE APPLIED THERE AS
% WELL!
\begin{verbatim}
    Symbols a,b;
    L   F = a+b;
    #$a1 = a+b;
    #$a2 = (a+b)^2;
    #$a3 = $a1^3;
    #write " One power: %$\n Two powers: %$\n Three powers: %$\n%s"\
           ,$a1,$a2,$a3," The end"
 One power: b+a
 Two powers: b^2+2*a*b+a^2
 Three powers: b^3+3*a*b^2+3*a^2*b+a^3
 The end
    .end

Time =       0.00 sec    Generated terms =          2
                F        Terms in output =          2
                         Bytes used      =         32
\end{verbatim}
We see that the writing occurs immediately after the \#write\index{\#write} 
instruction, because it is done by the preprocessor. Hence the output comes 
before the execution of the expression F.
% THIS EXAMPLE IS PART OF THE TESTSUITE. CHANGES HERE SHOULD BE APPLIED THERE AS
% WELL!
\begin{verbatim}
    S   x1,...,x10;
    L   MyExpression = (x1+...+x10)^4;
    .sort
    Format Fortran;
    #write <fun.f> "      FUNCTION fun(x1,x2,x3,x4,x5,x6,x7,x8,x9,x10)"
    #write <fun.f> "      REAL x1,x2,x3,x4,x5,x6,x7,x8,x9,x10"
    #write <fun.f> "      fun = %e",MyExpression(fun)
    #write <fun.f> "      RETURN"
    #write <fun.f> "      END"
    .end
\end{verbatim}
Some remarks are necessary here. Because the \#write is a preprocessor 
instruction, the .sort\index{.sort} is essential. Without it, the 
expression has not been worked out at the moment we want to write. The name 
of the expression is too long for fortran\index{fortran}, and hence the 
output file will use a different name (in this case the name `fun' was 
selected). The output file looks like
% THIS EXAMPLE IS PART OF THE TESTSUITE. CHANGES HERE SHOULD BE APPLIED THERE AS
% WELL!
\begin{verbatim}
      FUNCTION fun(x1,x2,x3,x4,x5,x6,x7,x8,x9,x10)
      REAL x1,x2,x3,x4,x5,x6,x7,x8,x9,x10
      fun = 24*x1*x2*x3*x4 + 24*x1*x2*x3*x5 + 24*x1*x2*x3*x6 + 24*x1*x2
     & *x3*x7 + 24*x1*x2*x3*x8 + 24*x1*x2*x3*x9 + 24*x1*x2*x3*x10 + 12*
          .....
     & x8 + 4*x6**3*x9 + 4*x6**3*x10 + x6**4 + 24*x7*x8*x9*x10 + 12*x7*
     & x8*x9**2
      fun = fun + 12*x7*x8*x10**2 + 12*x7*x8**2*x9 + 12*x7*x8**2*x10 + 
     & 4*x7*x8**3 + 12*x7*x9*x10**2 + 12*x7*x9**2*x10 + 4*x7*x9**3 + 4*
     & x7*x10**3 + 12*x7**2*x8*x9 + 12*x7**2*x8*x10 + 6*x7**2*x8**2 + 
     & 12*x7**2*x9*x10 + 6*x7**2*x9**2 + 6*x7**2*x10**2 + 4*x7**3*x8 + 
     & 4*x7**3*x9 + 4*x7**3*x10 + x7**4 + 12*x8*x9*x10**2 + 12*x8*x9**2
     & *x10 + 4*x8*x9**3 + 4*x8*x10**3 + 12*x8**2*x9*x10 + 6*x8**2*
     & x9**2 + 6*x8**2*x10**2 + 4*x8**3*x9 + 4*x8**3*x10 + x8**4 + 4*x9
     & *x10**3 + 6*x9**2*x10**2 + 4*x9**3*x10 + x9**4 + x10**4

      RETURN
      END
\end{verbatim}
and each time after 19 continuation lines we have to break the expression 
and use the \verb:fun = fun +: trick to continue.

%--#] write : 
%--#[ Some remarks :

\section{Some remarks}
It should be noted that the various constructions like 
\#do\index{\#do}/\#enddo\index{\#enddo}, 
\#procedure\index{\#procedure}/\#endprocedure\index{\#endprocedure}, 
\#switch\index{\#switch}/\#endswitch\index{\#endswitch} and 
\#if\index{\#if}/\#endif\index{\#endif} all 
create a certain environment. These environments cannot be interweaved. This 
means that one cannot make code of the type
\begin{verbatim}
     #do i = 1,5
      #if ( `MAX' > `i' )
         id f(`i') = g`i'(x);
     #enddo
      some statements
     #do i = 1,5
      #endif
     #enddo
\end{verbatim}
whether this could be considered useful or not. Similarly one cannot make a 
construction that might be very useful:
\begin{verbatim}
     #do i = 1,5
       #do j`i' = 1,3
     #enddo
       some statements
     #do i = 1,5
       #enddo
     #enddo
\end{verbatim}
Currently the syntax does not allow this. This may change in the future.

%--#] Some remarks : 


