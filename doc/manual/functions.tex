
\chapter{Functions}
\label{functions}

%--#[ General :

\noindent Functions\index{function} are objects that can have arguments. 
There exist several types of functions in \FORM. First there is the 
distinction between commuting\index{commuting} and 
noncommuting\index{noncommuting} functions. Commuting functions commute 
with all other objects. This property is used by the normalization routines 
that bring terms into standard form. Noncommuting functions do not commute 
necessarily with other noncommuting functions. They do however commute with 
objects that are considered to be commuting, like symbols, vectors and 
commuting functions. Various instances of the same noncommuting function 
but with different arguments do not commute either.

\noindent The next subdivision of the category of functions is in regular 
functions\index{function!regular}, tensors\index{tensor} and 
tables\index{table}. Tensors are special functions that can have only 
indices or vectors for their arguments. If an argument is a vector, it is 
assumed that this vector is there as the result of an index contraction. 
Tables are functions with automatic substitution rules. A table must have 
at least one table\index{table index} index\index{index!table}. Each time 
during normalization \FORM\ will check whether an instance of a table can be 
substituted. This means that undefined table elements will slow the program 
down somewhat.

\noindent All the various types of functions are declared with their own 
declaration statements. These are described in the chapter for the 
statements (see chapter~\ref{statements}).

%--#] General : 
%--#[ Wildcards :

One of the useful properties of functions is the 
wildcarding\index{wildcard} of their arguments during pattern matching. The 
following argument wildcards are possible:

\leftvitem{2cm}{x?}
\rightvitem{14cm}{Here x is a symbol. This symbol can match either a 
symbol, any numerical argument, or a complete subexpression argument that 
is not vectorlike or indexlike.}

\leftvitem{2cm}{i?}
\rightvitem{14cm}{Here i is an index. This index can match either an index, 
a vector (actually the dummy\index{dummy} index\index{index!dummy} of the 
vector that was contracted), or a complete subexpression that is vector like 
(again actually the contracted dummy index).}

\leftvitem{2cm}{v?}
\rightvitem{14cm}{Here v is a vector. This vector can match either a vector 
or a complete subexpression that is vector like.}

\leftvitem{2cm}{f?}
\rightvitem{14cm}{Here f is any functiontype. This function can match any 
function. It is the responsibility of the user to avoid problems in the 
right-hand side if f happens to match a tensor.}

\leftvitem{2cm}{?a}
\rightvitem{14cm}{This is an argument\index{argument field} field 
wildcard\index{wildcard!argument field}. This can match a 
complete set of arguments. The set can be empty. Argument field wildcards 
have a name that starts with a question mark followed by a name. They do 
not have to be declared as there cannot be confusion.}

%--#] Wildcards : 

\noindent In addition to the above syntax \FORM\ knows a number of special 
functions with well defined properties. All these functions have a name 
that ends in an underscore. In addition the names of these built in objects 
are case insensitive. This means for instance that the factorial function 
can be referred to as \verb:fac_:, \verb:Fac_: or \verb:FAC_: or whatever 
the user considers more readable. The built in functions are:

%--#[ abs_ :

\section{abs\_}\index{abs\_}\index{function!abs\_}
\label{funabs}
\noindent With one argument that is numerical it evaluates into the 
absolute value of the argument.

%--#] abs_ : 
%--#[ bernoulli_ :

\section{bernoulli\_}\index{bernoulli\_}\index{function!bernoulli\_}
\label{funbernoulli}
\noindent If it has one nonzero integer argument n, it evaluates into 
the n-th coefficient in the power series expansion of $x/(1-e^{-x})$.

%--#] bernoulli_ : 
%--#[ binom_ :

\section{binom\_}\index{binom\_}\index{function!binom\_}
\label{funbinom}
\noindent binom\_(n,i) $= n!/(i!(n-i)!)$. If the arguments are non 
integer or negative, no substitution is made.

%--#] binom_ : 
%--#[ conjg_ :

\section{conjg\_}\index{conjg\_}\index{function!conjg\_}
\label{funconjg}
\noindent Currently not doing anything.

%--#] conjg_ : 
%--#[ content_ :

\section{content\_}\index{content\_}\index{function!content\_}
\label{funcontent}
\noindent This function expects the name of a single expression or a dollar 
variable for its 
argument. If it finds this the content of this expression or dollar 
variable is returned. The 
content is defined as a term that has
\begin{itemize}
\item for its numerator the GCD of the numerators of all terms in the 
expression.
\item for its denominator the LCM of the denominators of all terms in the 
expression.
\item all the common subexpressions in all terms of the expression.
\item the most negative powers of all symbols and dotproducts with negative 
powers in the terms of the expression.
\end{itemize}
When there are no negative powers and no denominators in the coefficients, 
this definition of the content co\"{\i}ncides with the classical definition 
of the content of a polynomial over the integers. Our content has the 
property that if we divide the expression by it, we are left with an 
expression of which the coefficients are all integer, there are no negative 
powers and the GCD of all terms combined is one.

\noindent This function has one limitation. It will not consider 
noncommuting objects. Neither will it consider denominator functions.

\noindent Caveat: this function is evaluated each time it is encountered. 
Therefore the best thing is to evaluate it once in the definition of a 
dollar variable or an expression as in
\begin{verbatim}
   #$x = content_(F);
   Local G = (a+b)^10*$x;
\end{verbatim}
Here the content is computed only once. In
\begin{verbatim}
   Local G = (a+b)^10*content_(F);
\end{verbatim}
11 terms are generated and the content is only worked out when the 
terms are normalized. This means that it will be evaluated 11 times. If one 
does not like dollar variables and still wants to evaluate the content only 
once the code would be
\begin{verbatim}
   Local G = ab^10*content_(F);
   id ab = a+b;
\end{verbatim}
because now the term will be normalized before the substitution makes it 
into eleven terms. This assumes of course that the content does not contain 
the variable ab.

%--#] content_ : 
%--#[ count_ :

\section{count\_}\index{count\_}\index{function!count\_}
\label{funcount}
\noindent Similar to the count object in the if statement (see 
\ref{substaif}). This function expects the same arguments as the count 
object and returns the corresponding count value for the current term.

%--#] count_ : 
%--#[ d_ :

\section{d\_}\index{d\_}\index{function!d\_}
\label{fund}
\noindent The kronecker\index{kronecker} delta\index{delta!kronecker}. 
Should have two indices for arguments. Often indicated as 
$\delta^{\mu\nu}$. In automatic summation over the indices the d\_ often 
vanishes again as in
\verb:d_(mu,nu)*p(mu)*q(nu): $\rightarrow$ \verb:p.q: and similar 
replacements. Internally this object is treated in a rather special way. 
Hence it will not match a function wildcard.

%--#] d_ : 
%--#[ dd_ :

\section{dd\_}\index{dd\_}\index{function!dd\_}
\label{fundd}
\noindent This is a combinatorics\index{combinatorics} function. The tensor 
dd\_ with an even number of indices is equal to the totally symmetric 
tensor built up from products of kronecker delta's. Each term in this 
symmetric combination is normalized to one. In principle there are 
$n!/(2^{n/2}(n/2)!$ terms in this combination. The profit comes when some 
or all the indices are contracted with vectors and some of these vectors 
are identical. In that case \FORM\ will use combinatorics to generate only 
different terms, each with the proper prefactor. This can result in great 
time and space savings.

%--#] dd_ : 
%--#[ delta_ :

\section{delta\_}\index{delta\_}\index{function!delta\_}
\label{fundelta}
\noindent With one numerical argument the result is one if 
the argument is zero and zero otherwise. With two arguments the result is 
one if the arguments are numerical and identical. If they are numerical and 
they differ the result is zero. In all other cases nothing is done.

%--#] delta_ : 
%--#[ deltap_ :

\section{deltap\_}\index{deltap\_}\index{function!deltap\_}
\label{fundeltap}
\noindent If one argument and it is numerical the result is zero if 
the argument is zero and one otherwise. If two arguments, the result is 
zero if the arguments are numerical and identical. If they are numerical and 
they differ the result is one. In all other cases nothing is done.

%--#] deltap_ : 
%--#[ denom_ :

\section{denom\_}\index{denom\_}\index{function!denom\_}
\label{fundenom}
\noindent Internal function to describe denominators. Has a single 
argument. \verb:den(a+b): is printed as \verb:1/(a+b):.

%--#] denom_ : 
%--#[ distrib_ :

\section{distrib\_}\index{distrib\_}\index{function!distrib\_}
\label{fundistrib}
\noindent This is a combinatorics\index{combinatorics} function. It should 
have at least five arguments. If we have
\begin{verbatim}
    distrib_(type,n,f1,f2,x1,...,xm)
\end{verbatim}
with type and n integers, f1 and f2 functions and then a number of 
arguments there can be action if $-2 \le$ type $\le 2$. The typical action 
is that the arguments \verb:x1,...,xm: will be divided over the two 
functions in all possible ways. For each possibility a new term is 
generated. The relative order of the arguments is kept. If type is negative 
it is assumed that the collection of x-arguments is 
antisymmetric\index{antisymmetric} and hence the number of permutations 
needed to make the split will determine whether there will be a minus sign 
on the resulting term. When type is zero all possible divisions are 
generated. Hence there will be $2^m$ divisions. The second argument is then 
not relevant. If type is 1 or -1 the second parameter says that the first 
function should obtain n arguments. The remaining arguments go to the 
second function. If type is 2 or -2 the second function should obtain n 
arguments. Example:
% THIS EXAMPLE IS PART OF THE TESTSUITE. CHANGES HERE SHOULD BE APPLIED THERE AS
% WELL!
\begin{verbatim}
    Symbols x1,...,x4;
    CFunctions f,f1,f2;
    Local F = f(x1,...,x4);
    id  f(?a) = distrib_(-1,2,f1,f2,?a);
    Print +s;
    .end

   F =
       + f1(x1,x2)*f2(x3,x4)
       - f1(x1,x3)*f2(x2,x4)
       + f1(x1,x4)*f2(x2,x3)
       + f1(x2,x3)*f2(x1,x4)
       - f1(x2,x4)*f2(x1,x3)
       + f1(x3,x4)*f2(x1,x2)
      ;
\end{verbatim}
When adjacent x-arguments are identical \FORM\ uses combinatorics to avoid 
generating more terms than necessary.

%--#] distrib_ : 
%--#[ div_ :

\section{div\_}\index{div\_}\index{function!div\_}
\label{fundiv}
\noindent \verb:div_(x1,x2): is replaced by the quotient of the arguments. 
The arguments can be any valid subexpressions, provided the whole function 
fits inside a term. When an argument is only an active expression or a 
\$-expression it is only expanded during the division. This way the 
contents of such expressions can exceed the maximum term size. One should 
however realize that in that case the operation takes place in allocated 
memory. This function replaces the experimental function 
polydiv\_\index{polydiv\_}\index{function!polydiv\_} that existed in 
version 3.

%--#] div_ : 
%--#[ dum_ :

\section{dum\_}\index{dum\_}\index{function!dum\_}
\label{fundum}
\noindent Special function for printing virtual\index{virtual bracket} 
brackets\index{bracket}. \verb:dum_(a+b): is printed as \verb:(a+b):: the 
name of this function is not printed!

%--#] dum_ : 
%--#[ dummy_ :

\section{dummy\_}\index{dummy\_}\index{function!dummy\_}
\label{fundummy}
\noindent For internal use only.

%--#] dummy_ : 
%--#[ dummyten_ :

\section{dummyten\_}\index{dummyten\_}\index{function!dummyten\_}
\label{fundummyten}
\noindent For internal use only.

%--#] dummyten_ : 
%--#[ e_ :

\section{e\_}\index{e\_}\index{function!e\_}
\label{fune}
\noindent The Levi-Civita\index{Levi-Civita tensor} 
tensor\index{tensor!Levi-Civita}. It is a totally 
antisymmetric\index{antisymmetric} tensor with well defined contraction 
rules (see \ref{substacontract}).

%--#] e_ : 
%--#[ exp_ :

\section{exp\_}\index{exp\_}\index{function!exp\_}
\label{funexp}
\noindent Internal function with two arguments. Represents 
argument1 to the power argument2. Of course it is printed in the standard 
power notation.

%--#] exp_ : 
%--#[ exteuclidean_ :

\section{exteuclidean\_}\index{exteuclidean\_}\index{function!exteuclidean\_}
\label{funexteuclidean}

\noindent This is a number function. It expects two positive integer 
arguments. It then computes the Greatest Common Divider of these arguments 
with the use of the extended Euclidean algoritm. The answer will be in the 
same function but now there will be four arguments as in:
\begin{verbatim}
    Symbols x1,x2,x3,x4;
    Local F = exteuclidean_(54,84);
    Print;
    .sort

   F =
      exteuclidean_(54,84,-3,2);

    id exteuclidean_(x1?,x2?,x3?,x4?) = x1*x3+x2*x4;
    Print;
    .end

   F =
      6;
\end{verbatim}
\noindent We can see that we obtain the GCD with the relation that is 
characteristic for the extended Euclidean algorithm. When the two arguments 
are relative prime, one obtains the so-called modinverses of these numbers:
\begin{verbatim}
    Symbols x1,x2,x3,x4,a,b;
    Local F = exteuclidean_(97,101);
    Print;
    .sort

   F =
      exteuclidean_(97,101,25,-24);

    id exteuclidean_(x1?,x2?,x3?,x4?) = x1*x3+x2*x4
        +a*mod2_(1/97,101)+b*mod2_(1/101,97);
    Print;
    .end

   F =
      1 - 24*b + 25*a;
\end{verbatim}
\noindent Here 25 is the inverse of 97 when we calculate modulus 101 and 
-24 is the inverse of 101 when we calculate modulus 97.

\noindent This function can be very handy when a calculation has been done 
modulus various prime numbers and one would like to know the result modulus 
the product of these numbers. This combination is done with the aid of the 
Chinese remainder theorem\index{Chinese remainder theorem}:
\begin{verbatim}
   #procedure ChineseRemainder(NAME,NAME1,NAME2,M1,M2,PAR)
   *
   *   Assumes that NAME1 is an expression mod $M1
   *   Assumes that NAME2 is an expression mod $M2
   *   Creates $ch1r and $ch2r with the property that
   *   the expression NAME = NAME1*$ch1r+NAME2*$ch2rn
   *   is the corresponding equation mod $M1*$M2
   *
   Modulus 0; * we need to switch off previous settings.
   #$ch1r = exteuclidean_($`M1',$`M2');
   #inside $ch1r;
       id exteuclidean_(xxx1?,xxx2?,xxx3?,xxx4?) = xxx2*xxx4;
   #endinside;
   #$ch2r = exteuclidean_($`M1',$`M2');
   #inside $ch2r;
       id exteuclidean_(xxx1?,xxx2?,xxx3?,xxx4?) = xxx1*xxx3;
   #endinside;
   #$MM12 = $`M1'*$`M2';
   Modulus,plusmin,`$MM12';
   Local `NAME' = `NAME1i'*$ch1r+`NAME2i'*$ch2r;
   .sort
   *
   #endprocedure
\end{verbatim}

%--#] exteuclidean_ : 
%--#[ extrasymbol_ :

\section{extrasymbol\_}\index{extrasymbol\_}\index{function!extrasymbol\_}
\label{funextrasymbol}

\noindent This function expects a single argument. This argument can be a 
number or an extra symbol(see \ref{extrasymbols}). In either case the 
function is replaced by the expression that the corresponding extra symbol 
stands for.

\noindent If there are more arguments or the argument does not represent a 
legal extra symbol, no substitution is made.

%--#] extrasymbol_ : 
%--#[ fac_ :

\section{fac\_}\index{fac\_}\index{function!fac\_}
\label{funfac}
\noindent The factorial\index{factorial} function. If it has a single nonzero 
integer argument n it is replaced by n! but if the result is bigger than 
the maximum allowable number an error will result.

%--#] fac_ : 
%--#[ factorin_ :
 
\section{factorin\_}\index{factorin\_}\index{function!factorin\_}
\label{funfactorin}
\noindent When the argument is a single \$-variable\index{\$-variable} or 
an expression\index{expression} the function is replaced by the common 
factor in the terms of that \verb:$:-variable or expression. This common 
factor consists in the first place of all symbolic objects that occur in 
all terms. In addition the numerical factor consists of the GCD\index{GCD} 
of all numerators and the LCM\index{LCM} of all denominators. Hence if the 
\verb:$:-variable or expression is divided by the result of factorin\_ all 
coefficients become integer.

%--#] factorin_ : 
%--#[ farg_ :

\section{farg\_}\index{farg\_}\index{function!farg\_}
\label{funfarg}
\noindent For internal use only.

%--#] farg_ : 
%--#[ firstbracket_ :

\section{firstbracket\_}\index{firstbracket\_}\index{function!firstbracket\_}
\label{funfirstbracket}
\noindent In the case that there is a single argument and this 
single argument is the name of an expression, this function is replaced by 
the part that is outside brackets in the first term of the expression. If 
there are no brackets the function is replaced by one.

%--#] firstbracket_ : 
%--#[ firstterm_ :

\section{firstterm\_}\index{firstterm\_}\index{function!firstterm\_}
\label{funfirstterm}

\noindent This function expects the name of an expression or a dollar 
variable for its (single) argument. It will return the first term in this 
expression or dollar variable. When it has to obtain the first term of an 
expression, FORM uses the expression in the representation in which it was 
stored at the end of the previous module. If the expression did not exist 
in the previous module, it will attempt to use the expression as defined 
and processed in the current expression. If the expression has only been 
defined in the current module and has not yet been processed (as is the 
case when referring to the first term in the current expression) the answer 
will be unspecified. This use is considered illegal, even though it does 
not generate an error message.

%--#] firstterm_ : 
%--#[ g5_ :

\section{g5\_}\index{g5\_}\index{function!g5\_}
\label{fungfive}
\noindent The $\gamma_5$ Dirac gamma matrix. We assume here that it 
anticommutes with the other Dirac\index{Dirac} gamma\index{gamma matrices} 
matrices. Anybody who does not like that should program private libraries 
(this should not be too difficult with the cycle symmetric functions 
(see \ref{substafunctions})). There should be a single index to indicate 
the spinline.

%--#] g5_ : 
%--#[ g6_ :

\section{g6\_}\index{g6\_}\index{function!g6\_}
\label{fungsix}
\noindent There should be a single index to indicate the spinline. 
As in Schoonschip\index{Schoonschip} we use $\gamma_6 = 1+\gamma_5$.

%--#] g6_ : 
%--#[ g7_ :

\section{g7\_}\index{g7\_}\index{function!g7\_}
\label{fungseven}
\noindent There should be a single index to indicate the spinline. 
As in Schoonschip\index{Schoonschip} we use $\gamma_7 = 1-\gamma_5$.

%--#] g7_ : 
%--#[ g_ :

\section{g\_}\index{g\_}\index{function!g\_}
\label{fung}
\noindent The Dirac\index{Dirac} gamma\index{gamma matrices} matrix. Its 
first argument should be an index (either symbolic or numeric). Then follow 
zero, one or more indices to indicate a string of gamma matrices that 
belong together. Gamma matrices with the same first index are considered to 
belong together, but as long as the indices are symbolic no assumptions are 
made about whether they go together or not. Hence no commutation or 
anticommutation properties are applied for different spin lines unless the 
spinline indices are both numeric.

%--#] g_ : 
%--#[ gcd_ :

\section{gcd\_}\index{gcd\_}\index{function!gcd\_}
\label{fungcd}
\noindent \verb:gcd_(x1,...,xn): is replaced by the greatest common divisor 
of the arguments. The arguments can be any valid subexpressions, provided 
the whole function fits inside a term. When an argument is only an active 
expression or a \$-expression it is only expanded during evaluation of the 
GCD. This way the contents of such expressions can exceed the maximum term 
size. One should however realize that in that case the operation takes 
place in allocated memory.
This function replaces the experimental function 
polygcd\_\index{polygcd\_}\index{function!polygcd\_} that existed in 
version 3.

%--#] gcd_ : 
%--#[ gi_ :

\section{gi\_}\index{gi\_}\index{function!gi\_}
\label{fungi}
\noindent The unit Dirac gamma matrix. Should have a single index 
to indicate its spin line. Its is identical to a regular gamma matrix with 
no Lorenz indices: \verb:gi_(n) = g_(n):

%--#] gi_ : 
%--#[ id_ :

\section{id\_}\index{id\_}\index{function!id\_}
\label{funid}
\noindent This function is a crossbreed between the 
replace\_\index{replace\_}\ref{funreplace} function and the id 
statement\index{substaidentify}\ref{substaidentify}. To become active it 
needs an even number of arguments. The odd numbered arguments can be 
anything of the types:
\begin{description}
\item[] a single symbol, possibly to an integer power.
\item[] a single dotproducts, possibly to an integer power.
\item[] a single function, possibly with any number and type of arguments.
\end{description}
When \FORM encounters an id\_ function the last step of normalizing a term 
is to replace the id function by a number substitutions in which the odd 
arguments are replaced by the following even arguments. These are not 
wildcard substitutions as in the replace\_ function, but substitutions as 
in regular id statements. The matching of the odd arguemnts is done in a 
single step as in an id-al construction\ref{substaalso}. Hence
\begin{verbatim}
    id_(x^2,y+z,y,u+v,x,z+u)
\end{verbatim}
effectively becomes
\begin{verbatim}
    id x^2 = y+z;
    al y = u+v;
    al x = z+u;
\end{verbatim}
\FORM treats multiple occurrences of the id\_ function one at a time. It 
takes the leftmost occurrence first, takes the patterns from the term, 
expands the right hand sides, tries to normalize the resulting terms and 
only then continues with the next id\_ function. For this reason the id\_ 
function is noncommuting.

%--#] id_ : 
%--#[ integer_ :

\section{integer\_}\index{integer\_}\index{function!integer\_}
\label{funinteger}
\noindent This is a rounding\index{rounding} function. It should have 
either one or two arguments. If there is a single argument and it is 
numeric, it will be rounded down to become an integer. If there are two 
arguments of which the first is numeric and the second is either 1, 0 or 
-1, the result will be the rounded value of the first argument. If the 
second argument is 1, the rounding will be down, when it is -1, the 
rounding will be up and when it is zero the rounding will be towards zero. 
In all other cases nothing is done.

%--#] integer_ : 
%--#[ inverse_ :

\section{inverse\_}\index{inverse\_}\index{function!inverse\_}
\label{funinverse}
\noindent \verb:inverse_(x1,x2): expects two arguments which are 
polynomials in the same single variable. The return expression $x_3$ 
has the property that $x_1 x_3$ divided by $x_2$ has remainder 1. Or in 
other words: $x_3$ is the inverse of $x_1$ modulus $x_2$.
The arguments can be any valid subexpressions, provided the whole function 
fits inside a term. When an argument is an active expression or a 
\$-expression it is only expanded during the division. This way the 
contents of such expressions can exceed the maximum term size. One should 
however realize that in that case the operation takes place in allocated 
memory. 

%--#] inverse_ : 
%--#[ invfac_ :

\section{invfac\_}\index{invfac\_}\index{function!invfac\_}
\label{funinvfac}
\noindent One divided by the factorial\index{factorial} function. If it has 
a single nonzero integer argument n, it is replaced by 1/n!, but if this 
results in a number bigger than the maximum allowable number an error will 
result.

%--#] invfac_ : 
%--#[ makerational_ :

\section{makerational\_}\index{makerational\_}\index{function!makerational\_}
\label{funmakerational}

\noindent This function takes two arguments. Both are integers. We assume 
calculus modulus the second argument. The function is then replaced by a 
fraction of which both elements are less than the square root of the second 
argument and that, in calculus modulus this second number would give the 
same result as the first number modulus the second number. Example:
\begin{verbatim}
    #$m = prime_(1);
    #write <> "The prime number is %$",$m
The prime number is 2147483587
    L	F = MakeRational_(12345678,$m);
    Print;
    .sort

   F =
      9719/38790;

    Modulus `$m';
    Print;
    .end

   F =
      12345678;
\end{verbatim}
\noindent This function can be used to reconstruct fractions when calculus 
has been done modulus one or more prime numbers.

%--#] makerational_ : 
%--#[ match_ :

\section{match\_}\index{match\_}\index{function!match\_}
\label{funmatch}
\noindent Currently not active. Replaced automatically by 1.

%--#] match_ : 
%--#[ max_ :

\section{max\_}\index{max\_}\index{function!max\_}
\label{funmax}
\noindent If all its arguments are numeric, this function returns 
the maximum value of these arguments.

%--#] max_ : 
%--#[ maxpowerof_ :

\section{maxpowerof\_}\index{maxpowerof\_}\index{function!maxpowerof\_}
\label{funmaxpowerof}
\noindent If this function has a single argument that is a symbol, it 
returns the maximum power restriction of this symbol. If none was given it 
will be the installation dependent value MAXPOWER which is 10000 on 
32\index{32 bits} bit machines and 500000000 on 64\index{64 bits} bit 
machines.

%--#] maxpowerof_ : 
%--#[ min_ :

\section{min\_}\index{min\_}\index{function!min\_}
\label{funmin}
\noindent If all its arguments are numeric, this function returns 
the minimum value of these arguments.

%--#] min_ : 
%--#[ minpowerof_ :

\section{minpowerof\_}\index{minpowerof\_}\index{function!minpowerof\_}
\label{funminpowerof}
\noindent If this function has a single argument that is a symbol, it 
returns the minimum power restriction of this symbol. If none was given it 
will be the installation dependent value -MAXPOWER which is -10000 on 32 bit 
machines.

%--#] minpowerof_ : 
%--#[ mod_ :

\section{mod\_}\index{mod\_}\index{function!mod\_}
\label{funmod}
\noindent If there are two integer arguments and the second 
argument is a positive short integer (less than $2^{15}$ on 32 bit 
computers and less than $2^{31}$ on 64 bit computers) the return value is 
the first argument modulus the second. Note that if the second argument is 
not a prime number and the first argument contains a denominator, division 
by zero could occur. It is up to the user to avoid such cases. See also the 
mod2\_ function~\ref{funmod2} and the rem\_ function~\ref{funrem}.

The function has one peculiarity: when the second argument is one, the 
function is left untouched.

%--#] mod_ : 
%--#[ mod2_ :

\section{mod2\_}\index{mod2\_}\index{function!mod2\_}
\label{funmod2}
\noindent This gives basically the same action as the mod\_ function (see 
\ref{funmod}), but the answer will be in the range $-[(p-1)/2]$ to 
$+[(p+1)/2]$.

%--#] mod2_ : 
%--#[ nargs_ :

\section{nargs\_}\index{nargs\_}\index{function!nargs\_}
\label{funnargs}
\noindent Is replaced by an integer indicating the number of 
arguments that the function has.

%--#] nargs_ : 
%--#[ nterms_ :

\section{nterms\_}\index{nterms\_}\index{function!nterms\_}
\label{funnterms}
\noindent If this function has only one argument it is replaced by 
the number of terms inside this argument.

%--#] nterms_ : 
%--#[ numfactors_ :

\section{numfactors\_}\index{numfactors\_}\index{function!numfactors\_}
\label{funnumfactors}

\noindent This function returns the number of factors in a factorized 
expression (see the chapter on polynomials~\ref{polynomials}) or dollar 
variable~\ref{dollars}. It expects a single argument which should be the 
name of an expression or a dollar variable. If the expression or dollar 
variable has not been factorized, the function returns zero.

%--#] numfactors_ : 
%--#[ pattern_ :

\section{pattern\_}\index{pattern\_}\index{function!pattern\_}
\label{funpattern}
\noindent Currently not active. Replaced automatically by 1.

%--#] pattern_ : 
%--#[ poly_ :

\section{poly\_}\index{poly\_}\index{function!poly\_}
\label{funpoly}
\noindent This was an experimental function in version 3. It was for 
internal use with a whole category of other experimental functions of which 
the functionality has been replaced by better working functions that are 
more general. This category included the functions 
polyadd\_\index{polyadd\_}\index{function!polyadd\_},
polydiv\_\index{polydiv\_}\index{function!polydiv\_},
polygcd\_\index{polygcd\_}\index{function!polygcd\_},
polyintfac\_\index{polyintfac\_}\index{function!polyintfac\_},
polymul\_\index{polymul\_}\index{function!polymul\_},
polynorm\_\index{polynorm\_}\index{function!polynorm\_},
polyrem\_\index{polyrem\_}\index{function!polyrem\_} and
polysub\_\index{polysub\_}\index{function!polysub\_}.
See also the chapter on polynomials\ref{polynomials} and the functions 
gcd\_\ref{fungcd}, div\_\ref{fundiv} and rem\_\ref{funrem}.

%--#] poly_ : 
%--#[ prime_ :

\section{prime\_}\index{prime\_}\index{function!prime\_}
\label{funprime}

\noindent For a number of internal operations FORM needs prime numbers that 
are neither very large nor very small. Hence it generates, when needed 
prime numbers that still fit inside a single FORM word, but are maximal 
within that limitation. Hence for a 64-bits computer in which the largest 
positive `small' integer in FORM is $2^{31}-1$, it works its way down from 
there. Once it has determined that a number is prime it stores it in a 
list. The function prime\_ gives access to this list. The single argument 
n (n a positive integer) makes that \verb:prime_(n): will be replaced by 
the n-th member of the list. There is a limitation to the size of the list 
which is implementation dependent. The number will anyway never be smaller 
than the maximum power that is allowed for symbols. Example:
\begin{verbatim}
    Symbols x1,x2,x3,x4;
    ON highfirst;
    Local F = x1*prime_(1)+x2*prime_(2)
             +x3*prime_(3)+x4*prime_(4);
    Print;
    .end

   F =
      2147483587*x1 + 2147483579*x2 + 2147483563*x3 + 2147483549*x4;
\end{verbatim}
This function is useful when calculations generate very large intermediate 
coefficients, but in the end the answer is relatively simple again. In that 
case one can do the calculation modulus one or more prime numbers. If more 
prime numbers are used the Chinese remainder theorem\index{Chinese 
remainder theorem}. can be used (see the exteuclidean\_ 
function~\ref{funexteuclidean} to combine the results and the 
makerational\_ function~\ref{funmakerational} can be used if fractions have 
to be reconstructed. An example of this kind of use is given in the simple 
Groebner basis procedure that is in the packages library in the FORM site.

%--#] prime_ : 
%--#[ random_ :

\section{random\_}\index{random\_}\index{function!random\_}
\label{funrandom}
\noindent A random number generator. When the function has a single 
positive integer argument, the function will return a pseudo random number 
in the range of one to that number inclusive. Hence one can imitate a die 
roll with the call random\_(6). The program uses a random number generator 
as described in vol 2 of the "Art of computer programming, vol2" by D. 
Knuth with the parameters set at 89,38 to give as long a cycle as possible. 
For very large numbers the program pastes several random numbers together. 
The generator can be initialized with the preprocessor 
\#setrandom\ref{presetrandom}\index{\#setrandom} instruction. When running 
with TFORM or ParFORM each worker runs an independent generator with its 
own seed. The seeds of the workers are derived from the seed of the master 
and the number of the worker in a non-trivial way. It should be noted 
however that with workers it may be impossible to reproduce previous runs 
as it is non-deterministic which term ends up in which worker.

%--#] random_ : 
%--#[ ranperm_ :

\section{ranperm\_}\index{ranperm\_}\index{function!ranperm\_}
\label{funranperm}
\noindent Generates a random permutation of the arguments, with exception 
of the first argument which should be the name of a function. This function 
will then have the permuted arguments as in:
\begin{verbatim}
    CFunction f;
    Symbols x1,...,x5;
    Local F = ranperm_(f,1,2,3,4,5,6)
             +ranperm_(f,x1,x2,x3+x1,x4,x5);
    Print +s;
    .end

   F =
       + f(x5,x1,x3 + x1,x4,x2)
       + f(3,1,6,2,4,5)
      ;
\end{verbatim}
The permutation is generated with the same random number generator that is 
used by the function 
random\_~\ref{funrandom}\index{random}\index{function!random\_} and hence 
is susceptible to the same initialization procedure that can be executed 
with the \#setrandom~\ref{presetrandom}\index{setrandom} instruction.

%--#] ranperm_ : 
%--#[ rem_ :

\section{rem\_}\index{rem\_}\index{function!rem\_}
\label{funrem}
\noindent \verb:rem_(x1,x2): is replaced by the remainder of the division 
of $x_1$ by $x_2$. The arguments can be any valid subexpressions, provided 
the whole function fits inside a term. When an argument is only an active 
expression or a \$-expression it is only expanded during the division. This 
way the contents of such expressions can exceed the maximum term size. One 
should however realize that in that case the operation takes place in 
allocated memory.
This function replaces the experimental function 
polyrem\_\index{polyrem\_}\index{function!polyrem\_} that existed in 
version 3.

%--#] rem_ : 
%--#[ replace_ :

\section{replace\_}\index{replace\_}\index{function!replace\_}
\label{funreplace}
\noindent This function defines a rather general purpose 
replacement\index{replacement} mechanism. It should have pairs of 
arguments. Each pair consists of a single symbol, index, vector or 
function, followed by what this object should be replaced by in the entire 
term. Functions can only be replaced by functions, indices only by indices. 
A vector can be replaced by a single vector or by a vector like expression. 
A symbol can be replaced by a single symbol, a numerical expression or a 
complete subexpression that is not index like or vector like. This 
mechanism is sometimes needed to make replacements in ways that are very 
hard with the id\index{id} statements because those do not make 
replacements automatically inside function arguments (see 
\ref{substaidnew}). It also allows to exchange two variables as the 
replacements are executed simultaneously by the wildcard substitution 
mechanism.
\begin{verbatim}
    Multiply replace_(x,y,y,x);
\end{verbatim}
will exchange x and y. Because there is no definite order in which multiple 
replace\_ functions are treated, one should not use more than a single one 
at the same time inside a term. At times multiple replace\_ functions may 
lead to confusion inside \FORM.

%--#] replace_ : 
%--#[ reverse_ :

\section{reverse\_}\index{reverse\_}\index{function!reverse\_}
\label{funreverse}
\noindent Can only occur as an argument of a function. Is replaced 
by the reversed string of its own arguments.

%--#] reverse_ : 
%--#[ root_ :

\section{root\_}\index{root\_}\index{function!root\_}
\label{funroot}
\noindent If we have \verb:root_(n,x): and \verb:n: is a positive 
integer and \verb:x: is a rational number and \verb:y: is a rational number 
with $y^n = x$ (no imaginary numbers are considered and negative numbers 
are avoided if possible. Only one root is given) then \verb:root_(n,x): is 
replaced by \verb:y:. This function was originally intended for internal 
use. Do not hold it against the author that \verb:root_(2,1): is replaced 
by \verb:1:. In the case that it is needed the user should manipulate the 
sign or the complexity properties externally.

%--#] root_ : 
%--#[ setfun_ :

\section{setfun\_}\index{setfun\_}\index{function!setfun\_}
\label{funsetfun}
\noindent Currently not active.

%--#] setfun_ : 
%--#[ sig_ :

\section{sig\_}\index{sig\_}\index{function!sig\_}
\label{funsig}
\noindent Is replaced by the sign of the (numerical) argument, i.e. by -1 
if there is a single negative argument and by +1 if there is a single 
numerical argument that is greater or equal to zero.

%--#] sig_ : 
%--#[ sign_ :

\section{sign\_}\index{sign\_}\index{function!sign\_}
\label{funsign}
\noindent \verb:sign_(n): is replaced by \verb:(-1)^n: if n is an 
integer.

%--#] sign_ : 
%--#[ sum_ :

\section{sum\_}\index{sum\_}\index{function!sum\_}
\label{funsum}
\noindent General purpose sum\index{sum} function. The first argument should 
be the summation parameter (a symbol). The second argument is the starting 
point of summation, the third argument the `upper' limit and a potential 
fourth argument the increment. These numbers should all be integers. 
Summation stops when the summation parameter obtains a value that has 
passed the upper limit. The last argument is the summand, the object to be 
summed over. It can be any subexpression. If it contains the summation 
parameter, it will be replaced by its value for each generated term. 
Examples:
\begin{verbatim}
    sum_(j,1,4,sign_(j)*x^j/j)
    sum_(i,1,9,2,sign_((i-1)/2)*x^i*invfac_(i))
\end{verbatim}

%--#] sum_ : 
%--#[ sump_ :

\section{sump\_}\index{sump\_}\index{function!sump\_}
\label{funsump}
\noindent Special sum function. Its arguments are like for the 
sum\_ function, but each new term is the product of the previously 
generated term with the last argument in which the current value of the 
summation parameter has been substituted. The first term is always one. 
Example:
% THIS EXAMPLE IS PART OF THE TESTSUITE. CHANGES HERE SHOULD BE APPLIED THERE AS
% WELL!
\begin{verbatim}
    Symbol i,x;
    Local F = sump_(i,0,5,x/i);
    Print;
    .end

   F =
      1 + x + 1/2*x^2 + 1/6*x^3 + 1/24*x^4 + 1/120*x^5;
\end{verbatim}
This function is a leftover from the Schoonschip\index{Schoonschip} days. 
The ordinary sum\_ function is much more readable.

%--#] sump_ : 
%--#[ table_ :

\section{table\_}\index{table\_}\index{function!table\_}
\label{funtable}
\noindent For action the arguments should be the name of a table and then 
either the name of a function or one symbol for each dimension of the 
table. In the case of the list of symbols the return value will be a 
monomial in the given symbols in which the powers of the symbols correspond 
to the table indices of the defined table elements with the coefficients 
the table contents corresponding to those indices. In the case of a 
function name the return value will be a sum over terms in which the table 
elements are indicated by arguments in the given function while these 
functions are then multiplied by the corresponding table elements. This is 
one way to put a complete table inside an expression and store it (with the 
save statement of \ref{substasave}) in a binary way for a future run in 
which the table can be filled again with the 
fillexpression\index{fillexpression} (see \ref{substafillexpression}) 
statement. Note that for obvious reasons one should avoid using symbols or 
functions that also occur inside the table definitions.

%--#] table_ : 
%--#[ tbl_ :

\section{tbl\_}\index{tbl\_}\index{function!tbl\_}
\label{funtbl}
\noindent This function is the `table stub function' as used by the 
tablebase\index{tablebase} construction. This is explained in chapter 
\ref{tablebase}. It is mainly for internal use, but it could occur in the 
output.

%--#] tbl_ : 
%--#[ term_ :

\section{term\_}\index{term\_}\index{function!term\_}
\label{funterm}
\noindent This function has no arguments. It is replaced by the current 
term. It can be used to load the current term into a dollar variable as in
\begin{verbatim}
   $x = term_;
\end{verbatim}

%--#] term_ : 
%--#[ termsin_ :

\section{termsin\_}\index{termsin\_}\index{function!termsin\_}
\label{funtermsin}
\noindent If there is a single argument and this argument is the name of an 
active (or previously active during the current job) expression, the 
function is replaced by the number\index{number of terms} of terms in this 
expression. Stored expressions that were entered via a load statement (see 
\ref{substaload}) are excluded from this because for them \FORM\ would have 
to actually count the terms.

%--#] termsin_ : 
%--#[ termsinbracket_ :
 
\section{termsinbracket\_}\index{termsinbracket\_}\index{function!termsinbracket\_}
\label{funtermsinbracket}
\noindent If there is no argument, or the single argument is zero, the 
function is replaced by the number of terms in the current 
bracket\index{bracket}, provided the expression has been bracketed at its 
last sort and a keep brackets statement (see \ref{substakeep}) has been 
used. Note that the terms have to be counted. Hence this is a relatively 
expensive command. More options will be implemented in the future.

%--#] termsinbracket_ : 
%--#[ theta_ :

\section{theta\_}\index{theta\_}\index{function!theta\_}
\label{funtheta}
\noindent If there is a single numerical argument x the function is 
replaced by one if $x \ge 0$ and by zero if $x < 0$. If there are two 
numerical arguments $x_1$ and $x_2$ the function is replaced by one if $x_1 
= x_2$ or if the arguments are in natural order (if theta\_ would be a 
symmetric function there would be no reason to exchange the arguments) and 
by zero if the arguments are not in natural order (they would be exchanged 
in a symmetric function). In all other cases nothing is done.

%--#] theta_ : 
%--#[ thetap_ :

\section{thetap\_}\index{thetap\_}\index{function!thetap\_}
\label{funthetap}
\noindent If there is a single numerical argument x the function is 
replaced by one if $x > 0$ and by zero if $x \le 0$. If there are two 
numerical arguments $x_1$ and $x_2$ the function is replaced by zero if $x_1 
= x_2$ or if the arguments are not in natural order. If the arguments are 
in natural order the function is replaced by one. In all other cases 
nothing is done.

%--#] thetap_ : 

%--#[ Reserved names :

\section{Extra reserved names}

\noindent In addition there are some names that have been reserved for 
future use. At the moment these functions do not do very much. It is hoped 
that in the future some simplifications of the arguments can be 
implemented. These functions are:

\leftvitem{3cm}{sqrt\_}\index{sqrt\_}\index{function!sqrt\_}
\rightvitem{13cm}{The regular square root.}

\leftvitem{3cm}{ln\_}\index{ln\_}\index{function!ln\_}
\rightvitem{13cm}{The natural logarithm.}

\leftvitem{3cm}{sin\_}\index{sin\_}\index{function!sin\_}
\rightvitem{13cm}{The sine function.}

\leftvitem{3cm}{cos\_}\index{cos\_}\index{function!cos\_}
\rightvitem{13cm}{The cosine function.}

\leftvitem{3cm}{tan\_}\index{tan\_}\index{function!tan\_}
\rightvitem{13cm}{The tangent function.}

\leftvitem{3cm}{asin\_}\index{asin\_}\index{function!asin\_}
\rightvitem{13cm}{The inverse of the sine function.}

\leftvitem{3cm}{acos\_}\index{acos\_}\index{function!acos\_}
\rightvitem{13cm}{The inverse of the cosine function.}

\leftvitem{3cm}{atan\_}\index{atan\_}\index{function!atan\_}
\rightvitem{13cm}{The inverse of the tangent function.}

\leftvitem{3cm}{atan2\_}\index{atan2\_}\index{function!atan2\_}
\rightvitem{13cm}{Another inverse of the tangent function.}

\leftvitem{3cm}{sinh\_}\index{sinh\_}\index{function!sinh\_}
\rightvitem{13cm}{The hyperbolic sine function.}

\leftvitem{3cm}{cosh\_}\index{cosh\_}\index{function!cosh\_}
\rightvitem{13cm}{The hyperbolic cosine function.}

\leftvitem{3cm}{tanh\_}\index{tanh\_}\index{function!tanh\_}
\rightvitem{13cm}{The hyperbolic tangent function.}

\leftvitem{3cm}{asinh\_}\index{asinh\_}\index{function!asinh\_}
\rightvitem{13cm}{The inverse of the hyperbolic sine function.}

\leftvitem{3cm}{acosh\_}\index{acosh\_}\index{function!acosh\_}
\rightvitem{13cm}{The inverse of the hyperbolic cosine function.}

\leftvitem{3cm}{atanh\_}\index{atanh\_}\index{function!atanh\_}
\rightvitem{13cm}{The inverse of the hyperbolic tangent function.}

\leftvitem{3cm}{li2\_}\index{li2\_}\index{function!li2\_}
\rightvitem{13cm}{The dilogarithm function.}

\leftvitem{3cm}{lin\_}\index{lin\_}\index{function!lin\_}
\rightvitem{13cm}{The polylogarithm function.}

\noindent The user is allowed to use these functions, but it could be that 
in the future they will develop a nontrivial behaviour. Hence caution is 
required.

%--#] Reserved names : 

