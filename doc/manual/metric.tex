
\chapter{A few notes on the use of a metric}
\label{metric}

\noindent When FORM was designed, it was decided to make its syntax more or 
less independent of a choice of the metric\index{metric}. Hence statements 
and facilities that programs like Schoonschip\index{Schoonschip} or 
REDUCE\index{REDUCE} provide but which depend on the choice of a metric 
have been left out. Instead there are facilities to implement any choice of 
the metric, when the need really arises. When one makes a proper study of 
it, it turns out that one usually has to do very little or nothing. \hfill 
\vspace{2mm}

\noindent First one should realize that FORM does not know any specific 
metric by itself. Dotproducts are just objects of manipulation. It is 
assumed that when a common index of two vectors is contracted, this works 
out properly into a scalar object. This means that if one has a metric with 
upper and lower indices\index{indices!upper}\index{indices!lower}, one 
index is supposed to be an upper index and the other is supposed to be a 
lower index. If the user does not like this, it is his/her responsibility 
to force the system into a different action. This is reflected in the fact 
that FORM does not have an internal metric tensor\index{tensor!metric} 
$\eta_{\mu\nu}$. It has only a Kronecker\index{Kronecker} 
delta\index{delta!Kronecker} $\delta_{\mu\nu} =$ \verb:d_(mu,nu): with 
\verb:p(mu)*d_(mu,nu)*q(nu): $\rightarrow$ \verb:p.q: when mu and nu are 
summable indices\index{indices!summable}. \hfill \vspace{2mm}

\noindent The dependency of a metric usually enters with statements like 
$p^2 = \pm m^2$, which the user should provide anyway, because FORM does 
not have such knowledge. Connected to this is the choice of a 
propagator\index{propagator} as either $\gamma_\mu p_\mu + m$ or 
$\gamma_\mu p_\mu + i\ m$. This is also something the user should provide. 
The only objects that FORM recognizes and that could be considered as 
metric-dependent are the gamma matrices\index{matrices!gamma} and the 
Levi-Civita\index{Levi-Civita} tensor\index{tensor!Levi-Civita} 
\verb:e_:. Because the trace of a $\gamma_5$ involves a Levi-Civita tensor, 
the two are intimately connected. The anticommutator of two gamma matrices 
is defined with the Kronecker delta. Amazingly enough that works out well, 
provided that, if such Kronecker delta's survive in the output, they are 
interpreted as a metric tensor. This should be done with great care, 
because at such a point one does something that depends of the metric; one 
may have to select whether the indices are upper or lower indices. One 
should check carefully that the way the output is interpreted leads indeed 
to the results that are expected. This is anyway coupled to how one should 
interpret the input, because in such a case one would also have an input 
with `open' indices and give them a proper interpretation. The rule is that 
generally one does not have to do anything. The upper indices in the input 
will be upper indices in the output and the same for lower indices. \hfill 
\vspace{2mm}

\noindent The contraction\index{contraction} of two Levi-Civita tensors will give products of 
Kronecker delta's. This means that formally there could be an error of the 
sign of the determinant of the metric tensor, if one would like the 
Kronecker delta to play the role of a metric tensor. Hence it is best to 
try to avoid such a situation. \hfill \vspace{2mm}

\noindent In FORM the $\gamma_5$ is an object that anticommutes with 
the $\gamma_\mu$ and has $\gamma_5\gamma_5 = 1$. Its properties 
in the trace are 
\begin{eqnarray}
	Tr[\gamma_5\gamma_{m_1}\gamma_{m_2}\gamma_{m_3}\gamma_{m_4}] & = &
		\epsilon_{\mu_1\mu_2\mu_3\mu_4} \nonumber
\end{eqnarray}
This has a number of interesting consequences. The V-A and V+A currents are 
represented by $\gamma_7\gamma_\mu = (1-\gamma_5)\gamma_\mu$ and 
$\gamma_6\gamma_\mu = (1+\gamma_5)\gamma_\mu$ respectively. Under 
conjugation we have to replace $\gamma_5$ by $-\gamma_5$ as is not 
uncommon. \hfill \vspace{2mm}

\noindent There was a time that a conjugation\index{conjugation} operation 
was planned in FORM. As time progressed, it was realized that this would 
introduce problems with some of the internal objects. Hence some objects 
have the property that they are considered imaginary\index{imaginary}. In 
practise FORM does not do anything with this. Neither does it do anything 
with the declarations real\index{real}, complex\index{complex} and 
imaginary\index{imaginary}. If ever a way is found to implent a conjugation 
operator that will make everybody happy, it may still be built in. \hfill 
\vspace{2mm}

\noindent The above should give the user enough information to convert any 
specific metric to what is needed to make FORM do what is expected from it. 
Afterwards one can convert back, provided no metric\index{metric} specific 
operations are done. Such metric specific things are for instance needed in 
some types of approximations in which one substitutes objects by 
(vector)components halfway the calculation. In that case one cannot rely on 
that the conversions at the beginning and the end will be compensating each 
other. For this case FORM allows the user to define a private metric. All 
the tools exist to make this a success with the exception of a loss in 
speed of course. Let us have a look at the contraction of two Levi-Civita 
tensors in an arbitrary metric:
\begin{verbatim}
    Indices m1,m2,m3,n1,n2,n3,i1,i2,i3;
    Cfunction eta(symmetric),e(antisymmetric);
    Off Statistics;
    *
    *   We have our own Levi-Civita tensor e
    *
    Local F = e(m1,m2,m3)*e(m1,m2,m3);
    *
    *   We write the contraction as
    *
    id  e(m1?,m2?,m3?)*e(n1?,n2?,n3?) =
            e_(m1,m2,m3)*e_(i1,i2,i3)*
                eta(n1,i1)*eta(n2,i2)*eta(n3,i3);
    *
    *   Now we can use the internal workings of the contract:
    *
    Contract;
    Print +s;
    .sort

   F =
       + eta(i1,i1)*eta(i2,i2)*eta(i3,i3)
       - eta(i1,i1)*eta(i2,i3)^2
       - eta(i1,i2)^2*eta(i3,i3)
       + 2*eta(i1,i2)*eta(i1,i3)*eta(i2,i3)
       - eta(i1,i3)^2*eta(i2,i2)
      ;

    *
    *   For specifying a metric we need individual components:
    *
    Sum i1,1,2,3;
    Sum i2,1,2,3;
    Sum i3,1,2,3;
    Print +s;
    .sort

   F =
       + 6*eta(1,1)*eta(2,2)*eta(3,3)
       - 6*eta(1,1)*eta(2,3)^2
       - 6*eta(1,2)^2*eta(3,3)
       + 12*eta(1,2)*eta(1,3)*eta(2,3)
       - 6*eta(1,3)^2*eta(2,2)
      ;

    *
    *   And now we can provide the metric tensor
    *
    id  eta(1,1) = 1;
    id  eta(2,2) = 1;
    id  eta(3,3) = -1;
    id  eta(1,2) = 0;
    id  eta(1,3) = 0;
    id  eta(2,3) = 0;
    Print +s;
    .end

   F =
       - 6
      ;
\end{verbatim}
This is the ultimate in flexibility\index{flexibility} of course. It can 
also be worked out in a different way. In this case we try to change the 
behaviour of the Kronecker\index{Kronecker} delta\index{delta!Kronecker} a 
bit. This is dangerous\index{dangerous} and needs, in addition to a good 
understanding of what is happening, good testing to make sure that what the 
user wants is indeed what does happen. Here we use the 
FixIndex\index{fixindex} (\ref{substafixindex}) statement. This one assigns 
specific values to selected diagonal elements of the Kronecker delta. Of 
course it is the responsibility of the user to make sure that the 
calculation will indeed run into those elements. This is by no means 
automatic, because when FORM uses formal indices it never writes them out 
in components. Moreover, it would not be defined what would be the 
components connected to an index. The index could run over $0,1,2,3$ or 
over $1,2,3,4$, or maybe even over $5,7,9,11$. And what does an 
n-dimensional index run over? In the above example it is the sum 
(\ref{substasum}) statement that determines this. Hence this is fully under 
the control of the user. Therefore a proper way to deal with the above 
example would be
\begin{verbatim}
    Indices i1,i2,i3;
    FixIndex 1:1,2:1,3:-1;
    Off Statistics;
    *
    Local F = e_(i1,i2,i3)*e_(i1,i2,i3);
    Sum i1,1,2,3;
    Sum i2,1,2,3;
    Sum i3,1,2,3;
    Print +s;
    .sort

   F =
       + 6*e_(1,2,3)*e_(1,2,3)
      ;

    Contract;
    Print +s;
    .end

   F =
       - 6
      ;
\end{verbatim}
In the case that one would like to exchange the order of the summation and 
the contraction, while using the FixIndex mechanism, one needs to be more 
careful. In that case we have to prevent the indices from being summed over 
while they are indices of a Kronecker delta, because as long as the indices 
are symbolic, FORM will replace \verb:d_(i1,i1): by the dimension of 
\verb:i1:, and that is not what we want. Hence we have to declare the 
indices to be non-summable by giving them dimension zero:
\begin{verbatim}
    Indices i1=0,i2=0,i3=0;
    FixIndex 1:1,2:1,3:-1;
    Off Statistics;
    *
    Local F = e_(i1,i2,i3)*e_(i1,i2,i3);
    Contract;
    Print +s;
    .sort

   F =
       + d_(i1,i1)*d_(i2,i2)*d_(i3,i3)
       - d_(i1,i1)*d_(i2,i3)*d_(i2,i3)
       - d_(i1,i2)*d_(i1,i2)*d_(i3,i3)
       + 2*d_(i1,i2)*d_(i1,i3)*d_(i2,i3)
       - d_(i1,i3)*d_(i1,i3)*d_(i2,i2)
      ;

    Sum i1,1,2,3;
    Sum i2,1,2,3;
    Sum i3,1,2,3;
    Print +s;
    .end

   F =
       - 6
      ;
\end{verbatim}
As we can see, the automatic summation over the indices is not performed 
now and this gives us a chance to do the summation manually. After that the 
fixindex statement can have its effect. \hfill \vspace{2mm}

\noindent It should be clear from the above examples that it is usually 
much easier to manipulate the input in such a way that the terms with two 
Levi-Civita tensors have the negative sign from the beginning. This would 
give programs that are less complicated and much faster. \hfill \vspace{2mm}

\noindent Hence we are faced with the situation that in normal cases one 
does not do anything. If one wants to go beyond this and wants to interfere 
with the inner workings themselves by for instance inserting a factor $i$ 
in front of the $\gamma_5$ and emulating the upper and lower indices of a 
favorite metric, this leads from one problem to the next. Extreme care is 
needed. This is usually done by people who have first worked with other 
programs in which things don't work as naturally as in FORM. By the time 
one has really figured out how to deal with the metric and how to make use 
of the internal algorithms of FORM, one usually does not have to do very 
much again. \hfill \vspace{2mm}

\noindent As in the Zen\index{Zen} saying: \hfill \vspace{2mm}

\noindent To the beginning student mountains\index{mountains} are mountains 
and water\index{water} is water. To the advanced student\index{student} 
mountains stop being mountains and water stops being water. To the 
master\index{master} mountains are mountains again and water is water 
again. \hfill \vspace{2mm}

\noindent Of course the modern master also checks that what he expects the 
system to do, is indeed what the system does.
