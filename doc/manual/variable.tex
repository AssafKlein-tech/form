
\chapter{Variables}
\label{ch-variables}

The objects of symbolic manipulations are expressions\index{expression}. 
Expressions are built up from terms\index{terms} and terms are composed of 
variables. {\FORM} knows several types of variables, each 
of which has special rules assigned to it. The types of variables are 
symbols, vectors, indices, functions, sets, and expressions. In addition 
there are tensors and tables which are special functions, preprocessor 
variables\index{variables!preprocessor} (see chapter~\ref{preprocessor}),
and there are dollar variables\index{variables!dollar} (see 
chapter~\ref{dollars}). The expressions are used either in the definition 
of an expression or in the right hand side of an expression or a 
substitution. When an expression is used in the right hand side of another 
expression or a substitution, it will be replaced by its contents at the 
first opportunity. Therefore an expression will never occur as a variable 
in the output of other expressions and we will ignore their potential 
presence in the remainder of this chapter. Similarly preprocessor variables 
and  dollar variables will be replaced immediately when they are 
encountered.

The right hand side of an expression can consist of symbols, vectors, 
indices, functions and elements of a set. All these objects have to be 
declared before they can be used. The rules 
connected to each of these types of variables are described in the 
sections below.

\section{Names}

There are two types of names\index{names}. Regular 
names\index{names!definition} consist of alphabetic and numeric characters 
with the condition that the first character must be alphabetic. {\FORM} is 
case sensitive with respect to names. In addition there are {\bf formal 
names}. These names start with the character \verb:[: and end with a 
matching character \verb:]:. In between there can be any characters that 
are not intercepted by the preprocessor. This allows the use of variables 
like \verb:[x+a]:. Using formal names can improve the readability of 
programs very much, while at the same time giving the user the benefits of 
the greater speed. The use of denominators\index{denominators} that are 
composite (like \verb:1/(x+a):) is usually rather costly in time. Often 
\verb:1/[x+a]: is equally readable, while leading to the same results. Note 
however that the variable \verb:[x+a]: will have to be declared properly. 
On the other hand: {\FORM} may not have to know about x and a. These formal 
names can also be used for the names of expressions, but they are not valid 
for the names of dollar variables and the names of 
preprocessor variables\index{variables!preprocessor}.

Some names may contain special characters. All built in objects have for 
their last character an underscore\index{underscore} (\_). 
Dotproducts\index{dotproducts} (the scalar product of two vectors) consist 
of two vectors separated either by a period or by a dollar sign. The dollar 
sign is used by {\FORM}, when the output of the program has to be 
Fortran\index{fortran} compatible. The user can replace the dollar sign in 
the output by an arbitrary character by defining the variable 
"DotChar"\index{dotchar} in the setup\index{setup file} file. How this is 
done is explained in chapter~\ref{setup}. In the input the user may apply 
either the notation with the period or the notation with the dollar. It is 
however recommended to use the period\index{period} because in future 
versions the notation with the dollar may be dropped.  The above 
conventions avoid the possibility of conflicts with reserved names, 
allowing the user full freedom when choosing names. 

The dollar sign is also used as the first character in the name of dollar 
variables\index{variables!dollar}. The rest of the name should consist of 
alphanumeric characters of which the first should be alphabetic. The names 
of preprocessor variables\index{variables!preprocessor} should also consist 
of alphanumeric characters of which the first should be alphabetic. Also 
here the ones that are defined by the system have a trailing 
underscore\index{underscore} (\_) character.

With respect to the user defined names {\FORM} is case sensitive. This 
means that the variables a and A are different objects. With respect to 
system defined objects {\FORM} is case insensitive. Hence both d\_ and D\_ 
indicate the same Kronecker delta.

In many languages the use of the underscore\index{underscore} (\_) 
character is also permitted in the definition of user defined names. In 
{\FORM} this is NOT the case. Even though the earlier manuals `forbade' 
this specifically there was a bug in earlier versions that permitted it to 
some degree. And because people don't read manuals, there were those who 
used this character and even made it into a vital part of their naming 
conventions. This then broke when version 3 was introduced. It should be 
clear though that the underscore character is reserved for a completely 
different type of future use and hence nothing can be done about this. Just 
remember: it is never a good idea to use undocumented features without 
consulting with the development team first.

The complex conjugate\index{conjugate!complex} of a complex 
quantity is indicated by the character \verb:#: appended to the name of the 
variable. In the current version of {\FORM} not much is done with it. The 
latest approach is that it is seen as obsolete. If possible, please avoid 
using it.

The length of names\index{names!length} is not restricted in {\FORM}. There 
is one exception to this rule: names of expressions cannot be longer than 
16 characters. Of course in practise there are physical limits on the size 
of names, posed by the size of the memory of the computer being used.

\section{Symbols}

\label{sect-symbols}
Symbols\index{symbols} are plain objects that behave most like normal 
variables in hand manipulations. Many hand manipulations concern 
polynomial formulae of simple algebraic variables. {\FORM} assumes that 
symbols commute with all other objects and have a power connected to 
them. This power is limited to an installation dependent maximum and 
minimum. A power outside this range will lead to an error message. The 
user may override this built in restriction by one of private design that
is more restrictive. Any power that falls outside the user defined range 
leads to the removal of the term that contains the variable with this 
power. Such a power restriction can be defined for 
each symbol separately.

Symbols can also have complex conjugation\index{conjugation!complex} 
properties. A symbol can be declared to be real, imaginary or complex. This 
property is only relevant, when the complex conjugation operator is used. 
This operator has not been implemented and currently there are no plans to 
do so. 

The syntax of the statement that defines symbols is given by (see also 
\ref{substasymbols}): 
\begin{verbatim}
     S[ymbols]    name[#{R|I|C}][(min:max)];
\end{verbatim}
Each variable is declared by the presence of its name in a 
symbol-statement. If the \# symbol is appended, it should be followed by 
either the character C, I or R to indicate whether the variable is 
complex\index{complex}, imaginary\index{imaginary} or real\index{real}. The 
\#R is not really necessary, as the type `real' is the default. It is not 
relevant whether the C, I, R are in upper or in lower case. A power 
restriction\index{restriction!power} is indicated with a range between 
regular parentheses. If one of the two numbers is not present, the default 
value is taken. This default value is installation dependent, but it is at 
least -10000 and 10000 respectively. Each symbol-statement can define more 
than one variable. In that case the variables have to be separated either 
by comma's or by blanks. Example:
\begin{verbatim}
     S    x,y,z,a#c,b#c,c#c,r(-5:5),s(:20),t#i(6:9);
\end{verbatim}
In this statement x, y and z are normal real algebraic variables. The 
variables a, b and c are complex. This means that for each of these 
variables two entries are reserved in the property lists: one for the 
variable and one for its complex conjugate. The variable r has a power 
restriction: Any power outside the specified range will cause the term 
containing this power to be eliminated. This is particularly useful in 
power series expansions. The restrictions on s are such that there is 
no limitation on the minimum power of s --with the exception of the 
built in restrictions-- but a term with a power of s that is larger 
than 20 is eliminated. The variable t is imaginary. This means that 
under complex conjugation it changes sign. Its power restrictions are 
somewhat uncommon. Any power outside the range 6 to 9 is eliminated. 
There is however one exception: a term that does not contain t to any 
power ($t^0$) is not affected. 
% THIS EXAMPLE IS PART OF THE TESTSUITE. CHANGES HERE SHOULD BE APPLIED THERE AS
% WELL!
\begin{verbatim}
    s     x(:10),y;
    L     F=y^7;
    id    y=x+x^2;
    print;
    .end

Time =       0.01 sec    Generated terms =          4
                F        Terms in output =          4
                         Bytes used      =         54

   F =
      x^7 + 7*x^8 + 21*x^9 + 35*x^10;
\end{verbatim}
Note that all terms with a power greater than 10 do not even count 
as generated terms. They are intercepted immediately after the 
replacement, before any possible additional statements can be 
carried out.

There are several built in symbols\index{symbols!built in}. They are:

\noindent i\_\index{i\_}: it is defined by \verb:i_^2 = -1: and this 
property is used by {\FORM} to simplify terms. It is the only symbol that 
cannot be used as a dimension or a wildcard.

\noindent pi\_\index{pi\_}: a reserved variable which will eventually be 
used to indicate the variable $\pi$.

\noindent coeff\_\index{coeff\_}: this variable is automatically replaced 
by the coefficient of the current term.

\noindent num\_\index{num\_}: this variable is automatically replaced by 
the numerator of the coefficient of the current term.

\noindent den\_\index{den\_}: this variable is automatically replaced by 
the denominator of the coefficient of the current term.

\noindent extrasymbols\_\index{extrasymbols\_}: this symbol represents the 
number of extra symbols (see \ref{sect-extrasymbols}).

\section{Vectors}

\label{sect-vectors}
A vector\index{vectors} is an object with a single index\index{index}. This 
index represents a number that indicates which component of the vector is 
meant. Vectors have a dimension\index{dimension} connected to them which is 
the dimension of the vector space in which they are defined. In {\FORM} 
this dimension is by default set to 4. If the user likes to change this 
default, this can be done with the `Dimension'-statement. The use of this 
command affects the dimension of all vectors and the default dimension of 
indices. Its syntax is (see also \ref{substadimension}):
\begin{verbatim}
     Dimension number;
\end{verbatim}
or
\begin{verbatim}
     Dimension symbol;
\end{verbatim}
The number must be a number that fits inside a {\FORM} word which is 
an installation dependent size, but it will be at least 32767. 
The number must be positive or zero. Negative values are illegal. 
If a symbol is specified, it must have been declared before. Any symbol 
may be used with the exception of i\_\index{i\_}. 

The declaration of vectors (see \ref{substavectors}) is rather 
straightforward:
\begin{verbatim}
     V[ector] name [,MoreNames];
\end{verbatim}
The names of the vectors may be separated either by comma's or 
by blanks. Example: 
\begin{verbatim}
     V    p,q;
     I    mu,nu;
     L    F=p(mu)*q(nu);
\end{verbatim}

\section{Indices}

\label{sect-indices}
Indices\index{indices}\index{index} are objects that represent a number 
that is used as an integer argument for counting purposes. They are used 
mostly as the arguments of vectors or multidimensional arrays (or tensors). 
Their main property is that they have a dimension\index{dimension}. This 
dimension indicates what values the index can take. A four-dimensional 
index can usually take the values 1 to 4. A very important property of an 
index is found in the convention that it is assumed that an index that is 
used twice in the same term is summed over. This is called the 
Einstein\index{Einstein} summation\index{summation!Einstein} convention. 
Hence the term p(mu)$*$q(mu) is equivalent to the scalar product of the 
vectors p and q (which can also be written as p.q). 

There are of course also indices that should not be summed over. 
Such indices we call zero-dimensional. This is just a convention. 
To declare indices we use the statement (see also \ref{substaindex}): 
\begin{verbatim}
     Index name[={number|symbol}]
                   [,othername[={number|symbol}]];
\end{verbatim}
When the equals sign is used, this indicates the specification of a 
dimension. Indices that are not followed by an equals sign get the 
dimension that is currently the default dimension (see also 
\ref{substadimension})). The dimension can be either a number that is 
zero or positive (zero indicates that the summation convention does not 
apply for this index) or it can be any symbol with the exception of the 
symbol i\_. The symbol must have been declared before. 

The most important use of the dimension of an index is the built in rule 
that a Kronecker\index{Kronecker} delta\index{delta!Kronecker} with twice 
the same index is replaced by the dimension of this index, provided this 
index has a non-zero dimension. Therefore when mu is 4-dimensional, d\_(mu,
mu) will be replaced by 4 and when nu is n-dimensional, d\_(nu,nu) will be 
replaced by n. If rho is zero dimensional, the expression d\_(rho,rho) is 
left untouched.

In addition to the symbolic indices there is a number of fixed 
indices\index{indices!fixed} with a numeric\index{indices!numeric} value. 
The values of these indices runs from zero to an installation dependent 
number (usually 127). Users who like a different maximum value should 
consult chapter~\ref{setup} about the setup parameters. The numeric indices 
are all assumed to have dimension zero, hence no summation is applied to 
them. This means that they can be used for vector components. It is 
therefore perfectly legal to use:
\begin{verbatim}
     V    p,q,r;
     L    F=p(1)*q(1)*r(1)+p(2)*q(2)*r(2);
\end{verbatim}
When two numeric indices occur inside the same Kronecker delta, a value 
is substituted for this delta. Normally this value is one, when the two 
indices are identical and zero, when they are different. The value for 
the diagonal elements can be changed with the 
`FixIndex'-statement (see also \ref{substafixindex}): 
\begin{verbatim}
     Fi[xIndex] number:value [,number:value];
\end{verbatim}
This command assigns to d\_(number,number) the given value. 
This value must fit inside a single {\FORM} word. This means that this 
value can at least be in the range -32768 to +32767. For more 
details on the size of a {\FORM} word one should consult the 
installation manual. 

In the case of summable indices\index{indices!summable} the use of three 
times the same index in the same term would cause problems. {\FORM} will 
execute the contraction for the first pair it encounters, after which the 
third index is left. In the case of four or more indices the pairing for 
the contractions depends on the order in which the parts of the term are 
processed. Hence to the user the result may seem to be quasi random. 
Nothing can be done about this and the user should guard against such 
ambiguous notation\index{notation!ambiguous}.

There is a special version of the index declarations that is used for 
traces\index{traces} of gamma\index{gamma matrices} 
matrices\index{matrices!gamma} in n dimensions. If an index is declared with
\begin{verbatim}
    Symbols n,epsilon;
    Index m=n:epsilon;
\end{verbatim}
its dimension will be n and it is assumed that epsilon can be used for 
$(n-4)$ during the taking of the trace of a string of gamma matrices. It 
is also possible to use this notation in the dimension-statement. See 
also chapter~\ref{gammaalgebra} on the gamma matrices.

\section{Functions}

\label{sect-functions}
There are two classes of functions\index{functions}: {\bf commuting 
functions} which commute automatically with all other objects, and {\bf 
non-commuting functions} which do not necessarily commute with other 
non-commuting functions. An object is declared to be a 
commuting\index{commuting} function\index{function!commuting} with the 
`cfunction' command. Of this command the first two characters are 
mandatory, the others optional. An object is declared to be a 
non-commuting\index{non-commuting} function\index{function!non-commuting} 
with the `function' command. Here only the f is mandatory. The declaration 
of a function knows one option. This option concerns the complexity 
properties of the function. It is indicated by a \# following the name, 
after which one of the characters R, I, C specifies whether the function is 
real\index{real}, imaginary\index{imaginary} or complex\index{complex}. The 
declaration that a function is real is unnecessary as `real' is the default 
property. Example:
\begin{verbatim}
     CF   fa,fb,fc;
     F    ga,gb,gc#c;
\end{verbatim}
In this example the functions fa, fb, fc are commuting and the 
functions ga, gb and gc are not necessarily commuting. In addition the 
function gc is complex. More about functions and their conventions 
is explained in chapter~\ref{functions}.

Within the commutation classes there are several types of special 
functions. Currently these are tensors\index{tensors} and 
tables\index{tables}. The tables are described in section~\ref{substatable} 
and in chapter~\ref{tablebase}.

Tensors\index{tensors} are special functions. Their arguments can be 
indices and vectors only. When an argument is a vector, it is assumed that 
this vector has been put in this position as the result of an 
Einstein\index{Einstein} summation\index{summation!Einstein}, i.e., there 
used to be an index in this position, but the index was contracted with the 
index of the vector. Hence {\FORM} assumes that there is a linearity 
property with respect to such vectors. Tensors are declared with one of the 
following statements (see also pages~\ref{substatensors},
\ref{substantensors},\ref{substactensors}):
\begin{verbatim}
    T[ensors] t1;
    CT[ensors] t2;
    NT[ensors] t3;
\end{verbatim}
The type `ntensor' indicates a non-commuting tensor, while the other two 
types indicate commuting tensors. Note that the 'T' is a 
commuting tensor, while the 'F' indicates a non-commuting function. In 
addition to the above declarations one may add the same complexity 
properties that can be added for functions. This is currently not very 
useful though as there exists no complex conjugation 
operator yet. Internally a tensor is a function with special properties. 
Hence when function properties are discussed, usually these properties 
refer also to tensors, unless the type of the arguments would not allow 
the operations or arguments specified.

\section{Sets}

\label{sect-sets}
A set\index{sets} is a (non-empty) collection of variables that should 
all be of the same type. This type can be symbols, vectors, indices or 
functions. A set has a name which can be used to refer to 
it, and this name may not coincide with any of the other names in the 
program. A set is declared by giving its name, followed by a 
colon\index{colon}, after which the elements of the set are listed. The first 
element determines the type of all the elements of the set. All 
elements must have been declared as variables before the set-statement. 
There can be only one set per statement. Example (see also 
\ref{substaset}): 
\begin{verbatim}
     s    xa, xb, xc, xd, ya, x, y;
     i    mu, nu, rho;
     set exxes: xa, xb, xc, xd;
     set yyy: xc, xd, xb, ya;
     set indi:  mu, nu, rho, 1, 2, 3;
     set xandy: xa, ya;
\end{verbatim}
We see here that a single symbol (xa) can belong to more than one set. 
Also the fixed indices (1, 2 and 3) can be elements 
of a set of indices and the numbers that can be powers can also be 
members of a set of symbols (usually -9999 to + 9999). If this can cause 
confusion, {\FORM} will give a warning and interpret the set as a 
set of symbols.

In addition to the user defined sets there are some built in sets with a 
special meaning. These are:
\begin{description}
\item[int\_]\index{int\_} This is a set of symbols. It refers to all integer numbers 
	that fit inside a {\FORM} word.
\item[pos\_]\index{pos\_} This is a set of symbols. They are the positive integers that 
    fit inside a {\FORM} word.
\item[pos0\_]\index{pos0\_} A set of symbols. They are all non-negative integers that 
fit inside a {\FORM} word.
\item[neg\_]\index{neg\_}  A set of symbols. They are all negative integers that 
fit inside a {\FORM} word.
\item[neg0\_]\index{neg0\_}  A set of symbols. They are all non-positive integers that 
fit inside a {\FORM} word.
\item[symbol\_]\index{symbol\_} The set of all formal symbols. It excludes integers, 
    numbers and whole function arguments.
\item[fixed\_]\index{fixed\_} The set of all fixed indices.
\item[index\_]\index{index\_} The set of all indices.
\item[number\_]\index{number\_} The set of all rational numbers.
\item[even\_]\index{even\_} This is a set of symbols. It refers to all even integer numbers 
	that fit inside a {\FORM} word.
\item[odd\_]\index{odd\_} This is a set of symbols. It refers to all odd integer numbers 
	that fit inside a {\FORM} word.
\end{description}

Sets can be used during wildcarding\index{wildcarding}. When x is a symbol, 
the notation x? indicates `any symbol'. This is sometimes more than we 
want. In the case that we would like `any symbol that belongs to the set 
exxes' we would write x?exxes which is an unique notation as usually 
the question mark cannot be followed by a name. There should be no blank 
between the question mark and the name of the set. The object x?indi 
would result in a type mismatch error, if x is a symbol and indi a set of 
indices. 

This use of wildcards belonging to sets can be extended even more: 
The notation x?exxes?yyy means that x should belong to the set exxes, and
its replacement should be the corresponding element of set yyy. At first 
this notation looks unnecessarily complicated. The statement 
\begin{verbatim}
     id   x?exxes?yyy = x;
\end{verbatim}
should have the much simpler syntax
\begin{verbatim}
     id   exxes = yyy;
\end{verbatim}
This last notation cannot be maintained, when the patterns are more 
complicated, hence it has been omitted altogether.

When things become really complicated\index{complicated}, the sets can be 
used as kind of an array. They can be used with a fixed array index 
(running from 1 for the first element). When they have a symbolic argument 
(must be a symbol), they are either in the right hand side of an 
id-statement and the symbol must be replaced by a number by means of a 
wildcard substitution or in the left hand side and the symbol is 
automatically seen as a wildcard. The set must still follow the question 
mark of a wildcard. An example will clarify the above:
\begin{verbatim}
    s a1,a2,a3,b1,b2,b3,x,n;
    f g1,g2,g3,g;
    set aa:a1,a2,a3;
    set bb:b1,b2,b3;
    set gg:g1,g2,g3;
   
    id  g(x?aa[n]) = gg[n](bb[n]) + bb[2]*n;
\end{verbatim}
The n in the left hand side is automatically a symbol wildcard. x must 
match an element in aa and n takes its number. In the right hand side 
\verb:gg[n]: becomes an array element, when the n is substituted. The 
same holds for \verb:bb[n]:. The element \verb:bb[2]: is immediately 
replaced by b2, so there is rarely profit by using this, unless the 
preprocessor had something to do with the construction of this 
quantity. As should be clear from the above: the array elements are 
indicated with straight braces\index{braces}. 

Another use of sets is in the select  
option\index{option!select} of the id-statement. This is discussed in 
chapter~\ref{pattern} on pattern\index{pattern matching} matching. 

Neither the array properties of the sets nor the select option of the 
id-statement can be used in conjunction with the built in sets. These 
sets are not supposed to have a finite number of indices.

Apart from the above sets that were formally declared and used by name 
there is a second way to use sets. These sets are called {\bf implicitly 
declared sets\index{sets!implicitly declared}}. They are declared at the 
position that they are used and their use defines their contents. The 
elements of the set should be enclosed by a pair of curly 
brackets\index{brackets!curly} and the set is placed at the position where 
otherwise the name of the set would be used:
% THIS EXAMPLE IS PART OF THE TESTSUITE. CHANGES HERE SHOULD BE APPLIED THERE AS
% WELL!
\begin{verbatim}
    Symbols a1,a2,a3,b1,b2,b3,x,n;
    CFunctions g1,g2,g3,g;
    Local expr =
        g(a1)+g(a2)+g(a3)+g(x);
    id,g(x?{a1,a2,a3}[n]) = {g1,g2,g3}[n]({b1,b2,b3}[n]);
    print;
    .end

   expr =
      g1(b1) + g2(b2) + g3(b3) + g(x);
\end{verbatim}
Such a set exists internally only till the end of the module in which it 
is used. It can be used at all positions where named sets can be used. 
Hence they can also be used, when the array properties of sets are 
considered.

The preprocessor has to be able to distinguish these sets from strings for 
its calculator\index{calculator!preprocessor} (see 
chapter~\ref{preprocessor}). Usually this is no problem, because any 
regular name contains at least one character that is not accepted by this 
calculator. If the only elements in the set are numeric the 
comma\index{comma} will tell the preprocessor that it is a set and the 
calculator should not be used. This leaves the case of a set with a single 
numeric element. By placing a comma either before or after it the use of 
the calculator is vetoed. For the interpretation of the set this makes no 
difference.

When it is possible to demand an object to be inside a 
set\index{set!inside}, it should also be possible to demand that an object 
be outside a set\index{set!outside}. This is done with the `?!' operator 
instead of the `?' operator. The extra exclamation\index{exclamation} mark 
is like a `not' operator. It can be used only, when its use makes sense. 
Hence it cannot be used in conjunction with the array properties of sets 
and together with the select option of the id-statement. So its only use is 
in patterns of the type
\begin{verbatim}
    x?!setname
    x?!{a,b,c}
\end{verbatim}
as is done in
\begin{verbatim}
    id  x^n?!{,-1} = x^(n+1)/(n+1);
\end{verbatim}
There is a variation of the second type that is not possible with named 
sets\index{sets!named}:
\begin{verbatim}
    Symbols a,b,x,y,z;
    CFunction f;
    
    id  f(x?!{a,y?,z?})*f(y?!{b,x?,z?})*f(z?!{x?,y?})
            =  .........
\end{verbatim}
In this complicated pattern the z is easiest: It is not allowed to be equal 
to the objects that will be substituted for the wildcards x and y. The 
symbol x cannot be equal to the wildcards y and z, but in addition it 
should not be equal to a. A similar condition holds for y. One could argue 
that at least one of these conditions is superfluous from the strictly 
logical viewpoint. It depends however on the order of the declarations in 
how {\FORM} runs through the pattern, so it would require some trying to 
see which `not' specifications are superfluous. If for instance the first 
function is matched first, there is still no assignment for z. This means 
that the z? in the set cannot be used yet and hence it places no 
restrictions on x. Therefore it is the x? in the last function that causes 
x and z to be different. If on the other hand the last function would be 
matched first, we need the z? in the set of the first function. From the 
strict logical viewpoint, {\FORM} could go back over the pattern and still 
make the appropriate rejections, but this would cost too much extra time. 
As one can see, it is safer to specify both.

\section{The autodeclare conventions}

As we have seen above, all variables that are introduced by the user have 
to be declared. As such {\FORM} is a strong\index{strong typing} typing 
language. This isn't always handy. Hence it is possible to introduce some 
rules about the automatic declaration of classes of variables. This is done 
with the AutoDeclare\index{autodeclare} statement (see also 
\ref{substaautodeclare}). If we use the statements
\begin{verbatim}
     AutoDeclare Symbol x,tt;
     AutoDeclare CFunction f,t;
\end{verbatim}
any object encountered by the compiler of which the name starts with the 
character x will automatically be declared as a symbol. Also objects of 
which the name starts with the characters tt will be declared as symbols. 
Objects of which the name starts with the characters f or t, but not with 
the string tt, and that have not yet been declared will be declared 
automatically as commuting functions. As one can see, in the case of 
potential conflicts\index{conflicts} (like with t and tt) the more 
restrictive one takes precedence. This is independent of the order of the 
AutoDeclare statements. One disadvantage of the use of the AutoDeclare 
statement is that one looses a certain amount of control over the order of 
declaration of the variables, as now they will be declared in the order in 
which they occur in the statements. The order of the declaration determines 
the ordering of the objects in the output.

\section{Name lists}

\label{sect-namelists}
Sometimes it is necessary to see how {\FORM} has interpreted a set
of declarations. It can also be that declarations were made in an unlisted 
include file and that the user wants to know what variables have been 
defined. The lists\index{lists} of active variables\index{variables!lists} 
can be printed with the statement
\begin{verbatim}
     On names;
\end{verbatim}
This statement sets a flag that causes the listing of all name tables and 
default properties that are active at the moment that the compiler has 
finished compiling the current module and all modules after. The printing 
is just before the algebra processor takes over for the execution of the 
module -- assuming that no error condition exists. If the `On names' is 
specified in a module that ends with a .global-instruction, the name lists 
will be printed at the end of each module, as printing the name lists will 
then be the default option. If one likes to switch this flag off, this can 
be done with the statement
\begin{verbatim}
     Off names;
\end{verbatim}
which prohibits the printing of the name lists in the current module and 
all modules following.

\section{Dummy indices}

\label{sect-dummies}
Sometimes indices\index{indices!dummy} are to be summed over but due to the 
evaluation procedures some terms contain the index mu and other terms 
contain the index nu. There is a command to sum over indices in such a way 
that {\FORM} recognizes that the exact name of the index is irrelevant. 
This is the `sum'-statement (see also \ref{substasum}):%
% THIS EXAMPLE IS PART OF THE TESTSUITE. CHANGES HERE SHOULD BE APPLIED THERE AS
% WELL!
\begin{verbatim}
   i  mu,nu;
   f  f1,f2;
   L  F=f1(mu)*f2(mu)+f1(nu)*f2(nu);
   sum  mu;
   sum  nu;
   print;
   .end
\end{verbatim}
At first the expression contains two terms. After the summations {\FORM} 
recognizes the terms as identical. In the output we see the term: 
\begin{verbatim}
   2*f1(N1_?)*f2(N1_?)
\end{verbatim}
The \verb:N1_?: are dummy indices. 
The dimension of these dummy indices is the current 
default dimension\index{dimension!default} as set with the last 
dimension-statement. This may look like it is a restriction, but in 
practice it is possible to declare the default dimension to have one 
value in one module, take some sums, and do some more operations, and 
then give the default dimension another value in the next module. It should 
be realized however that then the dimension of the already existing dummy 
indices may change with it.

The scheme that is used to renumber\index{renumber} the 
indices\index{indices!renumber} in a term is quite 
involved. It will catch nearly all possibilities, but in order to avoid 
to try all $n!$ permutations, when there are n pairs of dummy indices, 
{\FORM} does not try everything. It is possible to come up with examples 
in which the scheme is not perfect. It is left as a 
challenge for the reader to find such an example. In the case that the 
scheme isn't sufficient one can use the Renumber statement (see 
\ref{substarenumber}) to force a complete renumbering. As this involves 
n! attempts in which n is the number of different dummy indices, this can 
become time consuming.

These dummy indices can be used to solve a well known problem in the 
automatic summation of indices. This problem occurs, when summed indices 
are found inside a subexpression that is raised to a power:
% THIS EXAMPLE IS PART OF THE TESTSUITE. CHANGES HERE SHOULD BE APPLIED THERE AS
% WELL!
\begin{verbatim}
    Index mu,nu;
    CFunctions f,g;
    Vectors p,q;
    Local F = (f(mu)*g(mu))^2;
    sum mu;
    id f(nu?) = p(nu);
    id g(nu?) = q(nu);
    print;
    .end

   F =
      p.p*q.q;
\end{verbatim}
Clearly the answer is not what we had in mind, when we made the program. 
There is an easy way out:
% THIS EXAMPLE IS PART OF THE TESTSUITE. CHANGES HERE SHOULD BE APPLIED THERE AS
% WELL!
\begin{verbatim}
    Index mu,nu;
    Symbol x;
    CFunctions f,g;
    Vectors p,q;
    Local F = x^2;
    repeat;
        id,once,x = f(mu)*g(mu);
        sum mu;
    endrepeat;
    id f(nu?) = p(nu);
    id g(nu?) = q(nu);
    print;
    .end

   F =
      p.q^2;
\end{verbatim}
This time things went better, because each sum-statement moves an index 
mu to a new dummy index.

There are some extra problems connected to dummy indices. Assume that we 
have the expression F which contains
% THIS EXAMPLE IS PART OF THE TESTSUITE. CHANGES HERE SHOULD BE APPLIED THERE AS
% WELL!
\begin{verbatim}
     F = f(N1_?,N2_?)*f(N2_?,N1_?);
\end{verbatim}
and next we have the module
% THIS EXAMPLE IS PART OF THE TESTSUITE. CHANGES HERE SHOULD BE APPLIED THERE AS
% WELL!
\begin{verbatim}
     Indices mu,nu,rho,si;
     Vectors p1,p2,p3,v;
     Tensor g;
     Local G = e_(mu,nu,rho,si)*g(mu,nu,p1,v)*g(rho,si,p2,v);
     sum mu,nu,rho,si;
     Multiply F^3;
     id  v = e_(p1,p2,p3,?);
     print;
     .end

   G =
      f(N1_?,N2_?)*f(N2_?,N1_?)*f(N3_?,N4_?)*f(N4_?,N3_?)*
      f(N5_?,N6_?)*f(N6_?,N5_?)*g(N7_?,N8_?,p1,N9_?)*
      g(N10_?,N11_?,p2,N12_?)*e_(p1,p2,p3,N9_?)*
      e_(p1,p2,p3,N12_?)*e_(N7_?,N8_?,N10_?,N11_?);
\end{verbatim}
Here the situation with the dummy indices becomes rather messy, and all 
earlier versions of {\FORM} were not prepared for this. Their answer could be:
\begin{verbatim}
    G =
      f(N1_?,N2_?)*f(N1_?,N2_?)*f(N1_?,N2_?)*f(N2_?,N1_?)*
      f(N2_?,N1_?)*f(N2_?,N1_?)*g(N1_?,N2_?,p2,N3_?)*
      g(N4_?,N5_?,p1,N6_?)*e_(p1,p2,p3,N3_?)*
      e_(p1,p2,p3,N6_?)*e_(N1_?,N2_?,N4_?,N5_?);
\end{verbatim}
which is clearly not what the program is supposed to give. In the current 
version we have made the tracing of the dummy indices and the renumbering 
of them at the proper moment a lot better. It is however not complete as a 
complete implementation might severely influence the speed of execution at 
some points. The scheme is complete for the inclusion of local and global 
expressions. On the other hand it doesn't work for the contents of dollar 
variables\index{variables!dollar}. Neither does it work for dummy indices 
introduced in user defined code as in
\begin{verbatim}
     id  x^n? = (f(N1_?)*g(N1_?))^n;
\end{verbatim}
For the latter case we showed a workaround above. Anyway there is a certain 
ambiguity here. Just imagine we write
\begin{verbatim}
     id  x^n? = f(N1_?)^n*g(N1_?)^n;
\end{verbatim}
Formally it is exactly the same, but what we mean is far from clear. For 
the dollar variables we considered the contracted dummy indices rare enough 
that it doesn't merit sacrificing speed. And then there is one more little 
caveat\index{caveat}. Global expressions that were stored with older 
versions of {\FORM} than version 3.2, but are read with version 3.2 or later 
would have a problem if the expression were to contain dummy indices. The 
newer version of the .sav files\index{files!.sav} will contain information 
about the dummy indices. {\FORM} can still read the old versions but will 
have to `invent' information by assuming that there are no dummy indices. 
If there are expressions with such dummy indices the best is to copy the 
expressions to a new expression and let the copying be followed by a .sort. 
That should set things straight. A final remark: if an elegant solution is 
found with which the above cases could be made to work without the penalty 
in execution time, it will be built in in the future.

\section{Kronecker delta's}

\label{sect-kroneckerdelta}
The built in object d\_ represents the Kronecker\index{Kronecker} 
delta\index{delta!Kronecker}. Even though this 
object looks a little bit like a tensor, internally it isn't treated as 
such. Actually it has its own data type. It must have exactly two arguments 
and these arguments should be either indices or vectors. A d\_ with at 
least one vector is immediately replaced, either by a vector with an index 
(if there is one vector and one index) or by a dotproduct (when there are 
two vectors). If a Kronecker delta contains an index that occurs also at 
another position in the same term, and if that index is summable, and if 
the index occurs as the index of a vector, inside a tensor, inside another 
d\_ or as the argument of a function, and the object inside which it occurs 
is not inside the argument of a function itself (unless the d\_ is inside 
the same argument) then the Einstein\index{Einstein} 
summation\index{summation!Einstein} convention is used and the 
d\_ is eliminated, while the second occurrence of the index is replaced by 
the other index in the d\_ (Are you still with us?). When a 
Kronecker delta has two identical indices and these indices are summable, 
the d\_ is replaced by the dimension of the index. If they are fixed 
indices, the d\_ is replaced by one, unless this value has been altered 
with the fixindex-statement. Some examples of Kronecker delta's are given 
in section~\ref{fund}.

\section{Extra Symbols}

\label{sect-extrasymbols}
\label{extrasymbols}
Starting with version 4.0 \FORM{} is equipped with a mechanism to replace
non-symbol objects by internally generated symbols. These are called the 
extra symbols. Their numbering starts at maximum number allowed for 
internal objects and then counts down. Hence their ordering will be 
opposite to what might otherwise be expected. It is possible to control 
their representation when they are to be printed in the output. For this 
there is the ExtraSymbols(\ref{substaextrasymbols}) statement. The 
definitions of the extra symbols can be made visible with the \%X option in 
the \#write preprocessor instruction.

Extra symbols can be introduced by the user with the ToPolynomial statement 
(\ref{substatopolynomial}). This statement replaces all objects that are 
not numbers or symbols to positive powers by extra symbols. This may be 
needed for some new manipulations and can also be very handy for output 
that is to be treated by for instance a FORTRAN or C compiler. The 
FromPolynomial statement replaces the extra symbols again by their original 
meaning.

% THIS EXAMPLE IS PART OF THE TESTSUITE. CHANGES HERE SHOULD BE APPLIED THERE AS
% WELL!   Not yet
\begin{verbatim}
    Vector p,q,p1,p2;
    CFunction f;
    CFunction Dot,InvDot;
    Symbol x,x1,x2;
    Set pdot:p,q;
    Off Statistics;
    Local F = x+x^2+1/x+1/x^2+f(x1)+f(x2)*p.q*x+f(x2)/p.q^2;
    id  p1?pdot.p2?pdot = Dot(p1,p2);
    id  1/p1?pdot.p2?pdot = InvDot(p1,p2);
    Print;
    .sort

   F =
      x^-2 + x^-1 + x + x^2 + f(x1) + f(x2)*Dot(p,q)*x + f(x2)*InvDot(p,q)^2;

    ExtraSymbols,array,Y;
    Format DOUBLEFORTRAN;
    ToPolynomial;
    Print;
    .sort

      F =
     & Y(1) + Y(1)**2 + Y(2) + Y(5)**2*Y(3) + x + x*Y(4)*Y(3) + x**2

    #write <sub.f> "      SUBROUTINE sub(Y)"
    #write <sub.f> "*"
    #write <sub.f> "*      Compute the extra symbols. Generated on `DATE_'"
    #write <sub.f> "*"
    #write <sub.f> "      REAL*8 Y(`EXTRASYMBOLS_')"
    #write <sub.f> "      REAL*8 Dot,InvDot"
    #write <sub.f> "      Dot(p1,p2)=p1(1)*p2(1)-p1(2)*p2(2)-p1(3)*p2(3)\
                                                              -p1(4)*p2(4)"
    #write <sub.f> "      InvDot(p1,p2)=1.D0/(Dot(p1,p2))"
    #write <sub.f> "*"
    #write <sub.f> "*        We still have to add definitions here."
    #write <sub.f> "*        And we have to import all the variables."
    #write <sub.f> "*"
    #write <sub.f> "%X"
    #write <sub.f> "*"
    #write <sub.f> "      RETURN"
    #write <sub.f> "      END"
    ExtraSymbols,underscore,Z;
    Format Normal;
    Format 80;
    Print;
    .end

   F =
      Z1_ + Z1_^2 + Z2_ + Z5_^2*Z3_ + x + x*Z4_*Z3_ + x^2;

    FromPolynomial;
    Print;
    .end

   F =
      x^-2 + x^-1 + x + x^2 + f(x1) + f(x2)*Dot(p,q)*x + f(x2)*InvDot(p,q)^2;

\end{verbatim}
In the ExtraSymbols statement we say that we want the extra symbols to be 
presented as an array with the name Y. The alternative is a set of symbols 
with names ending in an underscore, but that would not make the FORTRAN 
compiler very happy. Then we convert the expression to symbols. As one can 
see, everything got converted to elements of an array Y which are treated 
as symbols. After we have written the file sub.f (notice that 
EXTRASYMBOLS\_ is a built in symbol indicating the number of extra symbols) 
we change the representation to the (default) notation with an underscore 
and the character Z. The contents of the file sub.f are:
\begin{verbatim}
      SUBROUTINE sub(Y)
*
*      Compute the extra symbols. Generated on Sat Apr  2 20:40:33 2011
*
      REAL*8 Y(5)
      REAL*8 Dot,InvDot
      Dot(p1,p2)=p1(1)*p2(1)-p1(2)*p2(2)-p1(3)*p2(3)-p1(4)*p2(4)
      InvDot(p1,p2)=1.D0/(Dot(p1,p2))
*
*        We still have to add definitions here.
*        And we have to import all the variables.
*
      Y(1)=x**(-1)
      Y(2)=f(x1)
      Y(3)=f(x2)
      Y(4)=Dot(p,q)
      Y(5)=InvDot(p,q)

*
      RETURN
      END
\end{verbatim}
As one can see, with very little effort this routine can be made into a 
proper subroutine that computes all elements of the array Y which can then 
be used for computing the expression F.


\section{Restrictions}

There is a restriction\index{restrictions} on the total number of 
variables\index{variables!total number of} that {\FORM} can handle. For the 
number of symbols, vectors, indices, functions and sets together the exact 
number depends on the type of computer. For a computer with a 32-bits 
processor this number is 32768. This includes the built in objects. 
Individual types of variables (like symbols) are usually restricted to 
about 8000. For a 
computer with a 64-bits processor the maximum has been set arbitrarily at 
2000000000. In addition there are restrictions on the total amount of 
memory\index{memory!total amount of} needed by {\FORM} to maintain an 
administration of all these variables. These restrictions are set by the 
memory allocator of the computer on which {\FORM} is running.

\section{Some common bugs}

There is a type of error\index{error}\index{bug} by the user (including at 
times the author) that is so common that it deserves mentioning here. 
Consider the code:
\begin{verbatim}
     Symbol x1,x2
     Index m1,m2;
\end{verbatim}
As a statement it is perfectly legal\index{legal}, but it may produce 
rather funny errors at a later stage when we try to use m1 or m2. 
Inspection with the `On names;' statement shows that we have the symbols 
x1,x2,Index,m1,m2. This is most likely not what the user wanted. Closer 
inspection shows that we forgot the semicolon at the end of the symbol 
statement. We should have had:
\begin{verbatim}
     Symbol x1,x2;
     Index m1,m2;
\end{verbatim}
This is the most common error for which {\FORM} cannot give a direct error 
message (it is after all a legal statement). Hence when faced with 
mysterious errors or error messages, one could have a good look by using 
the `On names' statement. Maybe it shows something, and if not, one has to 
look for other causes.
