
\chapter{The dollar variables}
\label{dollars}

In the older versions of \FORM\ there were two types of variables: the 
preprocessor variables\index{variables!preprocessor} and the algebraic 
variables\index{variables!algebraic}. The preprocessor 
variables are string variables that are used by the edit features of the 
preprocessor to prepare the input for the compiler part of \FORM\@. The 
algebraic objects are the expressions and the various algebraic variables 
like the symbols, functions, vectors etc. There existed however very few 
possibilities to communicate from the algebraic level to the decision 
taking at the preprocessor level. This has changed dramatically with 
version 3 and the introduction of the dollar 
variables\index{variables!dollar}.

Dollar variables are basically (little) expressions that can be used to 
store various types of information. They can be used both as preprocessor 
objects as well as algebraic objects. They can also be defined and given 
contents both by the preprocessor and during execution on a term by term 
basis. Dollar variables are kept in memory. Hence it is important not to 
make them too big, because in that case performance might suffer.

What is a legal name for a dollar variable?
Dollar variables have a name that consists of a dollar sign (\verb:$:) 
followed by an alphabetic character and then potentially more alphanumeric 
characters. Hence \verb:$a: and \verb:$var: and \verb:$r4t78y0: are legal 
names and \verb:$1a: is not a legal name. The variables do not have to be 
declared. However \FORM\ will complain if a dollar variable is being used, 
before it has encountered a statement or an instruction in which the 
variable has been given a value. Hence giving a variable a value counts at 
the same time as a declaration.

What can be stored in a dollar variable?
\begin{itemize}
\item Algebraic expressions as in \verb:$var = (a+b)^2;:
\item Individual objects like indices, numbers, symbols.
\item Zero.
\item Parts of a term.
\item Argument\index{argument field} fields that consist of zero, one or 
more arguments.
\end{itemize}
Actually, the parts of a term are treated as a complete term and hence as a 
special case of an algebraic expression. Internally they are stored 
slightly differently for speed, but at the user level this should not be 
noticeable. Actually, with the exception of the argument fields, \FORM\ can 
convert one type into the other and will try so, depending on the use that 
is made of the specific dollar variable. In the case that a variable is 
used in a way that should not be possible (like the content of a variable 
is a symbol, but it is used in a position where an index is expected) there 
will be a runtime\index{runtime error} error\index{error!runtime}.

How is a variable used?
\begin{itemize}
\item As a preprocessor variable\index{variable!preprocessor}. This is done 
by putting the variable between a pair of `' as in \verb:`$var':. In this 
case the regular print routines of \FORM\ make a textual 
representation\index{representation!textual} of 
the variable as it exists at the moment that the preprocessor encounters 
this object, and this string is then substituted by the preprocessor as if 
it were the contents of a preprocessor variable.
\item Like an expression during execution time. This would be the case in 
the statement
\begin{verbatim}
    id x = y + $var;
\end{verbatim}
in which \verb:$var: is substituted 
in a way that is similar to the substitution of a local expression \verb:F: 
in the statement
\begin{verbatim}
    id x = y + F;
\end{verbatim}
except for that the dollar variable is always stored in the CPU memory.
\item As an algebraic object during execution time. This could be the case 
with any value of the variable that is not an expression. An example would 
be
\begin{verbatim}
    id f(?a) = f(?a,$var);
\end{verbatim}
in which the dollar variable contains an 
argument field.
\item As an algebraic object in a delayed substitution of a pattern or a 
special statement. This may need some clarification. If we have the 
statement
\begin{verbatim}
    id f($var) = anything;
\end{verbatim}
the compiler does not substitute 
the current value of \verb:$var:. The reason is that \verb:$var: could have 
a different value for each term that runs into this statement, while the 
compiler compiles the statement only once. Hence \FORM\ will substitute the 
value of \verb:$var: only at the moment that it will attempt the pattern 
matching. This is called delayed\index{delayed substitution} 
substitution\index{substitution!delayed}. If one likes the compiler to 
substitute a value, one can basically let the preprocessor take care of 
this by typing
\begin{verbatim}
    id f(`$var') = anything;
\end{verbatim}
A similar delayed 
substitution takes place in statements of the type \verb:Trace,$var;:.
\end{itemize}

How does one give a value to a dollar variable?
\begin{itemize}
\item In the preprocessor. This is done with an instruction of the type 
\verb:#$var = 0;:. This is an instruction that can run over more than one 
line. The r.h.s.\ can be any algebraic expression. Specifically it can 
contain dollar variables or local/global expressions. Such 
expressions are worked out during the preprocessing. Hence this variable 
acquires a value immediately.
\item During execution when control reaches a statement of the form 
\verb:$var = expression;:. Again the r.h.s.\ can contain any normal 
algebraic expression including dollar variables and local/global 
expressions. The r.h.s.\ will be evaluated and the value will be assigned to 
\verb:$var:. In the case that \verb:$var: had already a value, the old 
value will be deleted and the new value will be `installed'.
\item During execution when the dollar variable is assigned the value of a 
wildcard as in
\begin{verbatim}
    id f(x?$var) = whatever;
\end{verbatim}
If the function \verb:f: 
occurs more than once in a term, \verb:$var: will have the value of the 
last match. In the case that the value of the first match is needed one can 
use the option `once' in the id-statement as in
\begin{verbatim}
    id,once,f(x?$var) = whatever;
\end{verbatim}
In general one can paste the dollar variable to the end of any 
wildcard description. Hence one can use \verb:id f(x?{1,2,3}$var) = ...;: 
and
\begin{verbatim}
    id f(x?set[n?$var1]$var2) = ...;
\end{verbatim}
\end{itemize}
Note the difference between \verb:#$a = 0;: and \verb:$a = 0;:. One CANNOT 
make a wildcard\index{wildcard} construction for dollar variables 
themselves as in \verb:id f($var?) = ...;:

Dollar variables CANNOT have arguments as in \verb:$var(2): or something 
equivalent. There is however a solution at the preprocessor level for this 
by defining individual variables \verb:$var1: to \verb:$varn: and then 
using \verb:$var`i': or \verb:`$var`i'': for some preprocessor variable 
\verb:i:. The exception is the indication of factors when a dollar variable 
has been factorized (see the \#factdollar instruction~\ref{prefactdollar} 
and the factdollar statement~\ref{substafactdollar}). This is explained 
later in this chapter and in the chapter about 
polynomials~\ref{polynomials}.

Printing dollar\index{dollar!printing} variables:
\begin{itemize}
\item In the preprocessor one can use the \verb:#write: instruction
        (see \ref{prewrite}).
\item During execution one can use the Print statement (see 
        \ref{substaprint}).
\end{itemize}
In both cases one should use the format\index{format} string. The syntax is 
described in the chapters on these statements. The format descriptor of a 
dollar variable is \verb:%$: and this looks after the format string for the 
next dollar variable. Of course one can also use the dollar variable as a 
preprocessor variable when printing/writing in the preprocessor.

Examples.

Counting terms:
\begin{verbatim}
    S	a,b;
    Off statistics;
    L	F = (a+b)^6;
    #$a = 0;
    $a = $a+1;
    Print "      >> After %t we have %$ term(s)",$a;
    #write "     ># $a = `$a'"
     ># $a = 0
    .sort
      >> After  + a^6 we have 1 term(s)
      >> After  + 6*a^5*b we have 2 term(s)
      >> After  + 15*a^4*b^2 we have 3 term(s)
      >> After  + 20*a^3*b^3 we have 4 term(s)
      >> After  + 15*a^2*b^4 we have 5 term(s)
      >> After  + 6*a*b^5 we have 6 term(s)
      >> After  + b^6 we have 7 term(s)
    #write "     ># $a = `$a'"
     ># $a = 7
    .end
\end{verbatim}

\noindent Maximum power of x in an expression:
% THIS EXAMPLE IS PART OF THE TESTSUITE. CHANGES HERE SHOULD BE APPLIED THERE AS
% WELL! (DolVars_1)
\begin{verbatim}
    S	x,a,b;
    Off statistics;
    L	F = (a+b)^4+a*(a+x)^3;
    .sort
    #$a = 0;
    if ( count(x,1) > $a ) $a = count_(x,1);
    Print "      >> After %t the maximum power of x is %$",$a;
    #write "     ># $a = `$a'"
     ># $a = 0
    .sort
      >> After  + 3*x*a^3 the maximum power of x is 1
      >> After  + 3*x^2*a^2 the maximum power of x is 2
      >> After  + x^3*a the maximum power of x is 3
      >> After  + 4*a*b^3 the maximum power of x is 3
      >> After  + 6*a^2*b^2 the maximum power of x is 3
      >> After  + 4*a^3*b the maximum power of x is 3
      >> After  + 2*a^4 the maximum power of x is 3
      >> After  + b^4 the maximum power of x is 3
    #write "     ># $a = `$a'"
     ># $a = 3
    .end
\end{verbatim}

Starting with version 4, \FORM\ has the capability to factorize polynomials 
(see the chapter on polynomials~\ref{polynomials}). One type of objects 
that can be factorized is the dollar variables. The immediate question here 
is how to access the factors. As we mentioned before in this chapter, 
normally there is no direct way to use arguments for dollar variables. For 
the factors however we have a way of indexing the dollar variables as in 
\verb:$var[1]:,...,\verb:$var[n]: when there are n factors. The number of 
factors can be obtained as \verb:$var[0]:. In the index field can only be 
(nonnegative integer) numbers, dollar variables or factors of dollar 
variables that evaluate into (nonnegative integer) numbers.
\begin{verbatim}
    Symbol x,y;
    CFunction f1,f2;
    Local F = f1(x^2+2*x*y+y^2)+f1(x^4-y^4);
    id  f1(x?$x) = f2(x);
    FactDollar,$x;
    Do $i = 1,$x[0];
      Print "In %t factor %$ is %$",$i,$x[$i];
    Enddo;
    .end
In  + f2(y^2 + 2*x*y + x^2) factor 1 is y + x
In  + f2(y^2 + 2*x*y + x^2) factor 2 is y + x
In  + f2( - y^4 + x^4) factor 1 is  - 1
In  + f2( - y^4 + x^4) factor 2 is y - x
In  + f2( - y^4 + x^4) factor 3 is y + x
In  + f2( - y^4 + x^4) factor 4 is y^2 + x^2
\end{verbatim}

One thing to note is that the use of
\begin{verbatim}
     f(<$x[1]>,...,<$x[$x[0]]>)
\end{verbatim}
is illegal. \verb:$x[0]: will be inserted during execution time, while the 
expansion of the triple dot operator is done by the preprocessor. Hence we 
should use \verb:`$x[0]': but then \verb:$x: must be known and factorized 
already at compile time.
 
\section{Dollar variables in a parallel environment}
\label{pardollars}

When \FORM\ is used for parallel\index{parallel processing} processing, 
either by means of \ParFORM\index{ParFORM} or by means of 
\TFORM\index{TFORM}, there can be a problem with the dollar variables as in 
principle there is a central administration and dollar variables that are 
defined during running will in general have the last assigned value. In a 
parallel environment this can be nondeterministic\index{nondeterministic}. 
Look for instance at the following example:
\begin{verbatim}
    S   x,a,b;
    CF  f;
    L   F = f(a+b) + f(a+2*b);
    .sort
    id  f(x?$x) = f(x);
    Multiply,$x;
    Print;
    .end
\end{verbatim}
Usually this program will give the 'correct' answer, but in principle one 
thread could define \verb:$x: and then the next thread could overwrite this 
value before the first thread has used it. This is serious. Hence \FORM\ 
will veto\index{veto} the use of multiple threads/processors for modules in 
which dollar variables obtain values during the execution of the program, 
unless the user can give \FORM\ more information about the use of the 
dollar variables. In the above case the value of \verb:$x: will be local to 
each term and hence to each thread\index{thread}. The value in previous 
terms is unimportant. We can tell this to \FORM\ with a variety of the 
moduleoption\index{moduleoption} statement (see \ref{substamoduleoption}). 
This would be:
\begin{verbatim}
    S   x,a,b;
    CF  f;
    L   F = f(a+b) + f(a+2*b);
    .sort
    id  f(x?$x) = f(x);
    Multiply,$x;
    Print;
    ModuleOption,local,$x;
    .end
\end{verbatim}
In this case \FORM\ makes at the start of the execution of the module a 
copy of whatever value \verb:$x: has at that moment for each 
thread/processor (in this case no value yet and hence it gets set to zero) 
and then each thread/processor uses its own copy during execution. After 
the module has been completed the local copies are removed and the original 
global value is accessible again. This way execution will be safe in a 
parallel environment.

There are more cases that \FORM\ can handle in a parallel environment. 
These are also options in the moduleoption statement:
\begin{verbatim}
    ModuleOption,maximum,$a;
    ModuleOption,minimum,$b;
    ModuleOption,sum,$c;
\end{verbatim}
Here we say that \verb:$a: is accumulating a maximum numerical value, 
\verb:$b: collects a minimum numerical value and \verb:$c: is a numerical 
sum. In all three cases there is a central administration and the use of 
the variables has to be blocked for other threads/processors during the 
updating of the values. Sometimes that can be efficient, but in other 
programs that may actually make them slower. One should experiment. A 
sample program is given below:
% THIS EXAMPLE IS PART OF THE TESTSUITE. CHANGES HERE SHOULD BE APPLIED THERE AS
% WELL! (DolVarsParallel_1)
\begin{verbatim}
    S   a1,...,a10;
    L   F = (a1+...+a10)^3;
    .sort
    #$c = 0;
    Print +f "<%w> %t";
    Multiply,(a1+...+a10);
    $c = $c+1;
    ModuleOption,sum,$c;
    .sort
    #message $c = `$c'
    #$max = 0;
    #$min = 10;
    if ( count(a1,1) > $max ) $max = count_(a1,1);
    if ( count(a4,1) < $min ) $min = count_(a4,1);
    ModuleOption,maximum,$max;
    ModuleOption,minimum,$min;
    .sort
    #message $max = `$max'
    #message $min = `$min'
    .end
\end{verbatim}
The print statement is showing which thread is dealing with which term.

