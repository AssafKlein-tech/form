
\chapter{The setup}
\label{setup}

When FORM is started, it has a number of settings\index{setup} built in 
that were determined during its installation\index{installation}. If the 
user would like to alter these settings, it is possible to either specify 
their desired values in a setup file\index{file!setup} or to do so at the 
beginning of the program file\index{file!program}. There are two ways in 
which FORM can find a setup file. The first way is by having a file named 
`form.set'\index{form.set} in the current directory. If such a file is 
present, FORM will open it and interpret its contents as setup parameters. 
If this file is not present, one may specify a setup file with the -s 
option in the command tail. This option must precede the name of the input 
file. After the -s follow one or more blanks or tabs and then the full name 
of the setup file. FORM will try to read startup parameters from this file. 
If a file `form.set' is present, FORM will ignore the -s option and its 
corresponding file name. This order of interpretation allows the user to 
define an alias with a standard setup file which can be overruled by a 
local setup file. If, in the beginning of the program file, before any 
other statements with the exception of the \#- instruction and commentary 
statements, there are lines that start with \#: the remaining contents of 
these lines are interpreted exactly like the lines in the setup file. The 
specifications in the program file take precedence\index{precedence} over 
all other specifications. If neither of the above methods is used, FORM 
will use a built in set of parameters. Their values may depend on the 
installation and are given below. 

The following is a list of parameters that can be set. The syntax is rather 
simple: The full word must be specified (case insensitive), followed by one 
or more blanks or tabs and the desired number, string or character. 
Anything after this is considered to be commentary. In the setup file lines 
that do not start with an alphabetic character are seen as commentary. The 
sizes of the buffers are given in bytes, unless mentioned otherwise. A word 
is 2 bytes for 32\index{32 bits} bit machines and 4 bytes for 64\index{64 
bits} bit machines.

In FORM version 3.3 and later, it is also allowed to define 
preprocessor variables\index{preprocessor variables} (see also 
\ref{preprovariables}) in the setup file. In addition one can use 
preprocessor variables in the setup, provided it is not in the name of the 
parameter/keyword.

\leftvitem{3.8cm}{bracketindexsize\index{setup!bracketindexsize}\index{bracketindexsize}}
\rightvitem{12.6cm}{Maximum size in bytes of any individual index of a 
bracketted expression. Each expression will have its own index. The index 
starts with a relatively small size and will grow if needed. But it will 
never grow beyong the specified size. If more space is needed, FORM will 
start skipping brackets and find those back later by linear search. See 
also chapter~\ref{brackets} and section~\ref{substabracket}.}

\leftvitem{3.8cm}{CommentChar\index{setup!commentchar}\index{commentchar}}
\rightvitem{12.6cm}{This should be followed by one or more blanks and a 
single non-blank character. This character will be used to indicate 
commentary, instead of the regular $*$ in column 1.}

\leftvitem{3.8cm}{CompressSize\index{setup!compresssize}\index{compresssize}}
\rightvitem{12.6cm}{When compressing output terms, FORM needs a compression 
buffer. This buffer deals recursively with compression and decompression of 
terms that are either written or read. Its size will be at least 
MaxTermSize but when there is heavy use of expressions in the right hand 
side of definitions or substitution it would have to be considerably 
longer. It is hoped that in the future this parameter can be eliminated. 
CompressSize should be given in bytes.}

\leftvitem{3.8cm}{ConstIndex\index{setup!constindex}\index{constindex}}
\rightvitem{12.6cm}{This is the number of indices that are considered to be 
constant indices like in fixed vector components (the so-called fixed 
indices). The size of this parameter is not coupled to any array space, but 
it should not go much beyond 1000 on a 32\index{32 bits} bit machine. On a 
64\index{64 bits} bit machine it can go considerably further.}

\leftvitem{3.8cm}{ContinuationLines\index{setup!continuationlines}\index{continuationlines}}
\rightvitem{12.6cm}{The number of continuation lines that the local Fortran 
compiler will allow. This limits the number of continuation lines, when the 
output option `Format Fortran' (see \ref{substaformat}) is selected.}

\leftvitem{3.8cm}{Define\index{setup!define}\index{define}}
\rightvitem{12.6cm}{The syntax is as in the \#define instruction in the 
preprocessor (see \ref{preprovariables}), with the remark that in the setup 
file there should be no leading \# character as that would make the line 
into commentary. Example: \hfill \\
{\tt\ \ \ \ define MODULUS "31991"} \hfill \\
which could be used 
at a later point in the program to activate a modulus statement (see 
\ref{substamodulus}).}

\leftvitem{3.8cm}{DotChar\index{setup!dotchar}\index{dotchar}}
\rightvitem{12.6cm}{There should be a single character following this name 
(and the blank(s) after it). This character will be used instead of the \_, 
when dotproducts\index{dotproducts} are printed in Fortran\index{Fortran} 
output. This option is needed because some Fortran compilers do not 
recognize the underscore as a valid character. In the olden days one could 
use here the dollar character but nowadays many Fortran compilers do not 
recognize this character as belonging to a variable name.}

\leftvitem{3.8cm}{FunctionLevels\index{setup!functionlevels}\index{functionlevels}}
\rightvitem{12.6cm}{The maximum number of levels that may occur, when 
functions have functions in their arguments.}

\leftvitem{3.8cm}{HideSize\index{setup!hidesize}\index{hidesize}}
\rightvitem{12.6cm}{The size of the hide buffer. The size of this buffer is 
normally set equal to scratchsize (see below). If one uses the setting of 
HideSize after the setting of ScratchSize, one can give the hide buffer its 
own size. There are cases that this can make the program faster.}
 
\leftvitem{3.8cm}{IncDir\index{setup!incdir}\index{incdir}}
\rightvitem{12.6cm}{Directory (or path of directories) in which FORM will 
look for files if they are not to be found in the current directory. This 
involves files for the \#include\index{\#include} and \#call\index{\#call} 
instructions. This variable takes precedence over the 
Path\index{setup!path}\index{path} variable.}

%\leftvitem{3.8cm}{IndentSpace\index{setup!indentspace}\index{indentspace}}
%\rightvitem{12.6cm}{}

\leftvitem{3.8cm}{InsideFirst\index{setup!insidefirst}\index{insidefirst}}
\rightvitem{12.6cm}{Not having any effect at the moment.}
 
\leftvitem{3.8cm}{MaxNumberSize\index{setup!maxnumbersize}\index{maxnumbersize}}
\rightvitem{12.6cm}{Allows the setting of the maximum size of the numbers 
in FORM. The number should be given in words. For 32\index{32 bits} bit 
systems a word is two bytes and for 64\index{64 bits} bit systems a word is 
4 bytes. The number size is always limited by the maximum size of the terms 
(see MaxTermSize). Actually it has to be less than half of MaxTermSize 
because a coefficient contains both a numerator and a denominator. It is 
not always a good idea to have the number size at its maximum value, 
especially when MaxTermSize is large. In that case it could be very long 
before a runaway algorithm runs into limitations of size (arithmetic for 
very long fractions is not very fast due to the continuous need for 
computing GCD's)}

\leftvitem{3.8cm}{MaxTermSize\index{setup!maxtermsize}\index{maxtermsize}}
\rightvitem{12.6cm}{\label{setupmaxtermsize}
This is the maximum size that an individual term may occupy in words. This 
size does not affect any allocations. One should realize however that the 
larger this size is the heavier the demand can be on the workspace, because 
the workspace acts as a heap during the execution and sometimes allocations 
have to be made in advance, before FORM knows what the actual size of the 
term will be. Consequently the evaluation tree cannot be very deep, when 
WorkSpace / MaxTermSize is not very big. MaxTermSize controls mainly how 
soon FORM starts complaining about terms that are too complicated. Its 
absolute maximum is 32568 on 32\index{32 bits} bit systems and about $10^9$ 
on 64\index{64 bits} bit systems (of course the workspace would have to be 
considerably larger than that....).}

\leftvitem{3.8cm}{MaxWildCards\index{setup!maxwildcards}\index{maxwildcards}}
\rightvitem{12.6cm}{The maximum number of wildcards that 
can be active in a single matching of a pattern. Under normal circumstance 
the default value of 100 should be more than enough.}

\leftvitem{3.8cm}{NoSpacesInNumbers\index{setup!nospacesinnumbers}\index{nospacesinnumbers}}
\rightvitem{12.6cm}{\label{nospacesinnumbers}Long numbers are usually spread over several lines 
by placing a backspace character at the end of each line and then 
continuing at the next line. For cosmetic purposes FORM puts usually a few 
blank spaces at the beginning of the new line. FORM itself can read this but 
some programs cannot. Hence one can put FORM in a mode in which these 
blanks are omitted. The values of the variable are ON or OFF. There is also 
a command to change this behaviour at runtime. See the on and off commands 
at pages \ref{staonnospacesinnumbers} and \ref{staoffnospacesinnumbers}.}

\leftvitem{3.8cm}{NumStoreCaches\index{setup!numstorecaches}\index{numstorecaches}}
\rightvitem{12.6cm}{This number determines how many store caches (see 
the description of the SizeStoreCache setup parameter below) there will 
be. In the case of parallel processing this will be the number of caches 
per processor.}

\leftvitem{3.8cm}{NwriteStatistics\index{setup!nwritestatistics}\index{nwritestatistics}}
\rightvitem{12.6cm}{When this word is mentioned, the default setting for the 
statistics is that no run time statistics will be shown. Ordinarily they 
will be shown.}

\leftvitem{3.8cm}{NwriteThreadStatistics\index{setup!nwritethreadstatistics}
\index{nwritethreadstatistics}}
\rightvitem{12.6cm}{This variable has the values ON or OFF. It controls for 
TFORM whether the statistics of the individual threads will be printed. The 
default value is ON.}
 
\leftvitem{3.8cm}{OldOrder\index{setup!oldorder}\index{oldorder}}
\rightvitem{12.6cm}{A special flag (values ON/OFF) by which one can still 
select the old option of not checking for the order of statements inside a 
module. This should be used only in the case that it is nearly impossible 
to change a program to the new mode in which the order of the statements 
(declarations etc) is relevant. In the future this old mode may not exist.}
 
\leftvitem{3.8cm}{Parentheses\index{setup!parentheses}\index{parentheses}}
\rightvitem{12.6cm}{The maximum number of nestings of parentheses or 
functions inside functions. The variable may be eliminated in a later 
version.}
 
\leftvitem{3.8cm}{Path\index{setup!path}\index{path}}
\rightvitem{12.6cm}{Directory (or path of directories) in which FORM will 
look for files if they are not to be found in the current directory. This 
involves files for the \#include\index{\#include} and \#call\index{\#call} 
instructions. FORM will test this path after a potential path specified as 
IncDir\index{setup!incdir}\index{incdir}.}

%\leftvitem{3.8cm}{PolyGCDchoice\index{setup!polygcdchoice}\index{polygcdchoice}}
%\rightvitem{12.6cm}{}
 
\leftvitem{3.8cm}{ProcedureExtension\index{setup!procedureEetension}\index{procedureextension}}
\rightvitem{12.6cm}{The extension that will be used by FORM for finding the 
procedures that are in separate files. Restrictions on the strings used are 
as explained in the preprocessor 
\#procedureextension\index{\#procedureextension} instruction in section 
\ref{preprocedureextension}.}
 
\leftvitem{3.8cm}{ResetTimeOnClear\index{setup!resettimeonclear}\index{resettimeonclear}}
\rightvitem{12.6cm}{The value is ON or OFF. The default value is ON. This 
means that by default the clock is reset after each .clear\index{.clear} 
(see chapter \ref{modules} on modules) instruction at the end of a module.}

\leftvitem{3.8cm}{ScratchSize\index{setup!scratchsize}\index{scratchsize}}
\rightvitem{12.6cm}{The size of the input and the output buffers for the 
regular algebra processing. Terms are read in in chunks this size and are 
written to the output file using buffers of this size. There are either two 
or three of these buffers, depending on whether the hide\index{hide} 
facility is being used (see \ref{substahide}). These buffers must have a 
size that is at least as large as the MaxTermSize\index{maxtermsize}. These 
buffers act as caches for the files with the extension .sc1\index{.sc1}, 
.sc2\index{.sc2} and .sc3\index{.sc3}. See also the HideSize parameter 
above for the independent setting of the size of the hide buffer.}

\leftvitem{3.8cm}{SizeStoreCache\index{setup!sizestorecache}\index{sizestorecache}}
\rightvitem{12.6cm}{The size of the caches\index{caches} that are used for 
reading terms when stored expressions are used in the r.h.s. of a 
statement. Typically there are several such caches and they make the 
reading much faster. In the case of parallel processing these caches become 
very important because without them the different processes may all want to 
read from the .str\index{.str} file\index{file!store} at the same time and 
execution speed will suffer badly. The number of store caches is determined 
by the NumStoreCaches\index{numstorecaches} setup parameter which is 
described above. The size of these caches doesn't have to be very large as 
compared to some of the other buffers. It is recommended though to have 
them at least as large as MaxTermSize\index{maxtermsize} (see above).}
 
\leftvitem{3.8cm}{SlavePatchSize\index{setup!slavepatchsize}\index{slavepatchsize}}
\rightvitem{12.6cm}{For the parallel version. It is ignored in sequential 
versions. Tells FORM how big the patches of terms that are being 
distributed over the secondary processors should be. See also 
\ref{substaslavepatchsize}. If the number is too small, FORM will basically 
send individual terms.}
 
\leftvitem{3.8cm}{SortType\index{setup!sorttype}\index{sorttype}}
\rightvitem{12.6cm}{Possible values are "lowfirst"\index{lowfirst}, 
"highfirst"\index{highfirst} and "powerfirst"\index{powerfirst}. "lowfirst" 
is the default. Determines the order in which the terms are placed during 
sorting. In the case of lowfirst, lower powers of symbols and dotproducts 
come before higher powers. In the case of highfirst it is the opposite. In 
the case of powerfirst the combined powers of all symbols together are 
considered and the highest combined powers come first. See also the 
on\index{on} statement in \ref{substaon}.}

\leftvitem{3.8cm}{TempDir\index{setup!tempdir}\index{tempdir}}
\rightvitem{12.6cm}{This variable should contain the name of a directory 
that is the directory in which FORM should make its temporary files. If the 
-t (or -T) option is used when FORM is started, the TempDir variable in the 
setup file is ignored. FORM can create a number of different temporary 
files.}
 
\leftvitem{3.8cm}{ThreadBucketSize\index{setup!threadbucketsize}\index{threadbucketsize}}
\rightvitem{12.6cm}{Only relevant for \TFORM. The size of the number of 
terms sent to the workers simultaneously. 
For details see the chapter on the parallel version (\ref{parallel}).}
 
\leftvitem{3.8cm}{ThreadLoadBalancing\index{setup!threadloadbalancing}\index{threadloadbalancing}}
\rightvitem{12.6cm}{\indent Only relevant for \TFORM. Possible values are ON 
or OFF. For details see the chapter on the parallel version (\ref{parallel}).}
 
\leftvitem{3.8cm}{Threads\index{setup!threads}\index{threads}}
\rightvitem{12.6cm}{Only relevant for \TFORM (see chapter on the parallel 
version). Specifies the default number of worker threads to be used. The 
values 0 and 1 will indicate that running will only be done by the master 
thread (\ref{parallel}).}

\leftvitem{3.8cm}{ThreadScratchOutSize\index{setup!threadscratchoutsize}\index{threadscratchoutsize}}
\rightvitem{12.6cm}{The size of the output scratch buffers for each of the 
worker threads. These buffers will be used when the InParallel 
statement~\ref{substainparallel} is active. They are used to catch the 
output of the expressions as processed by the individual workers before 
they are copied to the output scratch buffer/file of the master. The output 
scratch buffer/file of each worker will never contain more than one 
expression at a time.}

\leftvitem{3.8cm}{ThreadScratchSize\index{setup!threadscratchsize}\index{threadscratchsize}}
\rightvitem{12.6cm}{The size of the input scratch buffers for each of the 
worker threads. These buffers are only used when the main scratch buffers 
of the master process isn't sufficient and scratch files have been made. 
When the buffers of the master are big enough, the workers only use 
pointers to the buffer of the master. Once there are scratch files the 
buffer is used for caching the input from those files. In that case each 
worker has its own cache. For reading purposes it can actually be counter 
productive if these buffers are very large. This parameter sets the value 
for the input and the hide\index{hide} scratch files. The output scratch 
size for the workers is set with the ThreadScratchOutSize parameter.}

%\leftvitem{3.8cm}{ThreadSortFileSynch\index{setup!threadsortfilesynch}\index{threadsortfilesynch}}
%\rightvitem{12.6cm}{\indent Only relevant for \TFORM. Possible values are ON 
%or OFF. For details see the chapter on the parallel version (\ref{parallel}).}

\leftvitem{3.8cm}{WorkSpace\index{setup!workspace}\index{workspace}}
\rightvitem{12.6cm}{The size of the heap that is used by the algebra 
processor when it is evaluating the substitution tree. It will contain 
terms, half finished terms and other information. The size of the workspace 
may be a limitation on the depth of a substitution tree.}

%\leftvitem{3.8cm}{ZipSize\index{setup!zipsize}\index{zipsize}}
%\rightvitem{12.6cm}{This parameter defines the maximum size of a buffer for 
%compression and decompression of parts of expressions with the ZIP 
%algorithm. Currently this algorithm is not in use yet, but it is one of the 
%relatively urgent future projects. ZIP compression can reduce intermediate 
%files by a factor 4.}
%%%%%%%for compressing .sav files?

Variables that take a path\index{path} for their value expect a sequence of 
directories, separated by colon characters as in the UNIX\index{UNIX} way 
to define such objects.

The above parameters are conceptually relatively easy. The parameters that 
are still left are more complicated and are often restricted in their 
size by some relationships. Hence it is necessary to understand the 
sorting inside FORM a little bit before using them. On the other hand 
these parameters can influence the performance noticeably. See also chapter 
\ref{sorting} for more details.

When terms are send to `output' by the main algebra engine, they are put 
inside a buffer. This buffer is called the `small\index{small buffer} 
buffer\index{buffer!small}'. Its size is given by the variable {\sl 
SmallSize\index{smallsize}}. When this buffer is full, or when the number 
of terms in this buffer exceeds a given maximum, indicated by the variable 
{\sl TermsInSmall\index{termsinsmall}}, the contents of the buffer are 
sorted. The sorting is done by pointers, hence it is important that the 
small buffer resides inside the physical memory. During the sorting it may 
happen that coefficients are added. The sum of two rational numbers can 
take more space than any of the individual numbers, so there will be a 
space problem. This has been solved by the construction of an extension to 
the small buffer. The variable {\sl SmallExtension\index{smallextension}} 
is the size of the small buffer together with this extension. The value for 
SmallExtension will always be at least 7/6 times the value of SmallSize.

The result of the sorting of the small buffer is written to the 
`large\index{large buffer} buffer\index{buffer!large}' (with the size {\sl 
LargeSize\index{largesize}}) as a single object and the filling of the 
small buffer can resume. Whenever there is not enough room in the large 
buffer for the result of sorting the small buffer, or whenever there are 
already a given number of these sorted `patches' in it (controlled by the 
variable {\sl LargePatches\index{largepatches}}) the buffer will be sorted 
by merging the patches\index{patch} to make room for the new results. The 
output is written to the sort file as a single patch. Then the results from 
the small buffer can be written to the large buffer. This game can continue 
till no more terms are generated. In the end it will be necessary to sort 
the results in the intermediate sort file\index{file!sort}. This can be 
done with up to {\sl FilePatches\index{filepatches}} at a time. Because 
file operations are notoriously slow the combination of the small buffer, 
the small extension and the large buffer is used for caching\index{cache} 
purposes. Hence this space can be split in `FilePatches' caches. The 
limitation is that each cache should be capable to contain at least two 
terms of maximal size. This means that the sum of SmallExtension and 
LargeSize must be at least FilePatches times 2*MaxTermSize*(bytes in short 
integer). It is possible to set the size of these caches directly with the 
variable {\sl SortIOsize\index{sortiosize}}. If the variable is too large, 
the variable FilePatches may be adjusted by FORM. If there are more than 
FilePatches patches in the sort file, a second sort file is needed for the 
output of each `superpatch'\index{superpatch}. When the first sort file has 
been treated, the second sort file can be treated in exactly the same way 
as its predecessor. This process will finish eventually. When there are at 
most FilePatches patches in a sort file, the output of their merging can be 
written directly to the regular output. For completeness we give a list of 
all these variables:

\leftvitem{3cm}{FilePatches\index{setup!filepatches}\index{filepatches}}
\rightvitem{13cm}{The maximum number of patches that can be merged 
simultaneously, when the intermediate sort file is involved.}

\leftvitem{3cm}{LargePatches\index{setup!largepatches}\index{largepatches}}
\rightvitem{13cm}{The maximum number of patches that is allowed in the 
large buffer. The large buffer may reside in virtual memory, due to the 
nature of the sort that is applied to it.}

\leftvitem{3cm}{TermsInSmall\index{setup!termsinsmall}\index{termsinsmall}}
\rightvitem{13cm}{The maximum number of terms that is allowed in the small 
buffer before it is sorted. The sorted result is either copied to the large 
buffer or written to the intermediate sort file (when LargeSize is too 
small).}

\leftvitem{3cm}{SmallSize\index{setup!smallsize}\index{smallsize}}
\rightvitem{13cm}{The size of the small buffer in bytes.}

\leftvitem{3cm}{SmallExtension\index{setup!smallextension}\index{smallextension}}
\rightvitem{13cm}{The size of the small buffer plus its extension.}

\leftvitem{3cm}{LargeSize\index{setup!largesize}\index{largesize}}
\rightvitem{13cm}{The size of the large buffer.}

\leftvitem{3cm}{SortIOsize\index{setup!sortiosize}\index{sortiosize}}
\rightvitem{13cm}{The size of the buffer that is used to write to the 
intermediate sorting file and to read from it. It should be noted that if 
this buffer is not very large, the sorting of large files may become rather 
slow, depending on the operating system. Hence we recommend a potential 
fourth stage in the sorting over having this number too small to fit more 
filepatches in the combined small and large buffer. Setting the small and 
large buffers to a decent size may avoid all problems by a: making more 
space for the caching, b: creating fewer file patches to start with.}

There is a second set of the above setup parameters for sorts of 
subexpressions\index{subexpressions} as in function arguments or in the 
term environment (see \ref{substaterm}). Because these things can happen 
with more than one level, whatever allocations have to be made (during 
runtime when needed) may have to be made several times. Hence one should be 
far more conservative here than with the global allocations. Anyway, those 
sorts should rarely involve anything very big. With the function arguments 
the condition is that the final result will fit inside a single term, but 
with the term environment no such restriction exists. The relevant 
variables here are subfilepatches, sublargepatches, sublargesize, 
subsmallextension, subsmallsize, subsortiosize and subtermsinsmall. Their 
meanings are the same as for the variables without the sub in front.

When FORM is running in parallel mode (either TFORM or ParFORM) each worker 
will need its own buffers. In ParFORM in which the processors each control 
their own memory, the size of each of these buffers are the same as for the 
master process. In TFORM with its shared memory the above sizes refer to 
the buffers of the master thread. The workers each get basically buffers 
with 1/N times the size of the buffer of the master. This may get made a 
bit bigger when potential conflicts with MaxTermSize occur.

The default settings are
\begin{center}
\begin{tabular}{lrr}
Variable         &      32-bits       &    64-bits   \\ \hline
bracketindexsize &      200000        & 200000 \\
commentchar &           $*$           & $*$ \\
compresssize &          90000         & 90000 \\
constindex &            128           & 128 \\
continuationlines &     15            & 15 \\
dotchar &               .             & . \\
filepatches &           256           & 256 \\
functionlevels &        30            & 30 \\
hidesize &              50000000      & 50000000 \\
incdir &                .             & . \\
%indentspace &                        & \\
insidefirst &           ON            & ON \\
largepatches &          256           & 256 \\
largesize &             50000000      & 50000000 \\
maxnumbersize &         200           & 200 \\
maxtermsize &           10000         & 40000 \\
maxwildcards &          100           & 100 \\
nospacesinnumbers &     OFF           & OFF \\
numstorecaches &        4             & 4 \\
nwritefinalstatistics & OFF           & OFF \\
nwritestatistics &      OFF           & OFF \\
nwritethreadstatistics &OFF           & OFF \\
oldorder &              OFF           & OFF \\
parentheses &           100           & 100 \\
path &                  .             & . \\
%polygcdchoice &        0             & 0 \\
scratchsize &           50000000      & 50000000 \\
sizestorecache &        32768         & 32768 \\
slavepatchsize &        10            & 10 \\
smallextension &        20000000      & 20000000 \\
smallsize &             10000000      & 10000000 \\
sortiosize &            100000        & 100000 \\
sorttype &              lowfirst      & lowfirst \\
subfilepatches &        64            & 64 \\
sublargepatches &       64            & 64 \\
sublargesize &          4000000       & 4000000 \\
subsmallextension &     800000        & 800000 \\
subsmallsize &          500000        & 500000 \\
subsortiosize &         32768         & 32768 \\
subtermsinsmall &       10000         & 10000 \\
tempdir &               .             & . \\
termsinsmall &          100000        & 100000 \\
threadbucketsize &      500           & 500 \\
threadloadbalancing &   ON            & ON \\
threads &               0             & 0 \\
threadsortfilesynch &   OFF           & OFF \\
threadscratchoutsize &  2500000       & 2500000 \\
threadscratchsize &     100000        & 100000 \\
workspace &             1000000       & 2000000
%zipsize &               32768        & 32768
\end{tabular}
\end{center}
If one compares these numbers with the corresponding numbers for older 
versions one will notice that here we assume that the standard computer 
will have much more memory available than in the `old time'. Basically we 
expect that a serious FORM user has at least 64 Mbytes available. If it is 
considerably less one should define a setup file with smaller settings.

More recently a new notation for large numbers has been allowed. One can 
use the characters K, M and G to indicate kilo (three zeroes), mega (6 
zeroes) and giga (9 zeroes) as in 10M for 10000000.

To find out what the setup values are, one can use the `ON,setup;' 
statement (\ref{substaon});

In version 3.3 and later one may use environment\index{environment} 
variables for the values of the setup parameters, either in the setup file 
or at the beginning of the .frm file. The environment variable is used as a 
preprocessor variable in the sense that its name is enclosed in a 
backquote-quote pair as in \verb:`VARNAME':. The variable will be looked 
for and if found it will be substituted. This can however not be done in a 
recursive way, because the regular routines that take care of the 
preprocessor variables are not active yet when the setups are read.
