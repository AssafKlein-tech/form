
\chapter{The TableBase}
\label{tablebase}

The tablebase\index{tablebase} statement controls a 
database\index{database}-like structure that allows \FORM\ to control massive 
amounts of data in the form of tables and
table\index{table elements}\index{table} elements.
The contents of a tablebase are formed by one or more table declarations 
and a number of fill\index{fill} statements. These fill statements however 
are not immediately compiled. For each fill statement a special fill 
statement is generated and compiled that is of the form
\begin{verbatim}
    Fill tablename(indices) = tbl_(tablename,indices,arguments);
\end{verbatim}
The function tbl\_\index{tbl\_} is a special function to make a temporary 
table substitution. It indicates that the corresponding element can be 
found in a tablebase that has been opened. At a later stage one can tell 
\FORM\ to see which table elements are actually needed and then only those 
will be loaded from the tablebase and compiled.

Tablebases have a special internal structure and the right hand sides of 
the fill statements are actually stored in a compressed\index{compressed} 
state. These tablebases can be created with special statements and uploaded 
with any previously compiled table. Hence one can prepare a tablebase in a 
previous job, to be used at a later stage, without the time penalty of 
loading the whole table at that later stage.

Assume we have a file named no11fill.h that loooks like
\begin{verbatim}
    Symbols ...;
    Table,sparse,no11fill(11,N?);
    Fill no11fill(-3,1,1,1,1,1,1,1,0,0,0) = ....
    Fill no11fill(-2,1,1,1,1,1,1,1,0,0,0) = ....
    etc.
\end{verbatim}
It should be noted that only sparse\index{sparse} tables can be stored 
inside a tablebase. The right hand sides could be typically a few kilobytes 
of formulas and there could be a few thousand of these fill statements. To 
make this into a tablebase one would use the program
\begin{verbatim}
    #-
    #include no11fill.h
    #+
    TableBase "no11.tbl" create;
    TableBase "no11.tbl" addto no11fill;
    .end
\end{verbatim}
The include\index{\#include} instruction makes that \FORM\ reads and compiles 
the table. Then the first tablebase statement creates a new tablebase file 
by the name no11.tbl. If such a file existed already, the old version will 
be lost. If one would like to add to an existing tablebase, one should use 
the `open'\index{open} keyword. The second tablebase statement adds the 
table no11fill to the tablebase file no11.tbl. This takes care of declaring 
the table, making an index of all elements that have been filled and 
putting their right hand sides, in compressed form, into the tablebase. The 
compression is based on the zlib\index{zlib} library, provided by Jean-loup 
Gailly\index{Gailly!Jean-loup} and Mark Adler\index{Adler!Mark} (version 
1.2.3, July 18, 2005) and it strikes a nice balance between speed and 
compression ratio.

The tablebase can be loaded in a different program as in
\begin{verbatim}
    TableBase "no11.tbl" open;
\end{verbatim}
This loads the main index\index{index!main} of the file into memory.

If one would like to compile the short version of the fill statements (the 
normal action at this point) one needs to use the load\index{load} option. 
Without any names of tables it will read the index of all tables. If tables 
are specied, only the index of those tables is taken and the proper 
tbl\_ fill statements are generated:
\begin{verbatim}
    TableBase "no11.tbl" open;
    TableBase "no11.tbl" load no11fill;
\end{verbatim}

If one would like to compile\index{compile} the complete tables, rather 
than just the shortened versions, one can use the enter option as in:
\begin{verbatim}
    TableBase "no11.tbl" open;
    TableBase "no11.tbl" enter no11fill;
\end{verbatim}

Let us assume we used the load option. Hence now an occurrence of a table 
element will be replaced by the stub\index{stub function}-function 
tbl\_\index{tbl\_}. In order to have this replaced by the actual right hand 
side of the original fill statement we have to do some more work. At a 
given moment we have to make \FORM\ look which elements are actually needed. 
This is done with the TestUse\index{testuse} statement as in
\begin{verbatim}
    TestUse no11fill;
\end{verbatim}
This does nothing visible. It just marks internally which elements will be 
needed and have not been entered yet.

The actual entering of the needed elements is done with the use\index{use} 
option:\begin{verbatim}
    TableBase "no11.tbl" use;
\end{verbatim}
If many elements are needed, this statement may need some compilation time. 
Note however that this is time at a moment that it is clear that the 
elements are needed, which is entirely different from a fixed time at the 
startup of a program when the whole table is loaded as would have to be 
done before the tablebase statement existed. Usually however only a 
part of the table is needed, and in the extreme case only one or two 
elements. In that case the profit is obvious.

At this point the proper elements are available inside the system, but 
because we have two versions of the table (one the short version with 
tbl\_, the other the complete elements) we have to tell \FORM\ to apply 
the proper definitions with the `apply'\index{apply} statement.
\begin{verbatim}
    Apply;
\end{verbatim}
Now the actual rhs will be inserted.

One may wonder why this has to be done in such a `slow' way with this much 
control over the process. The point is that at the moment the table 
elements are recognized, one may not want the rhs yet, because it may be 
many lines. Yet one may want to take the elements away from the main stream 
of action. Similarly, having a table element recognized at a certain stage, 
may not mean automatically that it will be needed. The coefficient may 
still become zero during additional manipulations. Hence the user is left 
with full control over the process, even though that may lead to slightly 
more programming. It will allow for the fastest program.

For the name of a tablebase we advise the use of the extension 
.tbl\index{.tbl} to avoid confusion.

Note that the above scheme may need several applications, if table elements 
refer in their definition to other table elements. This can be done with a 
construction like:
\begin{verbatim}
    #do i = 1,1
        TestUse;
        .sort
        TableBase "basename.tbl" use;
        Apply;
        if ( count(tbl_,1) ) Redefine i "0";
        .sort
    #enddo
\end{verbatim}
It will stay in the loop until there are no more tbl\_ functions to be 
resolved.

\medskip\noindent The complete syntax (more is planned):

%--#[ addto :

\section{addto}
\label{tbladdto}

\noindent Syntax:

TableBase "file.tbl" addto tablename;

TableBase "file.tbl" addto tablename(tableelement);

\noindent See also open (\ref{tblopen}) and create (\ref{tblcreate}).

\noindent Adds\index{addto} the contents of a (sparse\index{sparse}) table 
to a tablebase. The base must be either an existing tablebase (made 
accessible with an open statement) or a new tablebase (made available with 
a create statement). In the first version what is added is the collection 
of all fill statements that have been used to define elements of the 
indicated table, in addition to a definition of the table (if that had not 
been done yet). In the second version only individual elements of the 
indicated table are added. These elements are indicated as it should be in 
the left hand side of a fill\index{fill} statement.

\noindent One is allowed to specify more than one table, or more than one 
element. If one likes to specify anything after an element, it should be 
realized that one needs to use a comma for a separator, because blank 
spaces after a parenthesis are seen as irrelevant.

\noindent Examples:
\begin{verbatim}
    TableBase "no11.tbl" open;
    TableBase "no11.tbl" load;
    TableBase "no11.tbl" addto no11filb;
    TableBase "no11.tbl" addto no11fill(-3,1,1,1,1,2,1,1,0,0,0),
                               no11fill(-2,1,1,2,1,1,1,1,0,0,0);
\end{verbatim}

%--#] addto :
%--#[ apply :

\section{apply}
\label{tblapply}

\noindent Syntax:

Apply [number] [tablename(s)];

\noindent See also testuse (\ref{tbltestuse}) and use (\ref{tbluse}).

\noindent The actual application\index{apply} of fill\index{fill} 
statements that were taken from the tablebases. If no tables are specified, 
this is done for all tables, otherwise only for the tables whose names are 
mentioned. The elements must have been registered as used before with the 
application of a testuse\index{testuse} statement, and they must have been 
compiled from the tablebase with the use\index{use} option of the tablebase 
statement. The number refers to the maximum number of table elements that 
can be substituted in each term. This way one can choose to replace only 
one element at a time. If no number is present all occurrences will be 
replaced. This refers also to occurrences inside function arguments. If only 
a limited number is specified in the apply statement, the occurrences 
inside function arguments have priority.

%--#] apply :
%--#[ audit :

\section{audit}
\label{tblaudit}

\noindent Syntax:

TableBase "file.tbl" audit;

\noindent See also open (\ref{tblopen})

\noindent Prints\index{audit} a list of all tables and table elements that 
are defined in the specified tablebase. This tablebase needs to be opened 
first. As of the moment there are no options for the audit. Future options 
might include formatting of the output.

%--#] audit :
%--#[ create :

\section{create}
\label{tblcreate}

\noindent Syntax:

TableBase "file.tbl" create;

\noindent See also open (\ref{tblopen})

\noindent This creates\index{create} a new file\index{file!new} with the 
indicated name. This file will be initialized as a tablebase. If there was 
already a file with the given name, its old contents will be lost. If one 
would like to add to an existing tablebase, one should use the 
`open'\index{open} option.

%--#] create :
%--#[ enter :

\section{enter}
\label{tblenter}

\noindent Syntax:

TableBase "file.tbl" enter;

TableBase "file.tbl" enter tablename(s);

\noindent See also open (\ref{tblenter}) and load (\ref{tblload}).

\noindent Scans\index{enter} the specified tablebase and (in the first 
variety) creates for all elements of all tables in the tablebase a 
fill\index{fill} statement with its full contents. This is at times faster 
than reading the fill statements from a regular input 
file\index{file!input}, because the tablebase has its contents 
compressed\index{compress}. 
Hence this costs less file access time. When table names are specified, 
only the tables that are mentioned have their elements treated this way.

\noindent The tablebase must of course be open for its contents to be 
available.

\noindent If one would like \FORM\ to only see what elements are available 
and load that information one should use the load\index{load} option.

%--#] enter :
%--#[ load :

\section{load}
\label{tblload}

\noindent Syntax:

TableBase "file.tbl" load;

TableBase "file.tbl" load tablename(s);

\noindent See also open (\ref{tblopen}) and enter (\ref{tblenter}).

\noindent Scans\index{load} the index of the specified tablebase and (in 
the first variety) creates for all elements of all tables in the tablebase 
a fill\index{fill} statement of the type
\begin{verbatim}
    Fill tablename(indices) = tbl_(tablename,indices,arguments);
\end{verbatim}
This is the fill statement that will be used when elements of one of these 
tables are encountered. The function tbl\_ is called the (table)stub 
function. When table names are specified, only the tables that are 
mentioned have their elements treated this way.

\noindent The tablebase must of course be open for its contents to be 
available.

\noindent If one would like to actually load the complete fill statements, 
one should use the enter option.

%--#] load :
%--#[ off :

\section{off}
\label{tbloff}

\noindent Syntax:

TableBase "file.tbl" off subkey;

\noindent See also addto (\ref{tbladdto}) and off (\ref{tblon}).

\noindent Currently\index{off} only the subkey `compress'\index{compress} 
is recognized. It makes sure that no compression is used when elements are 
being stored in a tablebase with the addto\index{addto} option. This could 
be interesting when the right hand sides of the fill statements are 
relatively short.

%--#] off :
%--#[ on :

\section{on}
\label{tblon}

\noindent Syntax:

TableBase "file.tbl" on subkey;

\noindent See also addto (\ref{tbladdto}) and off (\ref{tbloff}).

\noindent Currently\index{on} only the subkey `compress'\index{compress} is 
recognized. It makes sure that compression with the gzip\index{gzip} 
algorithms is used when elements are being stored in a tablebase with the 
addto\index{addto} option. This is the default.

%--#] on :
%--#[ open :

\section{open}
\label{tblopen}

\noindent Syntax:

TableBase "file.tbl" open;

\noindent See also create (\ref{tblcreate})

\noindent This opens\index{open} an existing file with the indicated name. 
It is assumed that the file has been created\index{create} with the 
`create' option in a previous \FORM\ program. It gives the user access to the 
contents of the tablebase. In addition it allows the user to add to its 
contents.

\noindent Just like with other files, \FORM\ will look for the file in in 
current directory and in all other directories mentioned in the environment 
variable `FORMPATH'\index{FORMPATH} (see for instance the 
\#call\index{\#call} (\ref{precall}) and the \#include\index{\#include} 
(\ref{preinclude}) instructions).

%--#] open :
%--#[ testuse :

\section{testuse}
\label{tbltestuse}

\noindent Syntax:

TestUse;

TestUse tablename(s);

\noindent See also use (\ref{tbluse}).

\noindent Tests\index{testuse} for all elements of the specified tables (if 
no tables are mentioned, this is done for all tables) whether they are used 
in a stub\index{stub} function tbl\_\index{tbl\_}. If so, this indicates 
that these elements must be compiled from a tablebase, provided this has 
not been done already. The compilation will have to be done at a time, 
specified by the user. This can be done with the use\index{use} option. All 
this statement does is set some flags in the internals of \FORM\ for the 
table elements that are encountered in the currently active expressions.

%--#] testuse :
%--#[ use :

\section{use}
\label{tbluse}

\noindent Syntax:

TableBase "file.tbl" use;

TableBase "file.tbl" use tablename(s);

\noindent See also testuse (\ref{tbltestuse}) and apply (\ref{tblapply}).

\noindent Causes\index{use} those elements of the specified tables to be 
compiled, that a previous testuse\index{testuse} statement has encountered 
and that have not yet been compiled before. If no tables are mentioned this 
is done for all tables. The right hand sides of the definition of the table 
elements will not yet be substituted. That is done with an 
apply\index{apply} statement.

%--#] use :

