
\chapter{Modules}
\label{modules}

Modules\index{module} are the basic execution\index{execution} blocks. 
Statements\index{statements} are always part of a module, and they will be 
executed only when the module is executed. This is directly opposite to 
preprocessor instructions which are executed when they are encountered in 
the input stream.

Modules are terminated by a line that starts with a period\index{period}. 
Such a line is called the module\index{module instruction} instruction. 
Once the module instruction has been recognized, the compilation of the 
module is terminated and the module will be executed. All active 
expressions will be processed one by one, term by term. When each term of 
an expression has been through all statements of the module, the combined 
results of all operations on all the terms of the expression will be sorted 
and the resulting expression will be sent to the output. This can be an 
intermediate file\index{file!intermediate}, or it can be some 
memory\index{memory}, depending on the size of the output. If the combined 
output of all active expressions is less than the parameter 
``ScratchSize''\index{ScratchSize}, the results stay in memory. ScratchSize 
is one of the setup parameters (see chapter \ref{setup}).

A module consists in general of several types of statements:
\begin{description}
\item [Declarations\index{declarations}] These are the declarations of 
variables.
\item [Specifications\index{specifications}] These tell what to do with 
existing expressions as a whole.
\item [Definitions\index{definitions}] These define new expressions.
\item [Executable\index{executable statements} statements] The operations 
on all active expressions.
\item [OutputSpecifications\index{output specifications}] These specify the 
output representation.
\item [End-of-module specifications\index{end of module specifications}] 
Extra settings that are for this module only.
\item [Mixed statements\index{mixed statements}] They can occur in various 
classes. Most notably the print statement.
\end{description}
Statements must occur in such an order that no statement follows a 
statement of a later category. The only exception is formed by the mixed 
statements, which can occur anywhere. This is different from earlier 
versions of \FORM\ in which the order of the statements was not fixed. This 
did cause a certain amount of confusion about the workings of \FORM\@.

There are several types of modules.
\begin{description}
\item[.sort\index{.sort}] \label{instrsort} The general end-of-module. 
Causes execution of all active expressions, and prepares them for the next 
module.
\item[.end\index{.end}] \label{instrend} Executes all active expressions 
and terminates the program.
\item[.store\index{.store}] \label{instrstore} Executes all active 
expressions. Then it writes all active global expressions to an 
intermediate storage file\index{file!storage} and removes all other 
non-global expressions. Removes all memory of declarations except for those 
that were made before a .global instruction.
\item[.global\index{.global}] \label{instrglobal} No execution of 
expressions. It just saves declarations made thus far from being erased by 
a .store instruction.
\item[.clear\index{.clear}] \label{instrclear} Executes all active 
expressions. Then it clears all buffers with the exception of the main 
input stream. Continues execution in the main input stream as if the 
program had started at this point. The only parameters that cannot be 
changed at this point are the setup parameters. They remain. By default 
also the clock\index{clock} is reset. If this is not desired this can be 
changed by means of the ResetTimeOnClear\index{resettimeonclear} setup 
variable (see chapter \ref{setup}).
\end{description}
Each program must be terminated by a .end instruction. If such an 
instruction is absent and \FORM\ encounters an end-of-input it will issue a 
warning and generate a .end instruction.

Module instructions can contain a special commentary that will be printed 
in all statistics that are generated during the execution of the module. 
This special commentary is restricted to 24 characters (the statistics have 
a fixed format and hence there is only a limited amount of space 
available). This commentary is initiated by a colon and terminated by a 
semicolon. The characters between this colon and the semicolon are the 
special message, also called advertisement. Example
\begin{verbatim}
	.sort:Eliminate x;
\end{verbatim}
would give in the statistics something like
\begin{verbatim}
Time =       0.46 sec    Generated terms =        360
                F        Terms in output =        360
            Eliminate x  Bytes used      =       4506
\end{verbatim}
If the statistics are switched off, there will be no printing of this 
advertisement either.

For backwards compatibility there is still an obsolete\index{obsolete} 
mechanism to pass module options via the module instructions. This is a 
feature which will probably disappear in future versions of \FORM\@. We do 
give the syntax to allow the user to identify the option properly and 
enable proper translation into the moduleoption\index{moduleoption} 
statement (see \ref{substamoduleoption}).
\begin{verbatim}
    .sort(PolyFun=functionname);
    .sort(PolyFun=functionname):advertisement;
\end{verbatim}
causes the given function to be treated as a polynomial\index{polyfun} 
function. This means that its (single) argument would be treated as the 
coefficient of the terms. The action of \FORM\ on individual terms is
\begin{enumerate}
\item Ignore polynomial functions with more than one argument.
\item If there is no polynomial function with a single argument, generate 
one with the argument 1.\item If there is more than one polynomial function 
with a single argument, multiply the arguments and replace these functions 
with a single polynomial function with the product of the arguments for a 
single argument.
\item Multiply the argument of the polynomial function with the coefficient 
of the term. Replace the coefficient itself by one.
\end{enumerate}
If, after this, two terms differ only in the argument of their polynomial 
function \FORM\ will add the arguments and replace the two terms by a single 
term which is identical to the two previous terms except for that the 
argument of its polynomial function is the sum of their two arguments.

It should be noted that the proper placement of .sort\index{.sort} 
instructions in a \FORM\ program is an art by itself. Too many .sort 
instructions cause too much sorting, which can slow execution down 
considerably. It can also cause the writing of intermediate expressions 
which are much larger than necessary, if the next statements would cause 
great simplifications. Not enough .sort instructions can make that 
cancellations are postponed unnecessarily and hence much work will be done 
double. This can slow down execution by a big factor. First an example of a 
superfluous .sort:
\begin{verbatim}
    S	a1,...,a7;
    L	F = (a1+...+a7)^16;
    .sort

Time =      31.98 sec    Generated terms =      74613
                F        Terms in output =      74613
                         Bytes used      =    1904316
    id	a7 = a1+a2+a3;
    .end

Time =     290.34 sec
                F        Terms active    =      87027
                         Bytes used      =    2253572

Time =     295.20 sec    Generated terms =     735471
                F        Terms in output =      20349
                         Bytes used      =     538884
\end{verbatim}
Without the sort the same program gives:
\begin{verbatim}
    S	a1,...,a7;
    L	F = (a1+...+a7)^16;
    id	a7 = a1+a2+a3;
    .end

Time =     262.79 sec
                F        Terms active    =      94372
                         Bytes used      =    2643640

Time =     267.81 sec    Generated terms =     735471
                F        Terms in output =      20349
                         Bytes used      =     538884
\end{verbatim}
and we see that the sorting in the beginning is nearly completely wasted. 
Now a clear example of not enough .sort instructions. A common problem is 
the substitution of one power\index{power series} series into another. If 
one does this in one step one could have:
\begin{verbatim}
    #define MAX "36"
    S  j,x(:`MAX'),y(:`MAX');
    *
    *	Power series expansion of ln_(1+x)
    *
    L	F = -sum_(j,1,`MAX',sign_(j)*x^j/j);
    *
    *	Substitute the expansion of x = exp_(y)-1
    *
    id	x = x*y;
    #do j = 2,`MAX'+1
    id	x = 1+x*y/`j';
    #enddo
    Print;
    .end

Time =      76.84 sec    Generated terms =      99132
                F        Terms in output =          1
                         Bytes used      =         18

   F =
      y;
\end{verbatim}
With an extra .sort inside the loop one obtains for the same program (after 
suppressing some of the statistics:
\begin{verbatim}
    #define MAX "36"
    S  j,x(:`MAX'),y(:`MAX');
    *
    *	Power series expansion of ln_(1+x)
    *
    L	F = -sum_(j,1,`MAX',sign_(j)*x^j/j);
    *
    *	Substitute the expansion of x = exp_(y)-1
    *
    id	x = x*y;
    #do j = 2,`MAX'+1
    id	x = 1+x*y/`j';
    .sort: step `j';

Time =       0.46 sec    Generated terms =        360
                F        Terms in output =        360
                 step 2  Bytes used      =       4506
    #enddo
           .
           .
           .
Time =       3.07 sec    Generated terms =          3
                F        Terms in output =          1
                step 37  Bytes used      =         18
    Print;
    .end

Time =       3.07 sec    Generated terms =          1
                F        Terms in output =          1
                         Bytes used      =         18

   F =
      y;
\end{verbatim}
It is very hard to give general rules that are more specific than what has 
been said above. The user should experiment with the placements of the .sort 
before making a very large run. 

\section{Checkpoints}
\label{checkpoints}

If\index{checkpoints} \FORM\ programs have to run for a long time, the 
reliability of the hardware(computer system or network) or of the software 
infrastructure becomes a critical issue. Program 
termination\index{termination} due to unforeseen failures may waste days or 
weeks of invested execution time. The checkpoint mechanism was introduced 
to protect long running \FORM\ programs as good as possible from such 
accidental interruptions. With activated checkpoints \FORM\ will save its 
internal state and data from time to time on the hard disk. This data then 
allows a recovery from a crash\index{crash}.

The checkpoint mechanism can be activated or deactivated by {\tt 
On}\index{on} and {\tt Off}\index{off} statements. If the user has 
activated checkpoints, recovery\index{recovery} data will be written to disk 
at the end of a module execution. Options allow to influence the details of 
the saving mechanism. If a program is terminated during execution, \FORM\ can 
be restarted with the {\tt -R} option and it will continue its execution at 
the last saved recovery point.

The syntax of the checkpoint activation and deactivation is
\begin{verbatim}
    On checkpoint [<OPTIONS>];
    Off checkpoint;
\end{verbatim}

If no options are given, the recovery data will be saved at the end of every
module\index{module}. If one gives a time\index{time}
\begin{verbatim}
    On checkpoint <NUMBER>[<UNIT>];
\end{verbatim}
the saving will only be done if the given time has passed after the last 
saving. Possible unit specifiers are {\tt s, m, h, d} and the number will 
then be interpreted as seconds, minutes, hours, or days, respectively. The 
default unit is seconds.

If one needs to run a script\index{run a script} before or after the saving,
one can specify a script filename.
\begin{verbatim}
    On checkpoint runbefore="<SCRIPTFILENAME>";
    On checkpoint runafter="<SCRIPTFILENAME>";
    On checkpoint run="<SCRIPTFILENAME>";
\end{verbatim}
The option {\tt run}\index{run} sets both the scripts to be run before and 
after saving.The scripts must have the executable flag set and they must 
reside in the execution path of the shell\index{shell} (unless the filename 
already contains the proper path).

The scripts receive the module number\index{module number} as an argument 
(accessible as \$1 inside the script). The return value of the script 
running before the saving will be interpreted. If the script returns an 
error (non-zero return value), a message will be issued and the saving will 
be skipped. 

The recovery data will be written to files named {\tt FORMrecv.*} with 
various name extensions. If a file {\tt FORMrecv.tmp} exists, \FORM\ will not 
run unless one gives it the recovery option\index{recovery option}
{\tt -R}. This is to prevent the unintentional loss of recovery data. If 
\FORM\ terminates successfully, all the additional data files will be removed.

The additional recovery files will be created in the directory containing 
the scratch files.  The extra files will occupy roughly as much space as 
the scratch files\index{scratch files} and the save\index{save files} and 
hide files\index{hide files} combined. This extra space must be made 
available, of course.

If recovery data exists and \FORM\ is started with the {\tt -R} option, \FORM\ 
will continue execution after the last module that successfully wrote the 
recovery data. All the command line parameters that have been given to the 
crashed \FORM\ program\index{crashed \FORM\ program} must also be given to the 
recovering \FORM\ program. The input files are not part of the recovery data 
and will be read in anew when recovering. Therefore it is strongly 
discouraged to change any of these files between saving and recovery.
